\documentclass[../main.tex]{subfiles}
\begin{document}


\section{Abstract}\label{abstract-adjusting-size}

\chap{chap:adjusting_opacity} provided strong evidence for the effects
of systematically varying the opacities of scatterplot points on
participants' estimates of correlation in positively correlated
scatterplots. Utilising the same function and experimental paradigm, I
show in a single experiment that systematically varying the sizes of
scatterplot points is able to bias participants' estimates of
correlation to a greater degree than manipulations that only adjust
point opacity. In a condition where point size decreases non-linearly as
a function of residual distance, correlation estimation is significantly
biased upwards to correct for a historic underestimation bias to a
greater degree. I discuss the implications of these findings for the
mechanisms behind both opacity and size adjustments in scatterplots in
relation to correlation estimation, and recommend techniques for those
who design with the estimation of positive correlation in mind.

\section{Introduction}\label{introduction-adjusting-size}

While I was successful at changing participants' perceptions of
correlation in positively correlated scatterplots in
\chap{chap:adjusting_opacity}, the extent to which these perceptions
were changed was minimal. Figure \ref{fig-estimates-by-r-e2} illustrates
how participants' mean errors in \emph{r} estimation changed as a
function of the subjective \emph{r} value. Scatterplots employing
non-linear opacity decay produced the most drastic changes in
correlation estimation, however this was still small, with an effect
size of Cohen's \emph{d} = 0.19. Recent evidence suggests that with
regards to altering percepts in scatterplots, changes in point size may
be more effective than changes in opacity. In a fully-reproducible,
large sample (N = 150) study, I show that systematically altering point
size using the same function is not only able to more effectively
correct for the correlation underestimation bias, but is also able to
alter the shape of the correlation estimation curve.

\section{Related Work}\label{related-work-adjusting-size}

\subsection{Size and Perception}\label{size-and-perception}

\subsection{Scatterplot Point Size and Correlation
Perception}\label{scatterplot-point-size-and-correlation-perception}

\section{Methods}\label{methods-e3}

\subsection{Stimuli}\label{stimuli-e3}

The creation of stimuli in this experiment follows the same general
principles outlined in Section \chap{creating-stimuli},
\chap{chap:gen_methods}. As in \chap{chap:adjusting_opacity}, equation 1
was used to map point residuals to size values in the two non-linear
decay conditions:

\begin{equation}
  point_{size/opacity} = 1 - b^{residual}
\end{equation}

As in \chap{chap:adjusting_opacity}, the use of this equation produces a
curve around the identity line symmetrically opposing the
underestimation curve found in previous work. Additionally, a constant
of 0.2 was added to each raw size value, and a scaling factor of 4 was
utilised; these adjustments resulted in the smallest points in the
present experiment having a width of 12 pixels on a 1920x1080 pixel
monitor, which is consistent with the point size used in the experiments
described in \chap{chap:adjusting_opacity}. Examples of the stimuli used
in this experiment can be see in Figure~\ref{fig-exp3-examples-chap5}.

\begin{figure}

\centering{

\includegraphics[width=\textwidth]{5_adjusting_size_files/figure-latex/fig-exp3-examples-chap5-1.pdf}

}

\caption{\label{fig-exp3-examples-chap5}Examples of the stimuli used in
experiment 3, demonstrated with an \textit{r} value of 0.6.}

\end{figure}%

\subsection{Dot Pitch in Crowdsourced
Experiments}\label{dot-pitch-chap5}

When the experiments described in \chap{chap:adjusting_opacity} took
place, no method of obtaining dot pitch was implemented. Dot pitch is
defined as the distance between the dots (sub-pixels)
\cite{castellano_1992} that make up each pixel. Calculating dot pitch is
a requirement for the subsequent calculation of the physical on-screen
sizes of the scatterplot points that participants saw. In the preamble
to the current experiment, participants were asked to hold a standard
size credit/debit/ID card up to the monitor, and then to resize a
corresponding on-screen image until it matched the size of their
physical card \cite{screenscale}. These cards have a universal standard
size (ISO/IEC 7810 ID-1), which when combined with the monitor
resolution information recorded by Psychopy, and assuming a widescreen
16:9 aspect ratio, allows for the inference of dot pitch and therefore
the physical size of the points in the experience. Mean dot pitch was
0.33mm (\(SD = 0.06\)), corresponding to a physical size on the screen
of 4.32mm for the smallest points displayed. Section \ref{results-e3}
includes analysis that takes into account the physical on-screen sizes
of scatterplot points.

\subsection{Point Visibility Testing}\label{point-visibility-testing}

It is key that the manipulations used do not remove (or appear to
remove) data from scatterplots. Therefore, point visibility testing is
included in this experiment prior to the experimental items.
Participants were shown 6 scatterplots and were asked to enter in a text
box how many points were being displayed. These points were the same
size as the smallest points displayed in the experimental items. 5\% of
participants were correct on 5 out of 6 point visibility tests, while
95\% were correct on 6 out of 6. It should be noted that those
participants scoring 5/6 did not answer incorrectly, rather they did not
answer at all for a particular question, which is suggestive of a
mis-click or an initial misunderstanding of the task. Regardless, the
results of this test indicate a sufficient level of point visibility.

\subsection{Design}\label{design-e3}

Again, a fully repeated-measures, within-participants design was
employed. Each participant saw and responding to each of the 180
scatterplots in a fully randomised order. There were four scatterplots
for each of the 45 \emph{r} values corresponding to the four levels of
the size decay condition, examples of which are shown in
Figure~\ref{fig-exp3-examples-chap5}. The experiment itself is hosted on
Pavlovia \footnote{https://gitlab.pavlovia.org/Strain/exp\_size\_only}.

\subsection{Procedure}\label{procedure-e3}

Each participant viewed the PIS and provided consent through key presses
in response to consent statements. Participants were asked to provide
their age in a free text box, followed by their gender identity.
Participants then completed the 5-item Subjective Graph Literacy test
\cite{garcia_2016}, followed by the screen scale and point visibility
tasks described above. Participants were shown example of scatterplots
depicting \emph{r} values of 0.2, 0.5, 0.8, and 0.95. Section
\ref{results-e3} contains discussion of the potential effects of this
training. Following two practice trials, participants worked through the
series of 180 experimental and six attention check trials in a
randomised order. Visual masks preceded each plot.

\subsection{Participants}\label{participants-e3}

Normal to corrected-to-normal vision and English fluency were required.
Participants who had completed any of the experiments described in
\chap{chap:adjusting_opacity} were prevented from participating. Data
were collected from 164 participants. 14 failed more than 2 out of 6
attention check questions, and, as per the pre-registration, had their
submissions rejected from the study. The data from the remaining 150
participants were included in the full analysis (48\% male, 50 \%
female, and 2\% non-binary). Participants' mean age was 29.56 (\emph{SD}
= 8.54). Mean graph literacy score was 21.77 (\emph{SD} = 4.29). The
average time taken to complete the experiment was 39 minutes (SD = 14
minutes).

\section{Results}\label{results-e3}

To investigate the effects of point size adjustments on participants'
estimates of correlation, a linear mixed effects model was built whereby
the point size condition is a predictor for the difference between
objective \emph{r} values for each plot and participants' estimates of
\emph{r}. This model has random intercepts for participants and items. A
likelihood ratio test revealed that the model including size decay
function as a fixed effect explained significantly more variance than
the null model (\(\chi^2\)(3) = 3,508.84, \emph{p} \textless{} .001).
Figure~\ref{fig-e3-estimates} shows the mean errors in correlation
estimation for each size decay condition, along with 95\% confidence
intervals.

\begin{figure}

\centering{

\includegraphics[width=\textwidth]{5_adjusting_size_files/figure-latex/fig-e3-estimates-1.pdf}

}

\caption{\label{fig-e3-estimates}Estimated marginal means for the four
conditions tested in experiment 3. 95\% confidence intervals are shown.
The vertical dashed line represents no estimation error. The
overestimation zone is included to facilitate comparison to later work.}

\end{figure}%

\begin{table}

\caption{\label{tbl-contrasts-e3}Contrasts between different levels of
the size decay factor in experiment 3.}

\centering{

\begin{tabular}[t]{llrl}
\toprule
\multicolumn{2}{c}{Contrast} & \multicolumn{2}{c}{Statistics} \\
\cmidrule(l{3pt}r{3pt}){1-2} \cmidrule(l{3pt}r{3pt}){3-4}
  &    & Z ratio & \textit{p}\\
\midrule
Standard Size & Inverted Decay & 9.3 & <0.001\\
Standard Size & Linear Decay & 44.4 & <0.001\\
Standard Size & Non-Linear Decay & 50.1 & <0.001\\
Inverted Decay & Linear Decay & 35.1 & <0.001\\
Inverted Decay & Non-Linear Decay & 40.8 & <0.001\\
\addlinespace
Linear Decay & Non-Linear Decay & 5.7 & <0.001\\
\bottomrule
\end{tabular}

}

\end{table}%

This effect was driven by significant differences between means of
correlation estimation error between all conditions. Statistical tests
for contrasts were performed using the \texttt{emmeans} package
\cite{lenth_2024}, and are shown in Table~\ref{tbl-contrasts-e3}. To
test whether the observed results could be explained by difference in
participants' levels of graph literacy, an additional model was built.
This model is identical to the experimental model, but also includes
graph literacy as a fixed effect. Including graph literacy as a fixed
effect explained no additional variance (\(\chi^2\)(1) = 0.16, \emph{p}
= .690), indicating that the differences observed in participants'
correlation estimation performance were not as a result of differences
in levels of graph literacy.

While participants performed well on the point visibility task, another
facet of using a larger or smaller monitor with a lower or higher
resolution could have affected estimates of correlation. Comparing a
model including the dot pitch of participants' monitors to the
experimental model revealed a significant main effect (\(\chi^2\)(1) =
4.65, \emph{p} = .031). There was no interaction between size decay
condition and dot pitch; a 0.1mm decrease in dot pitch resulted in
correlation estimates decreasing by .03. Given the low range of dot
pitches gathered from participants 0.13mm to 0.63mm, the effect is not
substantial enough to warrant further discussion.

\begin{table}

\caption{\label{tbl-efs-e3}Cohen's \textit{d} effect sizes (left) and
summary statistics (right) for levels of the size decay factor in
experiment 3. Each effect size is compared to the reference level,
termed, ``Standard Size''.}

\begin{minipage}{0.50\linewidth}

\begin{tabular}[t]{lr}
\toprule
Effect & Cohen's \textit{d}\\
\midrule
Standard Size & \\
Inverted Decay & -0.11\\
Linear Decay & -0.54\\
Non-Linear Decay & -0.61\\
\bottomrule
\end{tabular}

\end{minipage}%
%
\begin{minipage}{0.50\linewidth}

\begin{tabular}[t]{lrr}
\toprule
Size & Mean & Standard
Error\\
\midrule
Standard Size & 0.17 & 0.013\\
Inverted Decay & 0.14 & 0.013\\
Linear Decay & 0.04 & 0.013\\
Non-Linear Decay & 0.02 & 0.013\\
\bottomrule
\end{tabular}

\end{minipage}%

\end{table}%

\begin{figure}

\centering{

\includegraphics[width=\textwidth]{5_adjusting_size_files/figure-latex/fig-estimates-by-r-e3-1.pdf}

}

\caption{\label{fig-estimates-by-r-e3}Participants' mean errors in
correlation estimates grouped by factor and by \textit{r} value. The
dashed horizontal line represents perfect estimation. Participants were
most accurate when presented with the plots featuring the non-linear
size decay function. Error bars show standard deviations of estimates.}

\end{figure}%

\section{Discussion}\label{discussion-e3}

Participants' errors in correlation estimation were significantly lower
in the non-linear size decay condition (see
Figure~\ref{fig-estimates-by-r-e3}) compared to all other conditions.
This finding provides support for the first hypothesis. Conversely, no
support was found for the second hypothesis, that estimates would be
least accurate in the inverted non-linear size decay condition. Errors
in this condition were significantly higher than for the other two decay
conditions, but were significantly lower than the errors that were
observed for the standard size condition.

\subsection{Increased Correlation Estimation
Accuracy}\label{increased-correlation-estimation-accuracy}

The mean error in correlation estimation for the non-linear size decay
condition in this experiment was .025, while for the equivalent opacity
decay condition described in experiment 2 (see
\chap{chap:adjusting_opacity}) was .086. This finding provides evidence
that point size is a stronger encoding channel for the manipulation of
perceived contrast than point opacity. If these effects are being driven
by the lower salience of more external points, the fact that a larger
effect of point size has been reported is congruent with research
showing clear influences of stimulus size on object salience and
perceptual weighting \cite{grice_1983, hong_2022, healey_2011}. The
present results therefore provide support for point salience/perceptual
weighting being a key driver of the effects observed; lower point
salience brought on by reduced point size in the exterior of the
scatterplot reduces the perceived width of the distribution of data
points around the regression line, biasing estimates of correlation
upwards and leading to a higher degree of accuracy. Other candidate
mechanisms do exist; similar results would be expected if a
feature-based attentional bias was responsible
\cite{sun_2016, hong_2022}; the current methodology does not allow for
distinguishing between these explanations, and it may be that both are
partially responsible.

The lack of support for the second hypothesis is surprising, and
suggests that point salience and perceived distributional width do not
form the whole story. There is evidence that larger stimuli exhibit
greater levels of spatial uncertainty \cite{alais_2004}, and it is
possible that this uncertainty results in a perceptual under-weighting
of contribution of these points during correlation estimation. This is
consistent with previous work \cite{warren_2002, warren_2004} suggesting
that the brain may make robust statistical use of visuo-spatial
information. These mechanisms act to downweight the influence of less
reliable information (in this case the higher spatial uncertainty of
larger exterior points) on subsequent perceptual estimates. In the
present experiment, this resulted in participants making more accurate
estimates of correlation in the inverted decay condition compared to the
standard size condition. It was suggested in
\chap{chap:adjusting_opacity} that inverted opacity manipulations may be
employed to correct for the \emph{overestimation} of correlation
observed with negatively correlation scatterplots \cite{sher_2017};
findings here indicate that using an inverted size decay function in
this way may not be appropriate.

\subsection{Constant Correlation Estimation
Precision}\label{constant-correlation-estimation-precision}

Unlike the experiments described in \chap{chap:adjusting_opacity}, in
which standard deviations of correlation estimation errors generally
became smaller as the actual \emph{r} value increased, distributions of
standard deviations here remained mostly constant. This was unexpected,
as previous work
\cite{rensink_2010, rensink_2012, rensink_2014, rensink_2017} routinely
finds precision in \emph{r} estimation to increase with the objective
\emph{r} value. This finding may be related to the nature of the stimuli
in the current study. At high values of \emph{r} there is a large amount
of overlap between points in the non-linear, non-linear inverted, and
linear size decay conditions. This overlap may blur the percept and
account for the absence of this effect, however cannot account for these
findings with regards to the standard size condition. Aside from the
inverted non-linear decay condition in \chap{chap:adjusting_opacity},
the finding that precision increased with \emph{r} was robust. Its
absence here is curious given that the standard size decay condition
here is identical to the full opacity condition in that chapter. Relying
on relative judgements means the interplay between scatterplots with
different visual features must be accounted for within a particular
experiment. The stimuli as \emph{r} approaches 1 in the current study
exhibit greater levels of visual variance than the stimuli in
experiments 1 and 2 (see \chap{chap:adjusting_opacity}), which may
explain the lack of increased precision here. Further testing is
required for a more concrete explanation.

Ultimately, my aim is to provide tools for the design of visualisations
more suited for the tasks they are intended to support. When that task
is the perception of positive correlation, the use of the non-linear
size decay condition described here is recommended. For other
scatterplot tasks, such as cluster separation or numerosity perception,
or other chart types, the use of the size manipulation may in fact be a
hindrance.

\subsection{Training}\label{training-e3}

Before beginning the experimental trials, participants viewed plots
depicting \emph{r} = 0.2, 0.5, 0.8, and 0.95 for a minimum of 8 seconds.
Comparing a model including session half as a fixed effect with the
origin experimental model revealed no significant effect (\(\chi^2\)(1)
= 1.06, \emph{p} = 0.30), suggesting that having more recently viewed
the example plots did not have an effect on participants' performance.

\subsection{Limitations}\label{limitations-e3}

\subsection{Future Work}\label{future-work-e3}




\end{document}
