\documentclass[../main.tex]{subfiles}
\begin{document}


\section{Abstract}\label{abstract-adjusting-size}

\chap{chap:adjusting_opacity} provided strong evidence for the effects
of systematically varying the opacities of scatterplot points on
participants' estimates of correlation in positively correlated
scatterplots. Utilising the same function and experimental paradigm, I
show in a single experiment that systematically varying the sizes of
scatterplot points is able to bias participants' estimates of
correlation to a greater degree than manipulations that only adjust
point opacity. In a condition where point size decreases non-linearly as
a function of residual distance, correlation estimation is significantly
biased upwards to correct for a historic underestimation bias to a
greater degree. I discuss the implications of these findings for the
mechanisms behind both opacity and size adjustments in scatterplots in
relation to correlation estimation, and recommend techniques for those
who design with the estimation of positive correlation in mind.

\section{Introduction}\label{introduction-adjusting-size}

While I was successful at changing participants' perceptions of
correlation in positively correlated scatterplots in
\chap{chap:adjusting_opacity}, the extent to which these perceptions
were changed was minimal. Figure \ref{fig-estimates-by-r-e2} illustrates
how participants' mean errors in \emph{r} estimation changed as a
function of the subjective \emph{r} value. Scatterplots employing
non-linear opacity decay produced the most drastic changes in
correlation estimation, however this was still small, with an effect
size of Cohen's \emph{d} = 0.19. Recent evidence suggests that with
regards to altering percepts in scatterplots, changes in point size may
be more effective than changes in opacity. In a fully-reproducible,
large sample (N = 150) study, I show that systematically altering point
size using the same function is not only able to more effectively
correct for the correlation underestimation bias, but is also able to
alter the shape of the correlation estimation curve.

\section{Related Work}\label{related-work-adjusting-size}

\subsection{Size and Perception}\label{size-and-perception}

\subsection{Scatterplot Point Size and Correlation
Perception}\label{scatterplot-point-size-and-correlation-perception}

\section{Methods}\label{methods-e3}

\subsection{Stimuli}\label{stimuli-e3}

The creation of stimuli in this experiment follows the same general
principles outlined in Section \ref{creating-stimuli},
\chap{chap:gen_methods}. As in \chap{chap:adjusting_opacity}, equation 1
was used to map point residuals to size values in the two non-linear
decay conditions:

\begin{equation}
  point_{size/opacity} = 1 - b^{residual}
\end{equation}

As in \chap{chap:adjusting_opacity}, the use of this equation produces a
curve around the identity line symmetrically opposing the
underestimation curve found in previous work. Additionally, a constant
of 0.2 was added to each raw size value, and a scaling factor of 4 was
utilised; these adjustments resulted in the smallest points in the
present experiment having a width of 12 pixels on a 1920x1080 monitor,
which is consistent with the point size used in the experiments
described in \chap{chap:adjusting_opacity}. Examples of the stimuli used
in this experiment can be see in Figure~\ref{fig-exp3-examples-chap5}.

\begin{figure}

\centering{

\includegraphics[width=\textwidth]{5_adjusting_size_files/figure-latex/fig-exp3-examples-chap5-1.pdf}

}

\caption{\label{fig-exp3-examples-chap5}Examples of the stimuli used in
experiment 3, demonstrated with an \textit{r} value of 0.6.}

\end{figure}%

\subsection{Dot Pitch in Crowdsourced
Experiments}\label{dot-pitch-chap5}

When the experiments described in \chap{chap:adjusting_opacity} took
place, no method of obtaining dot pitch was implemented. Dot pitch is
defined as the distance between the dots (sub-pixels)
\cite{castellano_1992} that make up each pixel. Calculating dot pitch is
a requirement for the subsequent calculation of the physical on-screen
sizes of the scatterplot points that participants saw. In the preamble
to the current experiment, participants were asked to hold a standard
size credit/debit/ID card up to the monitor, and then to resize a
corresponding on-screen image until it matched the size of their
physical card \cite{screenscale}. These cards have a universal standard
size (ISO/IEC 7810 ID-1), which when combined with the monitor
resolution information recorded by Psychopy, and assuming a widescreen
16:9 aspect ratio, allows for the inference of dot pitch and therefore
the physical size of the points in the experience. Mean dot pitch was
0.33mm (\(SD = 0.06\)), corresponding to a physical size on the screen
of 4.32mm for the smallest points displayed. Section \ref{results-e3}
includes analysis that takes into account the physical on-screen sizes
of scatterplot points.

\subsection{Point Visibility Testing}\label{point-visibility-testing}

It is key that the manipulations used do not remove (or appear to
remove) data from scatterplots. Therefore, point visibility testing is
included in this experiment prior to the experimental items.
Participants were shown 6 scatterplots and were asked to enter in a text
box how many points were being displayed. These points were the same
size as the smallest points displayed in the experimental items. 5\% of
participants were correct on 5 out of 6 point visibility tests, while
95\% were correct on 6 out of 6. It should be noted that those
participants scoring 5/6 did not answer incorrectly, rather they did not
answer at all for a particular question, which is suggestive of a
mis-click or an initial misunderstanding of the task. Regardless, the
results of this test indicate a sufficient level of point visibility.

\subsection{Design}\label{design-e3}

Again, a fully repeated-measures, within-participants design was
employed. Each participant saw and responding to each of the 180
scatterplots in a fully randomised order. There were four scatterplots
for each of the 45 \emph{r} values corresponding to the four levels of
the size decay condition, examples of which are shown in
Figure~\ref{fig-exp3-examples-chap5}. The experiment itself is hosted on
Pavlovia \footnote{https://gitlab.pavlovia.org/Strain/exp\_size\_only}.

\subsection{Procedure}\label{procedure-e3}

\subsection{Participants}\label{participants-e3}

Normal to corrected-to-normal vision and English fluency were required.
Participants who had completed any of the experiments described in
\chap{chap:adjusting_opacity} were prevented from participating. Data
were collected from 164 participants. 14 failed more than 2 out of 6
attention check questions, and, as per the pre-registration, had their
submissions rejected from the study. The data from the remaining 150
participants were included in the full analysis (48\% male, 50 \%
female, and 2\% non-binary). Participants' mean age was 29.56 (\emph{SD}
= 8.54). Mean graph literacy score was 21.77 (\emph{SD} = 4.29). The
average time taken to complete the experiment was 39 minutes (SD = 14
minutes).

\section{Results}\label{results-e3}

\section{Discussion}\label{discussion-e3}

Increased Correlation Estimation Accuracy

Constant Correlation Estimation Precision

\subsection{Training}\label{training-e3}

\subsection{Limitations}\label{limitations-e3}

\subsection{Future Work}\label{future-work-e3}




\end{document}
