\documentclass[../main.tex]{subfiles}
\begin{document}


\section{Abstract}\label{abstract-adjusting-size}

\chap{chap:adjusting_opacity} provided strong evidence for the effects
of systematically varying the opacities of scatterplot points on
participants' estimates of correlation in positively correlated
scatterplots. Utilising the same function and experimental paradigm, I
show in a single experiment that systematically varying the sizes of
scatterplot points is able to bias participants' estimates of
correlation to a greater degree than manipulations that only adjust
point opacity. In a condition where point size decreases non-linearly as
a function of residual distance, correlation estimation is significantly
biased upwards to correct for a historic underestimation bias to a
greater degree. I discuss the implications of these findings for the
mechanisms behind both opacity and size adjustments in scatterplots in
relation to correlation estimation, and recommend techniques for those
who design with the estimation of positive correlation in mind.

\section{Introduction}\label{introduction-adjusting-size}

While I was successful at changing participants' perceptions of
correlation in positively correlated scatterplots in
\chap{chap:adjusting_opacity}, the extent to which these perceptions
were changed was minimal. \ref{fig-estimates-by-r-e2} illustrates how
participants' mean errors in \emph{r} estimation changed as a function
of the subjective \emph{r} value. Scatterplots employing non-linear
opacity decay produced the most drastic changes in correlation
estimation, however this was still small, with an effect size of Cohen's
\emph{d} = 0.19. Recent evidence suggests that with regards to altering
percepts in scatterplots, changes in point size may be more effective
than changes in opacity. In a fully-reproducible, large sample (N = 150)
study, I show that systematically altering point size using the same
function is not only able to more effectively correct for the
correlation underestimation bias, but is also able to alter the shape of
the correlation estimation curve.

\section{Related Work}\label{related-work-adjusting-size}

\subsection{Size and Perception}\label{size-and-perception}

\subsection{Scatterplot Point Size and Correlation
Perception}\label{scatterplot-point-size-and-correlation-perception}

\section{Methods}\label{methods-e3}

\subsection{Stimuli}\label{stimuli-e3}

The creation of stimuli in this experiment follows the same general
principles outlined in Section \ref{creating-stimuli},
\chap{chap:gen_methods}. As in \chap{chap:adjusting_opacity}, equation 1
was used to map point residuals to size values in the two non-linear
decay conditions

\begin{equation}
  point_{size/opacity} = 1 - b^{residual}
\end{equation}

\subsection{Dot Pitch in Crowdsourced
Experiments}\label{dot-pitch-in-crowdsourced-experiments}

\subsection{Point Visibility Testing}\label{point-visibility-testing}

\subsection{Design}\label{design-e3}

\subsection{Procedure}\label{procedure-e3}

\subsection{Participants}\label{participants-e3}

Normal to corrected-to-normal vision and English fluency were required.
Participants who had completed any of the experiments described in
\chap{chap:adjusting_opacity} were prevented from participating. Data
were collected from 164 participants. 14 failed more than 2 out of 6
attention check questions, and, as per the pre-registration, had their
submissions rejected from the study. The data from the remaining 150
participants were included in the full analysis (48.00\% male, 50.00 \%
female, and 2.00\% non-binary). Participants' mean age was 29.56
(\emph{SD} = 8.54). Mean graph literacy score was 21.77 (\emph{SD} =
4.29). The average time taken to complete the experiment was 39 minutes
(SD = 14 minutes).

\section{Results}\label{results-e3}

\section{Discussion}\label{discussion-e3}

Increased Correlation Estimation Accuracy

Constant Correlation Estimation Precision

\subsection{Training}\label{training-e3}

\subsection{Limitations}\label{limitations-e3}

\subsection{Future Work}\label{future-work-e3}




\end{document}
