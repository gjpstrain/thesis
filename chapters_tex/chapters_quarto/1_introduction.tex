\documentclass[../main.tex]{subfiles}
\begin{document}


Data visualisation is the practice of pictorially presenting patterns
otherwise described by numbers. Lists or matrices of numbers may be able
to communicate simple trends, but for more complex patterns, harnessing
the human visual system \cite{midway_2020} through visualisation is
crucial for the communication of science and data. Effective data
visualisation is able to reduce the cognitive and perceptual loads on
viewers, regardless of their backgrounds levels of statistical knowledge
or experience. This outsourcing, while efficient, leaves the viewer
vulnerable to the design choices that were made when creating the
visualisation. For this reason, understanding \emph{how} design choices
affect interpretation is crucial to designing better data visualisations
and simultaneously inoculating viewers against poor or malevolent design
practices.

It is hard to understate the importance and ubiquity of data
visualisations. They are used in communication between experts, to those
with no background in science, and to everyone in between. They are
relied upon to communicate vital public health information
\cite{li_2021}, to present evidence in court cases \cite{bobko_1979},
and to facilitate collaboration and encourage engagement on climate
change \cite{nocke_2008, schuster_2024}.

test render

\section{Research Motivation}\label{research-motivation}

\section{Contributions}\label{contributions-introduction}

\section{Included Publications}\label{included-publications}

The research described in chapters 4, 5, 6, and 7 in this thesis is
adapted from earlier publications, the last of which is under review as
of writing. To avoid repetition, information and discussion that would
be repeated has been consolidated into the literature review and general
methodology chapters. \emph{Gabriel Strain} is the primary author of all
included papers.

\begin{itemize}
\item
  \emph{The Effects of Contrast on Correlation Perception in
  Scatterplots} \cite{strain_2023} is reproduced in
  \chap{chap:adjusting_opacity}. Sections
  \ref{methods-adjusting-opacity-e1},
  \ref{methods-adjusting-opacity-e2},
  \ref{analysis-adjusting-opacity-e1},
  \ref{analysis-adjusting-opacity-e2},
  \ref{discussion-adjusting-opacity-e1},
  \ref{discussion-adjusting-opacity-e2}, and
  \ref{general-discussion-adjusting-opacity} contain minimally altered
  parts of the published article.
\item
  \emph{Adjusting Point Size to Facilitate More Accurate Correlation
  Perception in Scatterplots} \cite{strain_2023b} is reproduced in
  \chap{chap:adjusting_size}. Sections \ref{methods-adjusting-size},
  \ref{analysis-adjusting-size}, \ref{discussion-adjusting-size}, and
  \ref{general-discussion-adjusting-size} contain minimally altered
  parts of the published article.
\item
  \emph{Effects of Point Size and Opacity Adjustments in Scatterplots}
  \cite{strain_2024} is reproduced in
  \chap{chap:interactions_opacity_size}. Sections
  \ref{methods-interactions}, \ref{analysis-interactions},
  \ref{discussion-interactions}, and
  \ref{general-discussion-interactions} contain minimally altered parts
  of the published article.
\item
  \emph{Effects of Alternative Scatterplot Designs on Belief}
  (\emph{under review}) is reproduced in \chap{chap:belief_change}.
  Sections \ref{beliefs-pre-study}, \ref{methods-beliefs-main},
  \ref{analysis-beliefs-main}, \ref{discussion-beliefs-main}, and
  \ref{general-discussion-beliefs} contain minimally altered parts of
  the published article.
\end{itemize}

\section{Overview of Thesis}\label{overview-of-thesis}




\end{document}
