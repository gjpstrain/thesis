\documentclass[../main.tex]{subfiles}
\begin{document}


Data visualisation is the practice of pictorially presenting patterns
otherwise described by numbers, and has been employed in one form or
another for thousands of years \cite{azzam_2013}. Lists or matrices of
numbers may be able to communicate simple trends, but for more complex
patterns, harnessing the human visual system \cite{midway_2020} through
visualisation is crucial for the communication of science and data.
Effective data visualisation is able to reduce cognitive and perceptual
loads on viewers, regardless of their backgrounds levels of statistical
knowledge or experience. This outsourcing, while efficient, leaves the
viewer vulnerable to the design choices that were made when creating the
visualisation. For this reason, understanding \emph{how} design choices
affect interpretation is crucial to designing better data visualisations
and simultaneously allows for the inoculation of viewers against poor or
malevolent design practices.

One cannot understate the importance and ubiquity of data
visualisations. They are used in to communicate with experts, with those
with no background in science, and with everyone in between. They are
relied upon to communicate vital public health information
\cite{li_2021}, to present evidence in court cases \cite{bobko_1979},
and to facilitate collaboration and encourage engagement on climate
change issues \cite{nocke_2008, schuster_2024}. Studying data
visualisation therefore has the potential to change the nature of
scientific study itself.

Interaction with a data visualisation involves steps from perception
(viewing), to cognition (interpreting), to behaviour (deciding and
acting). These stages must be examined both separately and together, as
there are bottom-up and top-down interactions present. In this thesis, I
therefore present a series of experiments whose aim was to investigate
and address a long-standing perceptual bias in scatterplots, a very
common form of data visualisation \cite{friendly_2005}. Following
attempts to address this bias, I further investigated the impacts these
attempts may have had on cognition and behaviour.

\section{Research Motivation}\label{research-motivation}

Data visualisations were once the preserve of the academic and
professional classes. Now, however, visualisation is everywhere. This
became especially apparent to me during the COVID-19 pandemic, when
people such as my parents, professionals with no background in
mathematics or statistics, were bombarded with data visualisations on a
daily basis. Despite not being able to fully articulate the whole data
story, they were still able to derive meaning from the visualisations
they were seeing. The ability of data visualisations to transcend
language and even mathematical ability motivated my study of them;
concepts such as exponential growth or the relatedness between variables
can be displayed in simple ways that do not rely on underlying
understanding of exponents or correlation coefficients.

Using data visualisations in such a way effectively provides a cognitive
proxy for viewers. This is an efficient way to communicate, however
still necessitates the presence of accurate and reliable perception. On
further investigation into the reliance that those who design
visualisations make on those who view them, I uncovered a historic
perceptual bias that was still being described in the literature,
seemingly with no attempt at correction. This bias was the
underestimation of correlation in positively correlated scatterplots. In
the literature, this robust effect had been observed in a number of
experimental paradigms, including direct estimation
\cite{strahan_1978, bobko_1979, cleveland_1982,
collyer_1990, lane_1985, lauer_1989, meyer_1997} and estimation via
bisection tasks \cite{rensink_2017}, and as of 2021, no attempts had
been made to correct for it. The goal of this thesis was therefore to
collect empirical evidence on the perception of correlation in
positively correlated scatterplots, use that information to create novel
visualisations with a view to correcting for the underestimation, and
further investigate the potential for these visualisations to effect the
things that people think and believe.

\section{Contributions}\label{contributions-introduction}

Through this thesis, I present a series of empirical experiments that
provide knowledge about how changing visual features in scatterplots can
affect how participants interpret the strength of the correlation
displayed. I demonstrate that systematically reducing the opacities and
sizes of points on scatterplots as a function of their distances from a
regression line can significantly increase estimates of correlation and
partially correct for a historic underestimation bias. Following this, I
show that consequences of employing these techniques are not limited to
perception, but in fact can be extended into a cognitive space to change
what people think and believe. I utilise large-sample, controlled
experiments with lay populations to provide generalisable conclusions
and recommendations for visualisation designers and researchers. Through
this thesis, I provide a framework for the large-scale testing of data
visualisations with lay audiences. Additionally, I hope to provide an
example of a project conducted entirely in an open and reproducible
manner.

\section{Included Publications}\label{included-publications}

The research described in chapters 4, 5, 6, and 7 in this thesis is
adapted from earlier publications, the last of which is under review as
of writing. To avoid repetition, information and discussion that would
be repeated has been consolidated into the literature review and general
methodology chapters. \emph{Gabriel Strain} is the primary author of all
included papers.

\begin{itemize}
\item
  \emph{The Effects of Contrast on Correlation Perception in
  Scatterplots} \cite{strain_2023} is reproduced in
  \chap{chap:adjusting_opacity}. Sections
  \ref{methods-adjusting-opacity-e1},
  \ref{methods-adjusting-opacity-e2},
  \ref{analysis-adjusting-opacity-e1},
  \ref{analysis-adjusting-opacity-e2},
  \ref{discussion-adjusting-opacity-e1},
  \ref{discussion-adjusting-opacity-e2}, and
  \ref{general-discussion-adjusting-opacity} contain minimally altered
  parts of the published article.
\item
  \emph{Adjusting Point Size to Facilitate More Accurate Correlation
  Perception in Scatterplots} \cite{strain_2023b} is reproduced in
  \chap{chap:adjusting_size}. Sections \ref{methods-adjusting-size},
  \ref{analysis-adjusting-size}, \ref{discussion-adjusting-size}, and
  \ref{general-discussion-adjusting-size} contain minimally altered
  parts of the published article.
\item
  \emph{Effects of Point Size and Opacity Adjustments in Scatterplots}
  \cite{strain_2024} is reproduced in
  \chap{chap:interactions_opacity_size}. Sections
  \ref{methods-interactions}, \ref{analysis-interactions},
  \ref{discussion-interactions}, and
  \ref{general-discussion-interactions} contain minimally altered parts
  of the published article.
\item
  \emph{Effects of Alternative Scatterplot Designs on Belief}
  (\emph{under review}) is reproduced in \chap{chap:belief_change}.
  Sections \ref{beliefs-pre-study}, \ref{methods-beliefs-main},
  \ref{analysis-beliefs-main}, \ref{discussion-beliefs-main}, and
  \ref{general-discussion-beliefs} contain minimally altered parts of
  the published article.
\end{itemize}

\section{Overview of Thesis}\label{overview-of-thesis}

In Chapter 2, I conduct a thorough review of the literature and provide




\end{document}
