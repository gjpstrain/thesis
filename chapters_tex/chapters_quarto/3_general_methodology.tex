\documentclass[../main.tex]{subfiles}
\begin{document}


\section{Introduction}\label{introduction}

As researchers, the selection of appropriate methods is crucial in order
that our interpretations may be based in truth. The selection of methods
necessarily entails research of its own, with approaches each having
distinct advantages and disadvantages, and thus the decisions that were
made are discussed below. It is worth noting that different methods may
produce different results, meaning this thesis cannot claim to be an
entire representation of knowledge, but rather a reflection of the
researcher, the ways the research was done, and the particular
philisophical environment that fostered the work.

\section{Experimental Methods}\label{experimental-methods}

In order to understand how changing visual and design features can
affect interpretations of scatterplots, the work presented here tests
hypotheses using controlled experiments. This allows for the isolation
of factors of interest from other features, and allows us to establish
causality with regards to effects. The use of empirical studies in
visualisation has a rich history \cite{lam_2012}, and facilitates the
move towards designing from the ground up based on theory, as opposed to
a more traditional paradigm in which visualisations are only tested
after they have been designed by technical experts.

\subsection{Experimental Design}\label{experimental-design}

\chap{chap:introduction}

\subsection{Tools for Testing}\label{tools-for-testing}

\subsection{Creating Stimuli}\label{creating-stimuli}

\subsection{Recruitment}\label{recruitment}

\section{Analytical Methods}\label{analytical-methods}

\subsection{Linear Mixed-Effects
Models}\label{linear-mixed-effects-models}

\subsection{Advantages Over Aggregate-Level Statistical
Tests}\label{advantages-over-aggregate-level-statistical-tests}

\subsection{Model Construction}\label{model-construction}

\subsection{Effects Sizes}\label{effects-sizes}

\subsection{Reporting Analyses}\label{reporting-analyses}

\section{Computational Methods}\label{computational-methods}

\subsection{Executable Reporting}\label{executable-reporting}

\subsection{Containerised
Environments}\label{containerised-environments}

\section{Reproducility In This
Thesis}\label{reproducility-in-this-thesis}

\subsection{Sharing Data and Code}\label{sharing-data-and-code}

\subsection{Executable Papers and Docker
Containers}\label{executable-papers-and-docker-containers}

\subsection{Experimental Resources}\label{experimental-resources}

\section{Conclusion}\label{conclusion}




\end{document}
