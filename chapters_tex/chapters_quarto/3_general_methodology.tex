\documentclass[../main.tex]{subfiles}
\begin{document}


\section{Introduction}\label{introduction}

In this chapter we describe our research methodologies. Chapters 4, 5,
and 6 share most aspects of experimental method, while the experiment
described in chapter 7 differs substantially. Throughout this chapter,
the reader should assume that we are referring to the entire body of
experimental work this thesis describes. Methods that differ regarding
the final experiment in chapter 7 are detailed along the way. In this
chapter, we discuss our experimental designs, the tools we use to build
and run our experiments, our approach to statistical analyses, and the
computational methods and practices we employed particularly with
regards to reproducibility and open science.

\section{Experimental Methods}\label{experimental-methods}

It is important to acknowledge that the way in which we conduct
experiments influences what we find and the conclusions that we may draw
from those findings. The decisions that lead us to designing experiments
in certain ways must be based not only on theory, but also on the
practical constraints imposed by external factors on the research team.
Concerns such as time, convenience, and cost must be addressed, and a
compromise between research that is \emph{valuable} and research that is
\emph{doable} must be reached. We focused on pragmatism and impact
throughout the course of this research project; happily, the research
journey we embarked on resulted in methodologies that satisfied both
principles. It is for this reason that we consider the framework we
present to be a key contribution of this thesis.

\subsection{Experimental Design}\label{experimental-design}

All but our final experiment utilised within-participants designs. Each
participant saw all experimental stimuli and provided a judgement of
correlation using a sliding scale between 0 and 1 (see
Figure~\ref{fig-slider}). Experiments 1 to 3 featured featured a single
experimental factor of design, all with 4 levels corresponding to
scatterplots with different design features. Experiment 4 employed a
factorial 2 \(\times\) 2 design. Experiment 5 is a departure from the
shared experimental paradigm of the previous experiments, and features a
1 factor, 2 level between-participants design.

\begin{figure}

\centering{

\includegraphics{../supplied_graphics/example_slider.png}

}

\caption{\label{fig-slider}An example of the slider participants used to
estimate correlation in experiments 1-4.}

\end{figure}%

\subsection{Tools for Testing}\label{tools-for-testing}

Whatever the design of our experiments, software plays a crucial role in
allowing us to carry them out. Fortunately, at the time of writing,
there is a wealth of tools available to facilitate the testing of
visualisations both in traditional lab-based tests and in online
experiments. As we adhere to the principles of open and reproducible
research \cite{ayris_2018}, we discount closed-source software, such as
Gorilla \cite{anwyl_2020} or E-prime \cite{eprime_2020}, as these rely
on paid licenses and do not allow us to share code with future
researchers. We settled on using PsychoPy \cite{peirce_2019} due to its
open-source status, flexibility regarding graphical and code-based
experimental design, and high level of timings accuracy
\cite{bridges_2020}. Using such a open-source tool not only facilitated
our own learning with regard to experiment building, but also enables to
contribute further examples of visualisation studies by hosting the
resulting experiments online for use and modification by future
researchers.

We elected to pursue online testing throughout this thesis. Doing so is
much quicker than carrying out in-person lab-based testing, meaning we
can collect data from a much larger number of participants. This reduces
the chances of detecting false positives during analysis and ensures
adequate levels of power despite the potential for small effects sizes.
Online testing also affords us access to diverse groups of participants
across our populations of interest, especially when compared to the
relatively homogeneous student populations usually accessed by doctoral
researchers.

\subsection{Creating Stimuli}\label{creating-stimuli}

\subsection{Recruitment}\label{recruitment}

\section{Analytical Methods}\label{analytical-methods}

\subsection{Linear Mixed-Effects
Models}\label{linear-mixed-effects-models}

\subsection{Advantages Over Aggregate-Level Statistical
Tests}\label{advantages-over-aggregate-level-statistical-tests}

\subsection{Model Construction}\label{model-construction}

\subsection{Effects Sizes}\label{effects-sizes}

\subsection{Reporting Analyses}\label{reporting-analyses}

\section{Computational Methods}\label{computational-methods}

\subsection{Executable Reporting}\label{executable-reporting}

\subsection{Containerised
Environments}\label{containerised-environments}

\section{Reproducibility In This
Thesis}\label{reproducibility-in-this-thesis}

\subsection{Sharing Data and Code}\label{sharing-data-and-code}

\subsection{Executable Papers and Docker
Containers}\label{executable-papers-and-docker-containers}

\subsection{Experimental Resources}\label{experimental-resources}

\section{Conclusion}\label{conclusion}




\end{document}
