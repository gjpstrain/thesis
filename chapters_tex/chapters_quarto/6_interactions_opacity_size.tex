\documentclass[../main.tex]{subfiles}
\begin{document}


\section{Abstract}\label{abstract-interactions}

\section{Introduction}\label{introduction-interactions}

\section{Related Work}\label{related-work-interactions}

\section{Hypotheses}\label{hypotheses-interactions}

A single experiment is present in this chapter based on the effects of
adjusting point opacity and size on correlation estimation established
in \chap{chap:adjusting_opacity} and \chap{chap:adjusting_size}. Here,
previously independently tested point opacity and size manipulations are
tested in both typical orientation (point opacity/size is reduced with
residual magnitude) and inverted orientation (point opacity/size is
increased with residual magnitude). Throughout this chapter, referral is
made to \emph{congruent} and \emph{incongruent} conditions with respect
to the combination of point opacity and size decay functions.
\emph{Congruent} conditions are those in which point opacity and size
decay act in the same direction (typical or inverted), while for
\emph{incongruent} conditions, point opacity and size decay act against
each other. Due to previous findings that non-linearly reducing point
opacity or size as a function of residual magnitude can bias correlation
estimates upwards, it is hypothesised that:

\begin{itemize}
\tightlist
\item
  H1: an increased reduction in correlation estimation error will be
  observed when congruent typical orientation decay functions are used.
  H2: the use of a congruent inverted function, such that point opacity
  and size are both increased with residual magnitude, will produce the
  least accurate estimates of correlation.
\end{itemize}

Owing to the finding from \chap{chap:@adjusting_size} that point size is
a stronger channel for biasing correlation estimation than point
opacity, it is also hypothesised that:

\begin{itemize}
\tightlist
\item
  H3: there will be a significant difference in correlation estimates
  between the two incongruent orientation conditions.
\end{itemize}

\section{Methods}\label{methods-e4}

\subsection{Stimuli}\label{stimuli-e4}

\subsection{Design}\label{design-e4}

\subsection{Procedure}\label{procedure-e4}

\subsection{Participants}\label{participants-e4}

Normal to corrected-to-normal vision and English fluency were required.
Participants who had completed any of the experiments described in
\chap{chap:adjusting_opacity} or \chap{chap:adjusting_size} were
prevented from participating. Data were collected from 158 participants.
8 failed more than 2 out of 6 attention check questions, and, as per the
pre-registration, had their submissions rejected from the study. The
data from the remaining 150 participants were included in the full
analysis (51\% male, 49 \% female, and 1\% non-binary). Participants'
mean age was 30.65 (\emph{SD} = 8.64). Mean graph literacy score was
22.49 (\emph{SD} = 3.55). The average time taken to complete the
experiment was 37 minutes (SD = 12.3 minutes).

\section{Results}\label{results-e4}

To investigate the effects of combining point opacity and size decay
functions on participants' estimates of correlation, a linear mixed
effects model was built whereby the particular combination of point
opacity and size decay function employed is a predictor for the
difference between objective \emph{r} values for each plot and
participants' estimates of \emph{r}. Deviation coding was used for each
of the experimental factors, which allows comparison between means of
\emph{r} estimation error and the grand mean. This model has random
intercepts for items and participants, and random slopes for
participants with regards to the size decay factor. A likelihood ratio
test revealed that the model including the opacity and size decay
conditions as predictors explained significantly more variance than the
null (\(\chi^2\)(3) = 5,286.81, \emph{p} \textless{} .001). There were
significant fixed effects of opacity and size decay function, as well as
a significant interaction between the two. Figure~\ref{fig-e4-estimates}
shows the mean errors in correlation estimation for each combination of
conditions, along with 95\% confidence intervals.

\begin{figure}

\centering{

\includegraphics[width=\textwidth]{6_interactions_opacity_size_files/figure-latex/fig-e4-estimates-1.pdf}

}

\caption{\label{fig-e4-estimates}Estimated marginal means for the four
conditions tested in experiment 4. 95\% confidence intervals are shown.
The vertical dashed line represents no estimation error.}

\end{figure}%

\begin{table}

\caption{\label{tbl-contrasts-e4}Contrasts between different levels of
the opacity and size decay factors in experiment 4.}

\centering{

\begin{tabular}[t]{llrl}
\toprule
\multicolumn{2}{c}{Contrast} & \multicolumn{2}{c}{Statistics} \\
\cmidrule(l{3pt}r{3pt}){1-2} \cmidrule(l{3pt}r{3pt}){3-4}
  &    & Z ratio & \textit{p}\\
\midrule
TO Size x IO Opacity & IO Size x IO Opacity & -10.945 & <0.001\\
TO Size x IO Opacity & TO Size x TO Opacity & 72.294 & <0.001\\
TO Size x IO Opacity & IO Size x TO Opacity & -2.256 & 0.108\\
IO Size x IO Opacity & TO Size x TO Opacity & 46.125 & <0.001\\
IO Size x IO Opacity & IO Size x TO Opacity & 17.838 & <0.001\\
\addlinespace
TO Size x TO Opacity & IO Size x TO Opacity & -37.438 & <0.001\\
\bottomrule
\end{tabular}

}

\end{table}%

The effects found were driven by significant difference between means of
correlation estimation error between all conditions besides that which
compares the two incongruent decay conditions. Statistical testing for
contrasts were performed using the \texttt{emmeans} package
\cite{lenth_2024}, and are provided in Table~\ref{tbl-contrasts-e4}.
Experiments 1, 2, and 3 all featured a comparative baseline condition.
In the former two experiments, this was the full contrast condition (see
\chap{chap:adjusting_opacity}), while in the latter, this was the
standard size condition (see \chap{chap:adjusting_size}). In the current
experiment, no baseline condition was used. Owing both to this and the
use of a linear mixed effects model with an interaction term, the use of
Cohen's \emph{d} as a measure of effect size would be inappropriate. In
its place, the amounts of variance in participants' errors in
correlation explanation explained by each fixed effect term and the
interaction term is represented as semi-partial R\textsuperscript{2}
\cite{nakagawa_2013}. These statistics were calculated using the
\texttt{r2glmm} package (version 0.1.2) \cite{r2glmm}, and are presented
along with model statistics in Table~\ref{tbl-model-stats-e4}.

\begin{table}

\caption{\label{tbl-model-stats-e4}Significances of fixed effects and
the interaction between them. Semi-partial R\textsuperscript{2} for each
fixed effect and the interaction term is also displayed in lieu of
effect sizes.}

\centering{

\begin{tabular}[t]{lrrrrll}
\toprule
  & Estimate & Standard Error & df & t-value & \textit{p} & R\textsuperscript{2}\\
\midrule
(Intercept) & 0.08 & 0.013 & 103.32 & 6.27 & <0.001 & \\
Size Decay & -0.14 & 0.005 & 148.39 & -25.77 & <0.001 & 0.104\\
Opacity Decay & 0.12 & 0.002 & 26327.21 & 63.71 & <0.001 & 0.087\\
Size Decay x Opacity Decay & 0.15 & 0.004 & 26327.13 & 38.47 & <0.001 & 0.034\\
\bottomrule
\end{tabular}

}

\end{table}%

Models including participants' graph literacy, their performance on the
point visibility task, the dot pitch of participants' monitors, and
which half a particular correlation judgement took place were built and
compared with the experimental model. While no significant effects of
graph literacy (\(\chi^2\)(1) = 3.50, \emph{p} = .061), performance on
the point visibility task (\(\chi^2\)(1) = 1.29, \emph{p} = .257), or
dot pitch (\(\chi^2\)(1) = 1.52, \emph{p} = .218) were found, there was
a significant effect of training (\(\chi^2\)(1) = 23.78, \emph{p}
\textless{} .001), with participants rating correlation .01 lower during
the second half. This drop suggests that having more recently viewed the
training plots may have increased participants estimates of correlation.
To further analyse this variability, a model was built including trial
number, allowing for the analysis of error. A significant effect of
trial number is also found (\(\chi^2\)(1) = 29.31, \emph{p} \textless{}
.001) on participants' correlation estimation errors.

\begin{figure}

\centering{

\includegraphics[width=\textwidth]{6_interactions_opacity_size_files/figure-latex/fig-e4-trial-number-1.pdf}

}

\caption{\label{fig-e4-trial-number}Comparing mean errors in correlation
estimation by trial number. Points represent unsigned mean errors for
each trial number. The plotted line is the locally estimated smoothed
curve, with the ribbon representing standard errors.}

\end{figure}%

Figure~\ref{fig-e4-trial-number} shows participants' unsigned mean
errors in correlation estimation against trial number. Variability in
error, as represented by the ribbon, stabilised quickly and remained
stable for most the experiment, only widening again around trial number
170. The simplest explanation for this is that participants, knowing
they were coming to the end of the experiment, became less vigilant and
rushed their judgements more. Regardless of statistical significance,
this effect is not large enough to warrant further investigation, at
least as it pertains to correlation estimation in scatterplots.

\begin{figure}

\centering{

\includegraphics[width=\textwidth]{6_interactions_opacity_size_files/figure-latex/fig-estimates-by-r-e4-1.pdf}

}

\caption{\label{fig-estimates-by-r-e4}Participants' mean errors in
correlation estimates grouped by factor and by \textit{r} value. The
dashed horizontal line represents perfect estimation. Participants were
most accurate when presented with the plots featuring the non-linear
size decay function. Error bars show standard deviations of estimates.}

\end{figure}%

\section{Discussion}\label{discussion-e4}

Hypothesis 1 received full support in this experiment. The combination
of typical orientation opacity and size decay functions produced the
most accurate estimates of correlation, although this also resulted in a
marked over-correction and consequent overestimation for many values of
\emph{r} (see Figure~\ref{fig-estimates-by-r-e4}). The second hypothesis
also received support; the combination of inverted opacity and size
decay functions produced the least accurate estimates of correlation. No
support was found for the third hypothesis, that there would be a
significant difference in correlation estimates between inverted
orientation opacity/typical orientation size plots and typical
orientation opacity/inverted orientation size plots. There was, however,
a significant interaction term present, providing evidence that the
combination of opacity and size decay functions is not additive in
natures.

\subsection{Combining Manipulations}\label{combining-manipulations}

\subsection{Estimation Precision}\label{estimation-precision}

\subsection{Relative Contributions of Opacity and Srize
Decay}\label{relative-contributions-of-opacity-and-srize-decay}

\subsection{Mechanisms}\label{mechanisms}

\subsection{Limitations}\label{limitations}

\subsection{Future Work}\label{future-work}

\section{Conclusion}\label{conclusion-interactions}




\end{document}
