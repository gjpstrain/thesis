\documentclass[../main.tex]{subfiles}
\begin{document}


\section{Abstract}\label{abstract-interactions}

\section{Introduction}\label{introduction-interactions}

\section{Related Work}\label{related-work-interactions}

\section{Hypotheses}\label{hypotheses-interactions}

\section{Methods}\label{methods-e4}

\subsection{Stimuli}\label{stimuli-e4}

\subsection{Design}\label{design-e4}

\subsection{Procedure}\label{procedure-e4}

\subsection{Participants}\label{participants-e4}

Normal to corrected-to-normal vision and English fluency were required.
Participants who had completed any of the experiments described in
\chap{chap:adjusting_opacity} or \chap{chap:adjusting_size} were
prevented from participating. Data were collected from 158 participants.
8 failed more than 2 out of 6 attention check questions, and, as per the
pre-registration, had their submissions rejected from the study. The
data from the remaining 150 participants were included in the full
analysis (51\% male, 49 \% female, and 1\% non-binary). Participants'
mean age was 30.65 (\emph{SD} = 8.64). Mean graph literacy score was
22.49 (\emph{SD} = 3.55). The average time taken to complete the
experiment was 37 minutes (SD = 12.3 minutes).

\section{Results}\label{results-e4}

To investigate the effects of combining point size and opacity decay
functions on participants' estimates of correlation, a linear mixed
effects model was built whereby the particular combination of point size
and opacity decay function employed is a predictor for the difference
between objective \emph{r} values for each plot and participants'
estimates of \emph{r}. Deviation coding was used for each of the
experimental factors, which allows comparison between means of \emph{r}
estimation error and the grand mean. This model has random intercepts
for items and participants, and random slopes for participants with
regards to the size decay factor. A likelihood ratio test revealed that
the model including the size and opacity decay conditions as predictors
explained significantly more variance than the null (\(\chi^2\)(3) =
5,286.81, \emph{p} \textless{} .001). There were significant fixed
effects of size and opacity decay function, as well as a significant
interaction between the two. Figure~\ref{fig-e4-estimates} shows the
mean errors in correlation estimation for combination of conditions,
along with 95\% confidence intervals.

\begin{figure}

\centering{

\includegraphics[width=\textwidth]{6_interactions_opacity_size_files/figure-latex/fig-e4-estimates-1.pdf}

}

\caption{\label{fig-e4-estimates}Estimated marginal means for the four
conditions tested in experiment 4. 95\% confidence intervals are shown.
The vertical dashed line represents no estimation error.}

\end{figure}%

\begin{table}

\caption{\label{tbl-contrasts-e4}Contrasts between different levels of
the size and opacity decay factors in experiment 4.}

\centering{

\begin{tabular}[t]{llrl}
\toprule
\multicolumn{2}{c}{Contrast} & \multicolumn{2}{c}{Statistics} \\
\cmidrule(l{3pt}r{3pt}){1-2} \cmidrule(l{3pt}r{3pt}){3-4}
  &    & Z ratio & \textit{p}\\
\midrule
TO Size x IO Opacity & IO Size x IO Opacity & -10.945 & <0.001\\
TO Size x IO Opacity & TO Size x TO Opacity & 72.294 & <0.001\\
TO Size x IO Opacity & IO Size x TO Opacity & -2.256 & 0.108\\
IO Size x IO Opacity & TO Size x TO Opacity & 46.125 & <0.001\\
IO Size x IO Opacity & IO Size x TO Opacity & 17.838 & <0.001\\
\addlinespace
TO Size x TO Opacity & IO Size x TO Opacity & -37.438 & <0.001\\
\bottomrule
\end{tabular}

}

\end{table}%

The effects found were driven by significant difference between means of
correlation estimation error between all conditions besides that which
compares the two incongruent decay conditions. Statistical testing for
contrasts were performed using the \texttt{emmeans} package
\cite{lenth_2024}, and are provided in Table~\ref{tbl-contrasts-e4}.

Models including participants' graph literacy, their performance on the
point visibility task, the dot pitch of participants' monitors, and
which half a particular correlation judgement took place were built and
compared with the experimental model. While no significant effects of
graph literacy (\(\chi^2\)(1) = 3.50, \emph{p} = .061), performance on
the point visibility task (\(\chi^2\)(1) = 1.29, \emph{p} = .257), or
dot pitch (\(\chi^2\)(1) = 1.52, \emph{p} = .218) were found, there was
a significant effect of training (\(\chi^2\)(1) = 23.78, \emph{p}
\textless{} .001), with participants rating correlation .01 lower during
the second half. This drop suggests that having more recently viewed the
training plots may have increased participants estimates of correlation.
To further analyse this variability, a model was built including trial
number, allowing for the analysis of error. A significant effect of
trial number is also found (\(\chi^2\)(1) = 29.31, \emph{p} \textless{}
.001) on participants' correlation estimation errors.

\begin{figure}

\centering{

\includegraphics[width=\textwidth]{6_interactions_opacity_size_files/figure-latex/fig-e4-trial-number-1.pdf}

}

\caption{\label{fig-e4-trial-number}Comparing mean errors in correlation
estimation by trial number. Points represent unsigned mean errors for
each trial number. The plotted line is the locally estimated smoothed
curve, with the ribbon representing standard errors.}

\end{figure}%

Figure~\ref{fig-e4-trial-number} is great.

\section{Discussion}\label{discussion-e4}

\section{Conclusion}\label{conclusion-interactions}




\end{document}
