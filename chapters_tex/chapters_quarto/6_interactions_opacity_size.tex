\documentclass[../main.tex]{subfiles}
\begin{document}


\section{Abstract}\label{abstract-interactions}

\chap{chap:adjusting_opacity} and \chap{chap:adjusting_size} each
provide evidence for the effects of changing the opacities and sizes of
scatterplot points on people's performance on a correlation estimation
task. It is clear from the results of the experiments presented in those
chapters, however, that the effects of changing point opacity and size
on correlation estimation in positively correlated scatterplots are
different, both with regards to the strength of the correction provided
and the shape of the estimation curve produced. Mechanistically,
however, point opacity and size may operate similarly by reducing the
influence of more exterior points in scatterplots. To further
investigate these mechanisms, and in the interest of providing a more
complete description of the ways in which changing point opacity and
size can bias correlation estimation, I present a single experiment
study combining the point opacity and size manipulations from
\chap{chap:adjusting_opacity} and \chap{chap:adjusting_size}. In a
condition where point opacity and size are reduced as a function of
residual magnitude, correlation estimation is significantly biased
upwards, a finding that suggests there are few limits with regards to
using point opacity and size to manipulate perceived contrast.
Additionally, this effect is stronger than the effects of changing point
opacity and size in isolation, and is greater than what might be
expected if the effects of point opacity and size on correlation
estimation were simply additive.

\section{Introduction}\label{introduction-interactions}

The work presented in \chap{chap:adjusting_opacity} and
\chap{chap:adjusting_size} shows that point opacity and size adjustments
can be used to bias perceptions of correlation in positively correlated
scatterplots. These findings are in opposition to work finding both bias
and variability in correlation perception to be invariant to both
uniform and irregular changes in point opacities and sizes
\cite{rensink_2012, rensink_2014}. Of course, when attempting to provide
tools for visualisation designers to design \emph{better}
visualisations, more options are preferable to few. In this spirit,
then, I endeavoured to answer the next big question; what happens when
point opacity and size manipulations are combined? Answering this
question would not only allow for a deeper exploration of the potential
mechanisms behind each decay function, but would further empower
visualisation designers with the knowledge of how altering point
opacities and sizes might affect people's interpretations of the
correlations displayed therein. In a fully-reproducible, large-sample
(\emph{N} = 150) study, I show that combining point opacity and size
manipulations can produce more powerful effects on correlation
estimation than either manipulation in isolation. Additionally, I
comment on the impact that the adjustment of each visual feature may
have on correlation estimation, and suggest future work to tune the
effects seen in \chap{chap:adjusting_opacity},
\chap{chap:adjusting_size}, and here.

\section{Related Work}\label{related-work-interactions}

Much of the related work presented here is covered in more detail in
\chap{chap:related_work} and in Sections
\ref{related-work-adjusting-opacity} and
\ref{related-work-adjusting-size} in \chap{chap:adjusting_opacity} and
\chap{chap:adjusting_size} respectively. For the reader's convenience,
and to properly contextualise the experimental methods and findings in
the present chapter, this related work is reproduced in brief here.

\subsection{Opacity and Contrast}\label{opacity-and-contrast}

Changing the opacities of points in scatterplots is standard practice
when visualising very large datasets to address overplotting
\cite{matejka_2015}. Figure \ref{fig-overplotting-examples} visualises
how point opacity is often reduced when large numbers of points are
present to preserve individual point discriminability. While work on the
effects of point opacity on correlation perception in scatterplots has
taken place, it is often contradictory. Earlier work found correlation
perception invariant to changes in the opacities of points
\cite{rensink_2012, rensink_2014}, while the work I completed in
\chap{chap:adjusting_opacity} found clear, if small, effects. That work
offered both point salience/perceptual weighting and spatial uncertainty
as potential drivers for the effects found. Lower stimulus contrast,
which for isolated stimuli is functionally identical to lower opacity,
is associated with lower salience \cite{healey_2012}, can bias
judgements of mean point position \cite{hong_2022}, increases error in
positional judgements \cite{wehrhahn_1990}, and can result in greater
uncertainty in speed perception \cite{champion_2017}. Due to mechanistic
accounts of both salience/perceptual weighting and spatial uncertainty
predicting results in the same direction regarding opacity adjustments,
the work I conducted in \chap{chap:adjusting_opacity} was unable to
distinguish between explanations rooted in point salience/perceptual
weighting and spatial uncertainty as drivers for the effect found.

Micallef et al.~\cite{micallef_2017} found that ``merging, dark dots''
support correlation estimation; despite only changing point opacity in a
\emph{uniform} manner, the sheer number of points used in that study
results in scatterplots that appear similar to those that reduce opacity
as a function of residual magnitude (see \chap{chap:adjusting_opacity}).
That this technique has been shown to produce more accurate correlation
estimates as compared to unadjusted scatterplots may explain why the
optimisation system employed in Micallef et al. \cite{micallef_2017}
conferred benefits regarding correlation estimation.

\subsection{Point Size}\label{point-size}

Again, for discriminability reasons, scatterplots dealing with larger
datasets tend to have smaller individual points. Section
\ref{point-size-chap5} explores this, and related bubble charts, in
greater detail. As with opacity, the work exploring the effects of point
size on correlation perception in scatterplots is conflicted. Some finds
correlation perception to be invariant to both global and irregular
changes in point opacity \cite{rensink_2012, rensink_2014}, while the
work presented in \chap{chap:adjusting_size} contradicts this, finding
effects on correlation perception that go beyond what is available
through the manipulation of point opacity alone.

While the mechanistic driver of the effects of point opacity on
correlation perception is unclear, the balance of evidence for point
size suggests a point salience/perceptual weighting account. There is
evidence that larger stimulus size is associated with lower levels of
spatial certainty \cite{alais_2004}, but higher levels of salience
\cite{healey_2012}, results which are supported by evidence that
reaction times are slower to smaller stimuli
\cite{gramazio_2014, osaka_1976}. The predicted effects of spatial
certainty and salience on correlation perception operate in the opposite
direction to one another, which lead to the conclusion in
\chap{chap:adjusting_size} that point salience is a more likely
candidate mechanism. In an investigation into perceptions of global
means in scatterplots, it has been found that perceptions are biased
towards areas that contain larger points \cite{hong_2022}. These
findings form the theoretical basis behind the third hypothesis in the
present work.

\section{Hypotheses}\label{hypotheses-interactions}

A single experiment is presented in this chapter based on the effects of
adjusting point opacity and size on correlation estimation established
in \chap{chap:adjusting_opacity} and \chap{chap:adjusting_size}. Here,
previously independently tested point opacity and size manipulations are
tested in both typical orientation (point opacity/size is reduced with
residual magnitude) and inverted orientation (point opacity/size is
increased with residual magnitude). Throughout this chapter, reference
is made to \emph{congruent} and \emph{incongruent} conditions with
respect to the combination of point opacity and size decay functions
used. \emph{Congruent} conditions are those in which point opacity and
size decay act in the same direction (typical or inverted), while for
\emph{incongruent} conditions, point opacity and size decay act against
each other. Due to previous findings that non-linearly reducing point
opacity or size as a function of residual magnitude can bias correlation
estimates upwards, and that increasing point opacity and size with
residual magnitude can further bias estimates downwards, it is
hypothesised that:

\begin{itemize}
\tightlist
\item
  H1: an increased reduction in correlation estimation error will be
  observed when congruent typical orientation decay functions are used.
\item
  H2: the use of congruent inverted functions, such that point opacity
  and size are both increased with residual magnitude, will produce the
  least accurate estimates of correlation.
\end{itemize}

Owing to the finding from \chap{chap:adjusting_size} that point size is
a stronger channel for biasing correlation estimation than point
opacity, it is also hypothesised that:

\begin{itemize}
\tightlist
\item
  H3: there will be a significant difference in correlation estimates
  between the two incongruent orientation conditions.
\end{itemize}

\section{Method}\label{method-e4}

\subsection{Open Research}\label{open-research-chap6}

The experiment was conducted according to the principles of open and
reproducible research \cite{ayris_2018}. All data and code for the
original paper are maintained in a GitHub repository\footnote{https://github.com/gjpstrain/size\_opacity\_and\_scatterplots}.
This repository also features an implementation of a Docker container
that enables the full recreation of the computational environment in
which the original paper was written. The experiment itself is hosted on
GitLab\footnote{https://gitlab.pavlovia.org/Strain/size\_and\_opacity\_additive\_exp}.
The hypotheses and analysis plans were pre-registered with the Open
Science Framework (OSF)\footnote{https://osf.io/j32sk}, and no
deviations from them were made.

\subsection{Stimuli}\label{stimuli-e4}

The creation of the stimuli in this experiment follow the same general
principles outlined in Section \ref{creating-stimuli},
\chap{chap:gen_methods}. \texttt{ggplot2} (version 3.5.0) was used to
create the stimuli. Again, equation 6.1 was used to map point residuals
to opacity and size values, with a value of b = 0.25:

\begin{equation}
  point_{size/opacity} = 1 - b^{residual}
\end{equation}

For the changes made to point size in this experiment, a constant of 0.2
was added to each raw value, along with a scaling factor of 4; as in
\chap{chap:adjusting_size}, these adjustments resulted in the smallest
points having a width of 12 pixels on a 1920 \(\times\) 1080 pixel
monitor, which is consistent with the point size used in
\chap{chap:adjusting_opacity} and the minimum point size used in
\chap{chap:adjusting_size}. With regards to changing the opacities of
points, an alpha = 0.2 floor was implemented, as informal piloting
indicated low levels of visibility when very small points also had
particularly low opacity. In this experiment, point opacity or size
manipulations in which a reduction with residual magnitude takes place
are referred to as \emph{typical orientation}, while those in which
point opacity or size are increased with residual magnitude are referred
to as \emph{inverted orientation}. Additionally, as the current
experiment only examines combinations of point opacity and size decay
manipulations, the nature of these combinations are classified. When
both point size and opacity decay operate in the same direction, that
condition is referred to as \emph{congruent}. When they operate in
opposition to each other, those conditions are referred to as
\emph{incongruent}. Labelled examples of the stimuli used in this
experiment can be viewed in Figure~\ref{fig-exp4-examples-chap6}.

\begin{figure}

\centering{

\includegraphics[width=\textwidth]{6_interactions_opacity_size_files/figure-latex/fig-exp4-examples-chap6-1.pdf}

}

\caption{\label{fig-exp4-examples-chap6}Examples of the stimuli used in
Experiment 4, demonstrated with an \textit{r} value of 0.6.}

\end{figure}%

\subsection{Point Visibility Testing}\label{point-visibility-testing-e4}

Discussions about the opacities and sizes of stimuli are difficult in
the context of online, crowdsourced experiments. Unfortunately, it is
not possible to exert much control over the types of device participants
use beyond insisting on laptops or desktop computers. In particular, the
varying physical sizes, resolutions, and dynamic ranges of participants'
monitors can make commenting on the opacities and sizes of stimuli
difficult. On the other hand, carrying out this kind of experimentation
produces findings that are more resilient to different viewing contexts
than traditional lab-based work. It is key that the manipulations
employed here do not remove data; this includes removing data by
rendering it practically invisible. As in \chap{chap:adjusting_size},
point visibility testing is included to address these concerns.
Participants were shown scatterplots containing between 2 and 7 points;
these points were the same size and opacity as the smallest and least
opaque points used in the experimental stimuli. Participants were
instructed to enter the number of points present for each plot in a
textbox. Participants scored an average of 74.89\% (\(SD\) = 32.25\%).
Despite the use of the opacity floor and point size constant and scaling
factor, some of the smallest, least opaque stimuli used were clearly not
visible to participants. This was most likely due to low contrast
between the foreground (scatterplot points) and the background, as
Experiment 4, \chap{chap:adjusting_size} found visibility mostly
invariant to point size. In an idealised experimental setup, minimum
point opacity and size would be calibrated on a per-monitor basis.
Analysis including participants' performance on the point visibility
task as a fixed effect is detailed in Section \ref{results-e4}.

\subsection{Dot Pitch}\label{dot-pitch}

As in \chap{chap:adjusting_size}, a method for inferring the dot pitch
of participants' monitors was included in this experiment
\cite{screenscale}. Section \ref{dot-pitch-chap5} details precisely how
this was accomplished. Mean dot pitch was 0.60mm (\(SD\) = 0.09),
corresponding to a physical on-screen size of 7.80mm on a 1920
\(\times\) 1080 pixel monitor for the smallest points displayed on a
hypothetical 35.54 \(\times\) 20.00cm monitor. Analysis including dot
pitch as a fixed effect is provided in Section \ref{results-e4}.

\subsection{Design}\label{design-e4}

A fully repeated-measures, 2 \(\times\) 2 factorial design was employed.
The first factor, point opacity decay, had two levels; typical
orientation and inverted orientation. The second factor, point size
decay, also had two levels; typical orientation and inverted
orientation. Each participant saw each combination of opacity and size
decay function scatterplots for a total of 180 experimental items.
Participants viewed these experimental items, along with 6 attention
check items, in a fully randomised order. The experiment was built using
PsychoPy and hosted on Pavlovia\footnote{https://gitlab.pavlovia.org/Strain/size\_and\_opacity\_additive\_exp}.

\subsection{Procedure}\label{procedure-e4}

Ethical approval for this experiment was granted by the University of
Manchester's Computer Science Departmental Panel (Ref:
2022-14660-24397). Participants viewed the PIS and provided consent
through key presses in response to consent statements. Participants were
asked to provide their age and gender identity. Participants completed
the 5-item Subjective Graph Literacy test \cite{garcia_2016}, followed
by the screen scale and point visibility tasks described in Section
\ref{method-e4}. Following the completion of the pre-experimental tests,
participants were briefly shown examples of scatterplots with
correlations of 0.2, 0.5, 0.8, and 0.95. Section \ref{results-e4}
contains a discussion of the potential effects of this training. Two
practice trials were allowed before the experiment began. Participants
worked through a series of 180 experimental and 6 attention check trials
in a fully randomised order while being asked to use a slider (see
Figure \ref{fig-exp-illustrate}, \chap{chap:gen_methods}) to estimate
the correlation to two decimal places. Visual masks preceded each trial.
The attention check trials explicitly asked participants to set the
slider to 0 or 1.

\subsection{Participants}\label{participants-e4}

150 participants were recruited using the Prolific platform
\cite{prolific}. Normal or corrected-to-normal vision and English
fluency were required for participation. Participants who had completed
any of the experiments described in \chap{chap:adjusting_opacity} or
\chap{chap:adjusting_size} were prevented from participating. Data were
collected from 158 participants. 8 failed more than 2 out of 6 attention
check questions, and, as per the pre-registration, had their submissions
rejected from the study. The data from the remaining 150 participants
were included in the full analysis (76 male, 73 female, and 1
non-binary). Participants' mean age was 30.6 (\emph{SD} = 8.6). Mean
graph literacy score was 22.5 (\emph{SD} = 3.5) out of 30. The mean time
taken to complete the experiment was 37 minutes (SD = 12.3 minutes).

\section{Results}\label{results-e4}

To investigate the effects of combining point opacity and size decay
functions on participants' estimates of correlation, a linear mixed
effects model was built whereby the particular combination of point
opacity and size decay function employed is a predictor for the
difference between objective \emph{r} values for each plot and
participants' estimates of \emph{r}. Deviation coding was used for each
of the experimental factors, which allows comparison between means of
\emph{r} estimation error for each condition and the grand mean. This
model has random intercepts for items and participants, and random
slopes for participants with regards to the size decay factor. A
likelihood ratio test revealed that the model including the opacity and
size decay conditions as predictors explained significantly more
variance than the null (\(\chi^2\)(3) = 5,286.81, \emph{p} \textless{}
.001). There were significant fixed effects of opacity and size decay
function, as well as a significant interaction between the two.
Figure~\ref{fig-e4-estimates} shows the mean errors in correlation
estimation for each combination of conditions, along with 95\%
confidence intervals.

\begin{figure}

\centering{

\includegraphics[width=\textwidth]{6_interactions_opacity_size_files/figure-latex/fig-e4-estimates-1.pdf}

}

\caption{\label{fig-e4-estimates}Estimated marginal means for the four
conditions tested in Experiment 4. 95\% confidence intervals are shown.
The vertical dashed line represents no estimation error.}

\end{figure}%

\begin{table}

\caption{\label{tbl-contrasts-e4}Contrasts between different levels of
the opacity and size decay factors in Experiment 4. TO = Typical
Orientation. IO = Inverted Orientation.}

\centering{

\begin{tabular}[t]{llrl}
\toprule
\multicolumn{2}{c}{Contrast} & \multicolumn{2}{c}{Statistics} \\
\cmidrule(l{3pt}r{3pt}){1-2} \cmidrule(l{3pt}r{3pt}){3-4}
  &    & Z ratio & \textit{p}\\
\midrule
TO Size x IO Opacity & IO Size x IO Opacity & -10.945 & <0.001\\
TO Size x IO Opacity & TO Size x TO Opacity & 72.294 & <0.001\\
TO Size x IO Opacity & IO Size x TO Opacity & -2.256 & 0.108\\
IO Size x IO Opacity & TO Size x TO Opacity & 46.125 & <0.001\\
IO Size x IO Opacity & IO Size x TO Opacity & 17.838 & <0.001\\
\addlinespace
TO Size x TO Opacity & IO Size x TO Opacity & -37.438 & <0.001\\
\bottomrule
\end{tabular}

}

\end{table}%

The effects found were driven by significant difference between means of
correlation estimation error between all conditions besides that which
compares the two incongruent decay conditions. Statistical testing for
contrasts were performed using the \texttt{emmeans} package
\cite{lenth_2024}, and are provided in Table~\ref{tbl-contrasts-e4}.
Experiments 1, 2, and 3 all featured identical comparative baseline
conditions. In the former two experiments, this was the full contrast
condition (see \chap{chap:adjusting_opacity}), while in the latter, this
was the standard size condition (see \chap{chap:adjusting_size}). In the
current experiment, no baseline condition was used. Owing both to this
and the use of a linear mixed effects model with an interaction term,
the use of Cohen's \emph{d} as a measure of effect size would be
inappropriate. In its place, the amounts of variance in participants'
errors in correlation estimation explained by each fixed effect term and
the interaction term is represented as semi-partial R\textsuperscript{2}
\cite{nakagawa_2013}. These statistics were calculated using the
\texttt{r2glmm} package (version 0.1.2) \cite{r2glmm}, and are presented
along with model statistics in Table~\ref{tbl-model-stats-e4}.

\begin{table}

\caption{\label{tbl-model-stats-e4}Significances of fixed effects and
the interaction between them. Semi-partial R\textsuperscript{2} for each
fixed effect and the interaction term is also displayed in lieu of
effect sizes.}

\centering{

\begin{tabular}[t]{lrrrrll}
\toprule
  & Estimate & Standard Error & df & t-value & \textit{p} & R\textsuperscript{2}\\
\midrule
(Intercept) & 0.08 & 0.013 & 103.32 & 6.27 & <0.001 & \\
Size Decay & -0.14 & 0.005 & 148.39 & -25.77 & <0.001 & 0.104\\
Opacity Decay & 0.12 & 0.002 & 26327.21 & 63.71 & <0.001 & 0.087\\
Size Decay x Opacity Decay & 0.15 & 0.004 & 26327.13 & 38.47 & <0.001 & 0.034\\
\bottomrule
\end{tabular}

}

\end{table}%

\begin{figure}

\centering{

\includegraphics[width=\textwidth]{6_interactions_opacity_size_files/figure-latex/fig-e4-trial-number-1.pdf}

}

\caption{\label{fig-e4-trial-number}Comparing mean errors in correlation
estimation by trial number. Points represent unsigned mean errors for
each trial number. The plotted line is the locally estimated smoothed
curve, with the ribbon representing standard errors.}

\end{figure}%

Models including participants' graph literacy, their performance on the
point visibility task, the dot pitch of participants' monitors, and
which half of the experiment a particular correlation judgement took
place were built and compared with the experimental model. While no
significant effects of graph literacy (\(\chi^2\)(1) = 3.50, \emph{p} =
.061), performance on the point visibility task (\(\chi^2\)(1) = 1.29,
\emph{p} = .257), or dot pitch (\(\chi^2\)(1) = 1.52, \emph{p} = .218)
were found, there was a significant effect of training (\(\chi^2\)(1) =
23.78, \emph{p} \textless{} .001), with participants rating correlation
.01 lower during the second half. This drop suggests that having more
recently viewed the training plots may have increased participants'
estimates of correlation. To further analyse this variability, a model
was built including trial number, allowing for the analysis of error
purely as a function of when an experimental trial took place. A
significant effect of trial number is also found (\(\chi^2\)(1) = 29.31,
\emph{p} \textless{} .001) on participants' correlation estimation
errors. Figure~\ref{fig-e4-trial-number} shows participants' unsigned
mean errors in correlation estimation against trial number. Variability
in error, as represented by the ribbon, stabilised quickly and remained
stable for most of the experiment, only widening again around trial
number 170. The simplest explanation for this is that participants,
knowing they were coming to the end of the experiment, became less
vigilant and rushed their judgements. Regardless of statistical
significance, this effect is not large enough to warrant further
investigation, at least as it pertains to correlation estimation in
scatterplots.

\section{Discussion}\label{discussion-e4}

\begin{figure}

\centering{

\includegraphics[width=\textwidth]{6_interactions_opacity_size_files/figure-latex/fig-estimates-by-r-e4-1.pdf}

}

\caption{\label{fig-estimates-by-r-e4}Participants' mean errors in
correlation estimates grouped by factor and by \textit{r} value. The
dashed horizontal line represents perfect estimation. Participants were
most accurate when presented with the plots in the congruent, typical
orientation condition. Error bars show standard deviations of
estimates.}

\end{figure}%

Hypothesis 1 received full support in this experiment. The combination
of typical orientation opacity and size decay functions produced the
most accurate estimates of correlation, although this also resulted in a
marked over-correction and consequent overestimation for many values of
\emph{r} (see Figure~\ref{fig-estimates-by-r-e4}). The second hypothesis
also received support; the combination of inverted opacity and size
decay functions produced the least accurate estimates of correlation. No
support was found for the third hypothesis, that there would be a
significant difference in correlation estimates between inverted
orientation opacity/typical orientation size plots and typical
orientation opacity/inverted orientation size plots. There was, however,
a significant interaction term present, providing evidence that the
combination of opacity and size decay functions is not additive in
nature.

Further confirmatory evidence is found in favour of the phenomena
reported in \chap{chap:adjusting_opacity} and
\chap{chap:adjusting_size}. Namely, that while manipulations of both
point opacity and size in scatterplots have significant effects on
correlation estimation, the effect of changing point size is stronger,
and that while manipulations such as those described in this thesis can
influence estimates of positive correlation in either direction, typical
orientation manipulations are more powerful than inverted ones. As
expected, there is also an effect of congruency on the extent to which a
manipulation can bias estimates of correlation; redundant encoding, such
as that present here in congruent conditions, is known to support visual
grouping and segmentation \cite{nothelfer_2017}. The findings presented
here provide evidence that redundancy can be exploited to change
perceptions of correlation.

Given the consistent finding throughout this thesis that point size is a
stronger encoding channel for the purposes of altering perceptions of
correlation compared to point opacity, the lack of support for the third
hypothesis was unexpected. Tentatively, this may be a result of the
non-additive nature of combining point opacity and size manipulations.
Despite this, it was found that point size explained a greater
proportion of the variance (.104) in the experimental model compared to
point opacity.

Taking into account the work presented in this chapter, along with that
described in \chap{chap:adjusting_opacity} and
\chap{chap:adjusting_size}, recommendations can be made for designers of
correlation visualisations:

\begin{itemize}
\tightlist
\item
  When \emph{r} is between approximately 0.3 and 0.75, and the
  scatterplot in question is intended solely for the communication of
  correlation, designers may wish to implement the non-linear size decay
  function, as findings have shown it to produce the most accurate
  correlation estimates in this range.
\item
  Outside of this range, and with the same caveats in place, designers
  may wish to implement the opacity decay function described in
  \chap{chap:adjusting_opacity}; while its effect on correlation
  estimation is small, it does significantly increase estimation
  accuracy.
\item
  There exists a combination of size and opacity decay functions that
  produces accurate correlation estimates while maintaining the
  increased \emph{r} estimation precision expected with high \emph{r}
  values. Finding this will require extensive future testing.
\end{itemize}

\subsection{Combining Manipulations}\label{combining-manipulations}

Figure~\ref{fig-e4-estimates} and Figure~\ref{fig-estimates-by-r-e4}
show how, on average, the combination of typical orientation opacity and
size decay functions results in an overestimation of \emph{r} for the
majority of values. While not solving the underestimation problem
directly, it demonstrates that with regards to using point opacity and
size manipulations to change estimates of correlation, there appear to
be few limitations. If correlation estimation can be over-corrected, as
in the typical orientation condition here, then there exists a
\emph{tuning} of the opacity and size decay parameters such that the
degree of correction is appropriate; Section \ref{future-work-e4}
explores the work that might be done to achieve this. The combination of
inverted orientation opacity and size decay functions also had the
expected effect, producing the lowest and least accurate estimates of
correlation. Combining inverted manipulations did not, however,
significantly change the shape of the estimation curve (see
Figure~\ref{fig-estimates-by-r-e4}). In addition to non-additive
interaction, the effects observed operate differently depending on the
direction of the change induced in perception. This finding may also
explain the lack of support for the third hypothesis, that there would
be a significant difference in estimation error between the two
incongruent conditions. Despite the size channel being more powerful
than opacity with regards to influencing correlation estimates, the fact
that this power depends on the direction the function is set causes
incongruent decay conditions to act against each other in unexpected
ways. Indeed, the incongruent condition that used a typical orientation
size decay function exhibited lower mean error than the one using
inverted orientation size decay (see
Figure~\ref{fig-estimates-by-r-e4}), however in each case opacity decay
appears to have blunted the power of the size decay function to the
extent that the difference in errors is not statistically significant.

\subsection{Estimation Precision}\label{estimation-precision}

Previous work has been consistent regarding the finding that \emph{r}
estimation precision increases with the objective \emph{r} value
displayed in the scatterplot
\cite{doherty_2007, rensink_2012, rensink_2014, rensink_2017}. The
experiments carried out in \chap{chap:adjusting_opacity} and
\chap{chap:adjusting_size} found that in some cases, precision in
\emph{r} estimation is constant across the range of \emph{r} values
investigated. For example, the use of a size decay function, whether
linear or non-linear decay in typical or inverted directions, results in
no change in \emph{r} estimation precision (see Section
\ref{results-e3}, \chap{chap:adjusting_size}). When point opacity was
altered in the same way, only an inverted decay function does not
exhibit the conventional increase in \emph{r} estimation precision with
increasing objective \emph{r} value. In the experiment described
presently, precision in \emph{r} estimation increased whenever a typical
orientation opacity decay function was used. This may be part of the
moderating effect of point opacity decay on the size decay function; the
visual character of scatterplots with high \emph{r} values that use the
size decay function eliminates the usual increase in precision one would
expect, however the introduction of the opacity decay function
normalises this to the point where precision is restored.

\subsection{Relative Contributions of Opacity and Size
Decay}\label{relative-contributions-of-opacity-and-size-decay}

\begin{figure}

\centering{

\includegraphics[width=\textwidth]{6_interactions_opacity_size_files/figure-latex/fig-all-est-curves-1.pdf}

}

\caption{\label{fig-all-est-curves}Plotting errors in \emph{r}
estimation against objective \emph{r} values for opacity and size decay
functions. On the left, opacity and size decay functions in combination
in the typical orientation congruent condition from the current
experiment. The plots in the centre show estimation error for opacity
and size decay functions in isolation from previous chapters. The
right-hand plot averages the comparative baseline (standard scatterplot)
conditions from the previous two chapters.}

\end{figure}%

Incorporating the data gathered in \chap{chap:adjusting_opacity} and
\chap{chap:adjusting_size} allows for the comparison of estimation
curves for size decay and opacity decay both in isolation and
combination. Figure~\ref{fig-all-est-curves} shows correlation
estimation error curves in the present experiment (typical orientation
congruent), and in the non-linear decay conditions for both opacity
decay (\chap{chap:adjusting_opacity}) and size decay
(\chap{chap:adjusting_size}) conditions. The ``no manipulations
present'' plot is averaged from the results of Experiments 1 to 3. Using
opacity decay alone significantly changes the amplitude of the
estimation curve while leaving its shape intact; this can be seen by
comparing opacity decay and standard scatterplot plots in
Figure~\ref{fig-all-est-curves}. Using size decay changes both the
amplitude and the shape of the estimation curve. When opacity and size
decay functions are combined, the shape of the curve is most similar to
that observed when size decay is employed in isolation. This is in line
with findings, both here and in previous work \cite{hong_2022}, that
size is a more potent encoding channel for the manipulation of
perceptual estimates derived from scatterplots. It would appear then
that the addition of the opacity curve moderates the effect of size
decay as a function of the objective \emph{r} value itself, without
affecting the general shape of the curve.

Using an opacity decay function in isolation has a small effect on
correlation estimation. It does little to change the shape of the
underestimation curve, but rather slightly biases \emph{r} estimates
upwards to partially correct for the underestimation observed with
standard scatterplots. Importantly, it also preserves the increase in
correlation estimation precision that we expect as the true \emph{r}
value increases. Using the size decay function in isolation has a more
dramatic effect. Size decay over-corrects at lower \emph{r} values,
leading to an overestimation effect; at higher values, underestimation
still occurs. At mid-range values of \emph{r}, however, the size decay
function performs significantly better than all others. One option for
tuning correlation estimation using the decay functions described in
this thesis would therefore be to use size decay in isolation for
mid-range \emph{r} values (around 0.3 to 0.75), and to use opacity decay
in isolation outside of this range. Combining the decay functions,
however, allows for the exploitation of the power of the size decay
function while maintaining the expected increase in \emph{r} estimation
precision that the opacity decay function confers. The simple
combination used in the present experiment does not represent an ideal
tuning, as participants overestimated \emph{r} for the majority of
values; the findings here do, however, confirm that there is the scope
to bias \emph{r} estimation to greater degrees compared to scatterplots
that use point opacity or size decay functions in isolation. Precise
values for the contributions of each encoding channel when they are used
simultaneously would be needed to begin doing this work. To describe the
effect that each decay function (and their combination) has on
correlation estimates, new curves can be derived that compare estimates
made with opacity and size decay to those made without.

\begin{figure}[H]

\centering{

\includegraphics[width=\textwidth]{6_interactions_opacity_size_files/figure-latex/fig-power-plot-e4-1.pdf}

}

\caption{\label{fig-power-plot-e4}Power is the difference between what
is observed when a decay function/combination of decay functions is used
and what is observed when no manipulation is used. The dashed line
represents the power that would be required to correct for the observed
underestimation of correlation in scatterplots. The integral of each
power curve over \textit{r} is provided, as well as the difference
between this integral and the integral of each required-power curve over
\textit{r}.}

\end{figure}%

This contribution is termed \emph{power}, and is visualised in
Figure~\ref{fig-power-plot-e4}. As can be seen in the right-hand plot,
size decay in isolation provides the closest to the required level of
correction, and combining point opacity and size decay results in gross
overestimation. Figure~\ref{fig-power-plot-e4} also includes the
integral of each power curve over \emph{r} as a measure of the total
power of each curve. Subtracting the integral of the required power
curve from the integral of the observed power curve allows a numerical
value to be calculated for the discrepancy between observed and required
power for each manipulation.

\subsection{Mechanisms}\label{mechanisms}

Findings in \chap{chap:adjusting_opacity} and \chap{chap:adjusting_size}
made the case for opacity and size decay acting primarily through point
salience and perceptual weighting, with the caveat that spatial
certainty also plays a small part in the mechanism behind the effects of
point size decay. The results in the present experiment are supportive
of this notion, with the highest and lowest mean estimates being
observed in the congruent typical and congruent inverted orientation
conditions respectively. These findings also support dot density
\cite{yang_2023} and featured-based attention bias accounts
\cite{hong_2022, sun_2016}. As all these mechanisms would produce
similar results, making conclusions about the relative potential
contributions of each is difficult. Nevertheless, on a higher level, the
evidence generally points towards a probability distribution account
\cite{rensink_2017, rensink_2022}. On a lower level, however, numerous
candidate mechanisms exist. The results from the current experiment
provide further evidence for a point salience/perceptual weighting
account. Hong et al.~\cite{hong_2022} found that the inclusion of larger
and more opaque scatterplot points was able to bias estimates of
positional means, but that the relative contributions (weights) of these
visual features with regards to perception change as a function of the
ranges of opacities and sizes used. It is clear from this evidence and
the present that the perceptual weightings of opacity and size are not
the same.

\subsection{Limitations}\label{limitations-e4}

First, participants' performance on the point visibility task. This was
poor, with an average of only 74.89\%. It would seem that despite the
implementation of the opacity floor and the size scaling factor and
constant, many stimuli were simply not visible. While modelling
indicates that this did not have a significant effect on errors of
correlation estimation, for many participants it will have seemed as if
data had been removed, violating the intended aims of the manipulations.
Addressing this properly would require a by-participant calibration of
the necessary minimum point opacity and size values to ensure
visibility, as these values will change as a function of head-to-monitor
distance and monitor characteristics.

While evidence has been found that combining point opacity and size
decay is not additive, it is difficult to comment precisely on what
proportions of the observed effects are a result of each manipulation.
Previous work \cite{hong_2022}, as well as the work carried out here and
in \chap{chap:adjusting_size}, suggest that point size has much greater
potential to change perceptual estimates, although ascertaining the
exact ways in which opacity and size manipulations interact would
require future work.

Due to the extensive testing of ``no manipulations present'' conditions
(\emph{full opacity} in \chap{chap:adjusting_opacity} and \emph{standard
size} in \chap{chap:adjusting_size}), a comparative baseline was not
included in this experiment. At the time it was judged that the
increased cost and experimental length was not worth the inclusion of
further conditions beyond the four that were tested.

Channels such as point size, colour, opacity, and shape have been used
in past work to encode variables beyond the standard two typically used
in scatterplots \cite{smart_2019, hong_2022}. While this work focuses
purely on correlation estimation, these techniques are likely to lead to
incorrect interpretations when scatterplots are designed with other
tasks in mind. Given evidence that size, shape, and colour are not
entirely separable scatterplot features \cite{smart_2019}, if viewers
assume that variations in point opacity/size correspond to additional
encoded variables, confounds in interpretation may be introduced. If
plots such as the ones presented here were to appear in the wild, it
would be necessary to clarify that they were designed to aid in the
rapid and intuitive interpretation of correlation (and \emph{only}
this). Irrespective of the potential for misinterpretation, strong
baseline evidence for a perceptual effect of changing point size and
opacity in scatterplots is provided that may be expanded on and further
exploited in future work.

\subsection{Future Work}\label{future-work-e4}

The finding that combining point opacity and size manipulations
significantly reduced point visibility suggests the potential for future
work investigating the calibration of scatterplot visual features for a
particular participant. Doing this would require a more dynamic
experimental environment in which stimuli could be regenerated
on-the-fly, but would allow researchers to test perceptions more
accurately.

Experimental limitations meant that this work did not feature a
comparative baseline condition, unlike in \chap{chap:adjusting_opacity}
and \chap{chap:adjusting_size}. Future work may wish to re-test all or
some of the conditions described here in addition to a baseline, no
manipulations condition.

There is evidence that viewers overestimate correlation in negatively
correlated scatterplots \cite{sher_2017}. Findings that correlation
perception in negatively correlated scatterplots functions symmetrically
to that of positively correlated scatterplots \cite{harrison_2014}
suggest that the techniques implemented here may be used (in a
symmetrical manner) to address the overestimation bias. Evidence that
the influence of size and opacity decay functions changes according to
the direction they are operating in means experimental work with
negatively correlated scatterplots would be required, and results may
differ significantly from findings related to the underestimation of
correlation in positively correlated scatterplots.

All the work in the present chapter, along with that in
\chap{chap:adjusting_opacity} and \chap{chap:adjusting_size}, uses the
same equation (reproduced below) to relate point residuals to specific
opacity and size values.

\begin{equation}
  point_{size/opacity} = 1 - b^{residual}
\end{equation}

Given the finding that the combination of point opacity and size decay
is not additive, there are a multitude of parameters that may be
adjusted and tested to explore the relative contributions of each to the
effects seen. The value of \emph{b}, which so far has only been b =
0.25, is one such parameter, and controls the severity of the fall-off
in point opacity or size of the decay function in question. The opacity
floor, size scaling factor, and size constant are other values that
might be changed. Of course, the equation used is not exhaustive; future
work may wish to investigate more complicated equations that link the
objective \emph{r} value to the decay condition.

Future experimental work may use the major axis through the probability
ellipse instead of the regression line as a baseline to change point
sizes and opacities; evidence that people often report the major axis
when asked to visually estimate the regression line \cite{collyer_1990}
suggests that this may produce a different pattern of results from those
seen here. If changes in dot density are driving changes in correlation
estimates, the congruent conditions here are an example of redundant
encoding. Future work may explore using different channels to
redundantly encode dot density, such as marker shape, orientation, or
colour. Further testing of opacity and size manipulations in isolation
and combination using different decay function parameters will allow
researchers to build a more complete picture of how these visual
features impact correlation estimation, and how we can exploit them to
correct for well-known biases.

Finally, while point salience/perceptual weighting is put forward as the
most likely driver of the effects observed, the data gathered here do
not explain the differences in the shapes of the observed correlation
estimation curves (see Figure~\ref{fig-all-est-curves}). The context
that a particular point manipulation is presented in, including the
objective \emph{r} value of the scatterplot, interacts with point
opacity and size adjustments in complex ways. Future work may wish to
use the same decay conditions while fixing the objective \emph{r} value
to explore these interactions in greater detail.

\section{Conclusion}\label{conclusion-interactions}

In a single experiment that combined the point opacity manipulation from
\chap{chap:adjusting_opacity} and the point size manipulation from
\chap{chap:adjusting_size}, evidence is provided that this combination
is not additive in nature. In addition to opening up questions about the
complexity of this interaction, this work shows that there is scope to
bias correlation estimates significantly; the finding that the
combination of non-linear opacity and size decay functions produces
marked overestimation emphasises this. Exploring exactly how these decay
functions interact with a view to tuning correlation perception is
beyond the scope of this thesis; the foundation has, however, been laid.




\end{document}
