\documentclass[../main.tex]{subfiles}
\begin{document}


\section{Abstract}\label{abstract-interactions}

\chap{chap:adjusting_opacity} and \chap{chap:adjusting_size} each
provide evidence for the effects of changing the opacities and sizes of
scatterplot points on people's performance on a correlation estimation
task. It is clear from the results of the experiments presented in those
chapters, however, that the effects of changing point opacity and size
on correlation estimation in positively correlated scatterplots are
different, both with regards to the strength of the correction provided
and the shape of the estimation curve produced. Mechanistically,
however, point opacity and size may operate similarly by reducing the
influence of more exterior points in scatterplots. To further
investigate these mechanisms, and in the interest of providing a more
complete description of the ways in which changing point opacity and
size can bias correlation estimation, I present a single experiment
study combining the point opacity and size manipulations from
\chap{chap:adjusting_opacity} and \chap{chap:adjusting_size}. In a
condition where point opacity and size are reduced as a function of
residual magnitude, correlation estimation is significantly biased
upwards; a finding that suggests the effects of changing point opacity
and size are not linearly additive.

\section{Introduction}\label{introduction-interactions}

The work presented in \chap{chap:adjusting_opacity} and
\chap{chap:adjusting_size} shows that point opacity and size adjustments
can be used to bias perceptions of correlation in positively correlated
scatterplots. These findings are in opposition to work finding both bias
and variability in correlation perception to be invariant to both
uniform and irregular changes in point opacities and sizes
\cite{rensink_2012, rensink_2014}. Of course, when attempting to provide
tools for visualisation designers to design \emph{better}
visualisations, more options are preferable to few. In this spirit,
then, I endeavoured to answer the next big question; what happens when
point opacity and size manipulations are combined? Answering this
question would not only allow for a deeper exploration of the potential
mechanisms behind each decay function, but would further empower
visualisation designers with the knowledge of how altering point
opacities and sizes might affect people's interpretations of the
correlations displayed therein.

\section{Related Work}\label{related-work-interactions}

\section{Hypotheses}\label{hypotheses-interactions}

A single experiment is present in this chapter based on the effects of
adjusting point opacity and size on correlation estimation established
in \chap{chap:adjusting_opacity} and \chap{chap:adjusting_size}. Here,
previously independently tested point opacity and size manipulations are
tested in both typical orientation (point opacity/size is reduced with
residual magnitude) and inverted orientation (point opacity/size is
increased with residual magnitude). Throughout this chapter, referral is
made to \emph{congruent} and \emph{incongruent} conditions with respect
to the combination of point opacity and size decay functions.
\emph{Congruent} conditions are those in which point opacity and size
decay act in the same direction (typical or inverted), while for
\emph{incongruent} conditions, point opacity and size decay act against
each other. Due to previous findings that non-linearly reducing point
opacity or size as a function of residual magnitude can bias correlation
estimates upwards, it is hypothesised that:

\begin{itemize}
\tightlist
\item
  H1: an increased reduction in correlation estimation error will be
  observed when congruent typical orientation decay functions are used.
\item
  H2: the use of a congruent inverted function, such that point opacity
  and size are both increased with residual magnitude, will produce the
  least accurate estimates of correlation.
\end{itemize}

Owing to the finding from \chap{chap:adjusting_size} that point size is
a stronger channel for biasing correlation estimation than point
opacity, it is also hypothesised that:

\begin{itemize}
\tightlist
\item
  H3: there will be a significant difference in correlation estimates
  between the two incongruent orientation conditions.
\end{itemize}

\section{Methods}\label{methods-e4}

\subsection{Stimuli}\label{stimuli-e4}

The creation of the stimuli in this experiment follows the same general
principles outline in Section \ref{creating-stimuli},
\chap{chap:gen_methods}. Again, equation 6.1 was used to map point
residuals to opacity and size values:

\begin{equation}
  point_{size/opacity} = 1 - b^{residual}
\end{equation}

For the changes made to point size in this experiment, a constant of 0.2
was added to each raw value, along with a scaling factor of 4; as in
\chap{chap:adjusting_size}, these adjustments resulted in the smallest
points having a width of 12 pixels on a 1920x1080 pixel monitor, which
is consistent with the point size used in \chap{chap:adjusting_opacity}
and the minimum point size used in \chap{chap:adjusting_size}. With
regards to changing the opacities of points, an alpha = 0.2 floor was
implemented, as informal piloting indicated low levels of visibility
when very small points were especially transparent. In this experiment,
point opacity or size manipulations in which a reduction with residual
magnitude takes place are referred to as \emph{typical orientation},
while those in which point opacity or size are increased with residual
magnitude are referred to as \emph{inverted orientation}. Additionally,
as the current experiment only examines combinations of point opacity
and size decay manipulations, the nature of these combinations are
classified. When both point size and opacity decay operate in the same
direction, that condition is referred to as \emph{congruent}. When they
operate in opposition to each other, those conditions are referred to as
\emph{incongruent}. Labelled examples of the stimuli used in this
experiment can be viewed in Figure~\ref{fig-exp4-examples-chap6}.

\begin{figure}

\centering{

\includegraphics[width=\textwidth]{6_interactions_opacity_size_files/figure-latex/fig-exp4-examples-chap6-1.pdf}

}

\caption{\label{fig-exp4-examples-chap6}Examples of the stimuli used in
experiment 4, demonstrated with an \textit{r} value of 0.6.}

\end{figure}%

\subsection{Point Visibility Testing}\label{point-visibility-testing}

Discussions about the opacities and sizes of stimuli are difficult in
the context of online, crowdsourced experiments. Unfortunately, it is
not possible to exert much control over the types of device participants
use beyond insisting on laptops or desktop computers. In particular, the
varying physical sizes, resolutions, and dynamic ranges of participants'
monitors can make commenting on the opacities and sizes of stimuli
difficult. On the other hand, carrying out this kind of experimentation
produces findings that are more resilient to different viewing contexts
than traditional lab-based work. It is key that the manipulations
employed here do not remove data; this includes removing data by
rendering it invisible. As in \chap{chap:adjusting_size}, point
visibility testing is included to address these concerns. Participants
were shown size scatterplots containing between 2 and 7 points; these
points were the same size and opacity and the smallest and most opaque
points used in the experimental stimuli. Participants were instructed to
enter the number of points present for each plot in a textbox.
Participants scored an average of 74.89\% (\(SD\) = 32.25\%). Despite
the use of the opacity floor and point size constant and scaling factor,
some of the smallest, least opaque stimuli used were clearly not visible
to participants. This was most likely due to low contrast between the
foreground (scatterplot points) and the background, as experiment 4,
\chap{chap:adjusting_size} found visibility mostly invariant to point
size. In an idealised experimental setup, minimum point opacity and size
would need to calibrated on a per-monitor basis. Analysis including
participants' performance on the point visibility task as a fixed effect
is detailed in Section \ref{results-e4}

\subsection{Dot Pitch}\label{dot-pitch}

As in \chap{chap:adjusting_size}, a method for inferring the dot pitch
of participants' monitors was included in this experiment
\cite{screenscale}. Section \ref{dot-pitch-chap5} details precisely how
this was accomplished. Mean dot pitch was 0.60mm (\(SD\) = 0.09),
corresponding to a physical on-screen size of 7.80mm on a 1920
\(\times\) 1080 pixel monitor for the smallest points displayed on a
hypothetical 35.54 \(\times\) 20.00cm monitor. Analysis including dot
pitch as a fixed effect is provided in Section \ref{results-e4}

\subsection{Design}\label{design-e4}

A full repeated-measures, 2 \(\times\) 2 factorial design was employed.
Each participant saw each combination of opacity and size decay function
scatterplots for a total of 180 experimental items. Participants viewed
these experimental items, along with 6 attention check items, in a fully
randomised order. The experiment is hosted on Pavlovia \footnote{https://gitlab.pavlovia.org/Strain/size\_and\_opacity\_additive\_exp}.

\subsection{Procedure}\label{procedure-e4}

Participants viewed the PIS and provided consent through key presses in
response to consent statements. Participants were asked to provide their
age and gender identity. Participants completed the 5-item Subjective
Graph Literacy test \cite{garcia_2016}, followed by the screen scale and
point visibility tasks described in Section \ref{methods-e4}. Following
the completion of the pre-experimental tests, participants were briefly
shown examples of scatterplots with correlations of 0.2, 0.5, 0.8, and
0.95. Section \ref{results-e4} contains a discussion of the potential
effects of this training. Two practice trials were allowed before the
experiment began. Participants worked through a series of 180
experimental and 6 attention check trials in a fully randomised order
while being asked to use a slider (see Figure \ref{fig-slider},
\chap{chap:general_methods}) to estimate the correlation to two decimal
places. Visual masks preceded each trial. The attention check trials
explicitly asked participants to set the slider to 0 or 1.

\subsection{Participants}\label{participants-e4}

Normal to corrected-to-normal vision and English fluency were required.
Participants who had completed any of the experiments described in
\chap{chap:adjusting_opacity} or \chap{chap:adjusting_size} were
prevented from participating. Data were collected from 158 participants.
8 failed more than 2 out of 6 attention check questions, and, as per the
pre-registration, had their submissions rejected from the study. The
data from the remaining 150 participants were included in the full
analysis (51\% male, 49 \% female, and 1\% non-binary). Participants'
mean age was 30.65 (\emph{SD} = 8.64). Mean graph literacy score was
22.49 (\emph{SD} = 3.55). The average time taken to complete the
experiment was 37 minutes (SD = 12.3 minutes).

\section{Results}\label{results-e4}

To investigate the effects of combining point opacity and size decay
functions on participants' estimates of correlation, a linear mixed
effects model was built whereby the particular combination of point
opacity and size decay function employed is a predictor for the
difference between objective \emph{r} values for each plot and
participants' estimates of \emph{r}. Deviation coding was used for each
of the experimental factors, which allows comparison between means of
\emph{r} estimation error and the grand mean. This model has random
intercepts for items and participants, and random slopes for
participants with regards to the size decay factor. A likelihood ratio
test revealed that the model including the opacity and size decay
conditions as predictors explained significantly more variance than the
null (\(\chi^2\)(3) = 5,286.81, \emph{p} \textless{} .001). There were
significant fixed effects of opacity and size decay function, as well as
a significant interaction between the two. Figure~\ref{fig-e4-estimates}
shows the mean errors in correlation estimation for each combination of
conditions, along with 95\% confidence intervals.

\begin{figure}

\centering{

\includegraphics[width=\textwidth]{6_interactions_opacity_size_files/figure-latex/fig-e4-estimates-1.pdf}

}

\caption{\label{fig-e4-estimates}Estimated marginal means for the four
conditions tested in experiment 4. 95\% confidence intervals are shown.
The vertical dashed line represents no estimation error.}

\end{figure}%

\begin{table}

\caption{\label{tbl-contrasts-e4}Contrasts between different levels of
the opacity and size decay factors in experiment 4.}

\centering{

\begin{tabular}[t]{llrl}
\toprule
\multicolumn{2}{c}{Contrast} & \multicolumn{2}{c}{Statistics} \\
\cmidrule(l{3pt}r{3pt}){1-2} \cmidrule(l{3pt}r{3pt}){3-4}
  &    & Z ratio & \textit{p}\\
\midrule
TO Size x IO Opacity & IO Size x IO Opacity & -10.945 & <0.001\\
TO Size x IO Opacity & TO Size x TO Opacity & 72.294 & <0.001\\
TO Size x IO Opacity & IO Size x TO Opacity & -2.256 & 0.108\\
IO Size x IO Opacity & TO Size x TO Opacity & 46.125 & <0.001\\
IO Size x IO Opacity & IO Size x TO Opacity & 17.838 & <0.001\\
\addlinespace
TO Size x TO Opacity & IO Size x TO Opacity & -37.438 & <0.001\\
\bottomrule
\end{tabular}

}

\end{table}%

The effects found were driven by significant difference between means of
correlation estimation error between all conditions besides that which
compares the two incongruent decay conditions. Statistical testing for
contrasts were performed using the \texttt{emmeans} package
\cite{lenth_2024}, and are provided in Table~\ref{tbl-contrasts-e4}.
Experiments 1, 2, and 3 all featured a comparative baseline condition.
In the former two experiments, this was the full contrast condition (see
\chap{chap:adjusting_opacity}), while in the latter, this was the
standard size condition (see \chap{chap:adjusting_size}). In the current
experiment, no baseline condition was used. Owing both to this and the
use of a linear mixed effects model with an interaction term, the use of
Cohen's \emph{d} as a measure of effect size would be inappropriate. In
its place, the amounts of variance in participants' errors in
correlation explanation explained by each fixed effect term and the
interaction term is represented as semi-partial R\textsuperscript{2}
\cite{nakagawa_2013}. These statistics were calculated using the
\texttt{r2glmm} package (version 0.1.2) \cite{r2glmm}, and are presented
along with model statistics in Table~\ref{tbl-model-stats-e4}.

\begin{table}

\caption{\label{tbl-model-stats-e4}Significances of fixed effects and
the interaction between them. Semi-partial R\textsuperscript{2} for each
fixed effect and the interaction term is also displayed in lieu of
effect sizes.}

\centering{

\begin{tabular}[t]{lrrrrll}
\toprule
  & Estimate & Standard Error & df & t-value & \textit{p} & R\textsuperscript{2}\\
\midrule
(Intercept) & 0.08 & 0.013 & 103.32 & 6.27 & <0.001 & \\
Size Decay & -0.14 & 0.005 & 148.39 & -25.77 & <0.001 & 0.104\\
Opacity Decay & 0.12 & 0.002 & 26327.21 & 63.71 & <0.001 & 0.087\\
Size Decay x Opacity Decay & 0.15 & 0.004 & 26327.13 & 38.47 & <0.001 & 0.034\\
\bottomrule
\end{tabular}

}

\end{table}%

Models including participants' graph literacy, their performance on the
point visibility task, the dot pitch of participants' monitors, and
which half a particular correlation judgement took place were built and
compared with the experimental model. While no significant effects of
graph literacy (\(\chi^2\)(1) = 3.50, \emph{p} = .061), performance on
the point visibility task (\(\chi^2\)(1) = 1.29, \emph{p} = .257), or
dot pitch (\(\chi^2\)(1) = 1.52, \emph{p} = .218) were found, there was
a significant effect of training (\(\chi^2\)(1) = 23.78, \emph{p}
\textless{} .001), with participants rating correlation .01 lower during
the second half. This drop suggests that having more recently viewed the
training plots may have increased participants estimates of correlation.
To further analyse this variability, a model was built including trial
number, allowing for the analysis of error. A significant effect of
trial number is also found (\(\chi^2\)(1) = 29.31, \emph{p} \textless{}
.001) on participants' correlation estimation errors.

\begin{figure}

\centering{

\includegraphics[width=\textwidth]{6_interactions_opacity_size_files/figure-latex/fig-e4-trial-number-1.pdf}

}

\caption{\label{fig-e4-trial-number}Comparing mean errors in correlation
estimation by trial number. Points represent unsigned mean errors for
each trial number. The plotted line is the locally estimated smoothed
curve, with the ribbon representing standard errors.}

\end{figure}%

Figure~\ref{fig-e4-trial-number} shows participants' unsigned mean
errors in correlation estimation against trial number. Variability in
error, as represented by the ribbon, stabilised quickly and remained
stable for most the experiment, only widening again around trial number
170. The simplest explanation for this is that participants, knowing
they were coming to the end of the experiment, became less vigilant and
rushed their judgements more. Regardless of statistical significance,
this effect is not large enough to warrant further investigation, at
least as it pertains to correlation estimation in scatterplots.

\begin{figure}

\centering{

\includegraphics[width=\textwidth]{6_interactions_opacity_size_files/figure-latex/fig-estimates-by-r-e4-1.pdf}

}

\caption{\label{fig-estimates-by-r-e4}Participants' mean errors in
correlation estimates grouped by factor and by \textit{r} value. The
dashed horizontal line represents perfect estimation. Participants were
most accurate when presented with the plots featuring the non-linear
size decay function. Error bars show standard deviations of estimates.}

\end{figure}%

\section{Discussion}\label{discussion-e4}

Hypothesis 1 received full support in this experiment. The combination
of typical orientation opacity and size decay functions produced the
most accurate estimates of correlation, although this also resulted in a
marked over-correction and consequent overestimation for many values of
\emph{r} (see Figure~\ref{fig-estimates-by-r-e4}). The second hypothesis
also received support; the combination of inverted opacity and size
decay functions produced the least accurate estimates of correlation. No
support was found for the third hypothesis, that there would be a
significant difference in correlation estimates between inverted
orientation opacity/typical orientation size plots and typical
orientation opacity/inverted orientation size plots. There was, however,
a significant interaction term present, providing evidence that the
combination of opacity and size decay functions is not additive in
nature.

Further confirmatory evidence is found in favour of the phenomena
reported in \chap{chap:adjusting_opacity} and
\chap{chap:adjusting_size}. Namely, that while manipulations of both
point opacity and size in scatterplots have significant effects on
correlation estimation, the effect of changing point size is stronger,
and that while manipulations such as those described in this thesis can
influence estimates of positive correlation in either direction, typical
orientation manipulations are more powerful than inverted ones. As
expected, there is also an effect of congruency on the extent to which a
manipulation can bias estimates of correlation; redundant encoding, such
as that present here in congruent conditions, is known to support visual
grouping and segmentation \cite{nothelfer_2017}. The findings presented
here provide evidence that redundancy can be exploited to change
perceptions of correlation.

Given the consistent finding throughout this thesis that point size is a
stronger encoding channel for the purposes of altering perceptions of
correlation compared to point opacity, the lack of support for the third
hypothesis was unexpected. Tentatively, this may be a result of the
non-additive nature of combining point opacity and size manipulations.
Despite this, it was found that point size explained a greater
proportion of the variance (.104) in the experimental model compared to
point opacity.

Taking into account the work presented in this chapter, along with that
described in \chap{chap:adjusting_opacity} and
\chap{chap:adjusting_size}, recommendations can be made for designers of
correlation visualisations:

\begin{itemize}
\tightlist
\item
  When \emph{r} is between approximately 0.3 and 0.75, and the
  scatterplot in question is intended solely for the communication of
  correlation, designers may wish to implement the non-linear size decay
  function, as findings have shown it to produce the most accurate
  correlation estimates in this range.
\item
  Outside of this range, and with the same caveats in place, designers
  may wish to implement the opacity decay function described
  in\chap{chap:adjusting_opacity}; while its effect on correlation
  estimation is small, it does significantly increase estimation
  accuracy.
\item
  There exists a combination of size and opacity decay functions that
  produces accurate correlation estimates while maintaining the
  increased \emph{r} estimation precision expected with high \emph{r}
  values. Finding this will require extensive future testing.
\end{itemize}

\subsection{Combining Manipulations}\label{combining-manipulations}

fig-e4-estimates and Figure~\ref{fig-estimates-by-r-e4} show how, on
average, the combination of typical orientation opacity and size decay
functions results in an overestimation of \emph{r} for the majority of
values. While not solving the underestimation problem directly, it
demonstrates that with regards to using point opacity and size
manipulations to change estimates of correlation, there appear to be few
limitations. If correlation estimation can be over-corrected, as in the
typical orientation condition here, then there exists a \emph{tuning} of
the opacity and size decay parameters such that the degree of correction
is appropriate; Section \ref{future-work-e4} explores the work that
might be done to achieve this. The combination of inverted inverted
orientation opacity and size decay functions also had the expected
effect, producing the lowest and least accurate estimates of
correlation. Combining inverted manipulations did not, however,
significantly change the shape of the estimation curve (see
Figure~\ref{fig-estimates-by-r-e4}). In addition to non-additive
interaction, the effects observed operate differently depending on the
direction of the change induced in perception. This finding may also
explain the lack of support for the third hypothesis, that there would
be a significant difference in estimation error between the two
incongruent conditions. Despite the size channel being more powerful
than opacity with regards to influencing correlation estimates, the fact
that this power depends on the direction the function is set causes
incongruent decay conditions to act against each other in unexpected
ways. Indeed, the incongruent condition that used a typical orientation
size decay function exhibited lower mean error than the one using
inverted orientation size decay (see
Figure~\ref{fig-estimates-by-r-e4}), however in each case opacity decay
appears to have blunted the power of the size decay function to the
extent that the difference in errors is not statistically significant.

\subsection{Estimation Precision}\label{estimation-precision}

\subsection{Relative Contributions of Opacity and Srize
Decay}\label{relative-contributions-of-opacity-and-srize-decay}

\subsection{Mechanisms}\label{mechanisms}

\subsection{Limitations}\label{limitations}

\subsection{Future Work}\label{future-work-e4}

\section{Conclusion}\label{conclusion-interactions}




\end{document}
