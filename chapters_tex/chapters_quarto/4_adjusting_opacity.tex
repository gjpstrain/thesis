\documentclass[../main.tex]{subfiles}
\begin{document}


\section{Abstract}\label{abstract-adjusting-opacity}

Scatterplots are common data visualisations utilised for communication
with experts and lay people alike. Despite being widely studied, it is
common for people to underestimate the level of correlation displayed in
them. The weight of evidence points toward changes in the opacities of
scatterplot points being unable to change perceptions of correlation,
however this was not tested rigorously using systematic adjustments.
Drawing on evidence that the shape of a scatterplot's point cloud may
drive correlation perception, I conducted exploratory work addressing
this underestimation bias. In two experiments (total \emph{N} = 300),
evidence is provided that changing the opacities of scatterplot points
\emph{can} have small effects on participants' performance on a
correlation estimation task. The systematic adjustment of point opacity
as a function of residual distance is able to alter estimates to a
greater degree and correct for the underestimation bias. In this
chapter, I also present an early pilot study that was ultimately not
included in any published works.

\section{Preface: Learning From an Early Pilot Study}\label{pilot-study}

\section{Introduction}\label{introduction-adjusting-opacity}

\subsection{Overview}\label{overview-adjusting-opacity}

\section{Related Work}\label{related-work-adjusting-opacity}

\subsection{Transparency, Contrast, Opacity, and Formal
Definitions}\label{transparency-contrast-opacity-and-formal-definitions}

\begin{itemize}
\tightlist
\item
  include the ``formalising contrast'' part of the original papers
  general methods section here
\item
  also include justification for referring to ``opacity'' instead of
  contrast
\end{itemize}

\subsection{Effects of Point Opacity on Correlation
Estimation}\label{effects-of-point-opacity-on-correlation-estimation}

\section{General Methods}\label{shared-methods-adjusting-opacity}

The experiments described in this chapter share multiple aspects of
their procedures. Both experiments were built using PsychoPy
\cite{peirce_2019} and are hosted on pavlovia.org. Both use 1-factor,
4-level designs. Ethical approval for both experiments was granted by
the University of Manchester's Computer Science Departmental Panel (Ref:
2022-14660-24397). In each experiment, participants were shown the
respective Participant Information Sheet (henceforth PIS) and provided
consent through key presses in response to consent statements.
Participants were asked to provide their age and gender identity, after
which they completed the 5-item Subjective Graph Literacy test described
by Garcia-Retamero et al.~\cite{garcia_2016} and discussed in Section
\ref{graph-literacy-lit-review} of the literature review. Early piloting
with a graduate student in humanities suggested the potential for
participants to be unfamiliar with the visual nature of different values
of Pearson's \emph{r}. Participants were therefore shown examples of
\emph{r} = 0.2, 0.5, 0.8, and 0.95 (see
Figure~\ref{fig-training-slide-adjusting-opacity}); a discussion of the
effects of this training is provided in Section
\ref{training-adjusting-opacity}. Participants were given two practice
trials to familiarise themselves with the response slider.

\begin{figure}

\centering{

\includegraphics[width=\textwidth]{../supplied_graphics/example-plots.png}

}

\caption{\label{fig-training-slide-adjusting-opacity}Participants viewed
these plots for at least eight seconds before being allowed to continue
to the practice trials.}

\end{figure}%

\begin{figure}

\centering{

\includegraphics[width=\textwidth]{../supplied_graphics/visual_mask.png}

}

\caption{\label{fig-mask-adjusting-opacity}An example of a visual mask
displayed for 2.5 seconds before each experimental trial.}

\end{figure}%

Each trial was preceded by text that either told the participant:

\begin{itemize}
\tightlist
\item
  Please look at the following plot and use the slider to estimate the
  correlation (n = 180).
\item
  Please IGNORE the correlation displayed and set the slider to 1 (n =
  3) or 0 (n = 3).
\end{itemize}

The latter instructions were attention checks, and were formatted with
red text to increase their visibility. Each experimental trial was
preceded by a visual mask (see Figure~\ref{fig-mask-adjusting-opacity})
that was displayed for 2.5 seconds. Participants were instructed to make
their judgements as quickly and accurately as possible, but there was no
time limit per trial. Both experiments described here use a fully
repeated-measures, within-participants design. Participants saw all 180
experimental items, corresponding to \textasciitilde27,000 individual
judgements per experiment, in a fully randomised order.

Both experiments were conducted according to principles of open and
reproducible research. All data and analysis code for the origin paper
is available on GitHub \footnote{https://github.com/gjpstrain/contrast\_and\_scatterplots}.
Experiment 1 \footnote{https://gitlab.pavlovia.org/Strain/exp\_uniform\_adjustments}
and 2 \footnote{https://gitlab.pavlovia.org/Strain/exp\_spatially\_dependent}
are hosted on Pavlovia.org, while the Open Science Framework hosts
pre-registrations \footnote{Experiment 1 - https://osf.io/tuexh.
  Experiment 2 - https://osf.io/6f5ev}. It is important to note at this
point that experiment 2 was conducted prior to experiment 1; when the
original paper was written, the order of presentation of the experiments
was swapped to make the narrative more cohesive. I preserve this order
in the present chapter.

\section{Experiment 1: Uniform Opacity
Adjustments}\label{experiment-1-uniform-opacity-adjustments}

\subsection{Introduction}\label{introduction-adjusting-opacity-e1}

Owing to the robust effects of altering stimulus opacity on perception
described above \cite{wehrhahn_1990, champion_2017}, it was hypothesised
that there would be a greater spread of estimates of correlation for
plots with lower global opacity compared to higher opacity plots.

\subsection{Method}\label{methods-adjusting-opacity-e1}

\subsubsection{Participants}\label{participants}

150 participants were recruited using the Prolific platform
\cite{prolific}. Normal to corrected-to-normal vision and English
fluency were required. Participants who had completed the pre-study were
prevented from participating. Data were collected from 158 participants.
8 failed more than 2 out of 6 attention check questions, and, as per the
pre-registration, had their submissions rejected from the study. The
data from the remaining 150 participants were included in the full
analysis (50.67\% male, 47.33 \% female, and 1.33\% non-binary).
Participants' mean age was 28.29 (\emph{SD} = 8.59). Mean graph literacy
score was 21.79 (\emph{SD} = 4.47). The mean time taken to complete the
experiment was 33 minutes (SD = 10 minutes).

\subsubsection{Design}\label{design}

For each of the 45 \emph{r} values, there were four versions of each
plot corresponding to the four levels of point opacity. Examples of each
of these can be seen in Figure~\ref{fig-exp1-examples-chap4},
demonstrated with an \emph{r} value of 0.6.

\begin{figure}

\centering{

\includegraphics[width=\textwidth]{4_adjusting_opacity_files/figure-latex/fig-exp1-examples-chap4-1.pdf}

}

\caption{\label{fig-exp1-examples-chap4}Examples of the stimuli used in
experiment 1, demonstrated with an \textit{r} value of 0.6. Here,
``opacity'' refers to the alpha value used by ggplot.}

\end{figure}%

\subsection{Analysis}\label{analysis-adjusting-opacity-e1}

To investigate the effects of opacity condition on participants'
estimates of correlation, a linear mixed effects model was built whereby
opacity condition is a predictor for the difference between objective
\emph{r} values for each plot and participants' estimates of \emph{r}.
This model has random intercepts for items and participants. A
likelihood ratio test revealed that the mode including global opacity as
a fixed effect explained significantly more variance than a null model
(\(\chi^2\)(3) = 223.13, \emph{p} \textless{} .001).
\textbf{?@fig-e1-estimates} shows the mean errors in correlation
estimation for each opacity condition.

\subsection{Discussion}\label{discussion-adjusting-opacity-e1}

\section{Experiment 2: Spatially-Dependent Opacity
Adjustments}\label{experiment-2-spatially-dependent-opacity-adjustments}

\subsection{Introduction}\label{introduction-adjusting-opacity-e2}

\subsection{Methods}\label{methods-adjusting-opacity-e2}

150 participants were recruited using the Prolific platform
\cite{prolific}. Normal to corrected-to-normal vision and English
fluency were required. Participants who had completed the pre-study were
prevented from participating. Data were collected from 158 participants.
7 failed more than 2 out of 6 attention check questions, and, as per the
pre-registration, had their submissions rejected from the study. The
data from the remaining 150 participants were included in the full
analysis (51.33\% male, 46.00 \% female, and 2.67\% non-binary).
Participants' mean age was 27.05 (\emph{SD} = 7.37). Mean graph literacy
score was 21.71 (\emph{SD} = 4.06). The average time taken to complete
the experiment was 33 minutes (SD = 10 minutes).

\subsection{Analysis}\label{analysis-adjusting-opacity-e2}

\subsection{Discussion}\label{discussion-adjusting-opacity-e2}

\section{General Discussion}\label{general-discussion-adjusting-opacity}

\subsection{Training}\label{training-adjusting-opacity}




\end{document}
