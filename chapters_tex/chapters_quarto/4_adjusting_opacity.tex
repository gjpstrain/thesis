\documentclass[../main.tex]{subfiles}
\begin{document}


\section{Abstract}\label{abstract-adjusting-opacity}

Scatterplots are common data visualisations utilised for communication
with experts and lay people alike. Despite being widely studied, it is
common for people to underestimate the level of correlation displayed in
them. The weight of evidence points toward changes in the opacities of
scatterplot points being unable to change perceptions of correlation,
however this was not tested rigorously using systematic adjustments.
Drawing on evidence that the shape of a scatterplot's point cloud may
drive correlation perception, I conducted exploratory work addressing
this underestimation bias. In two experiments (total \emph{N} = 300),
evidence is provided that changing the opacities of scatterplot points
\emph{can} have small effects on participants' performance on a
correlation estimation task. The systematic adjustment of point opacity
as a function of residual distance is able to alter estimates to a
greater degree and correct for the underestimation bias. In this
chapter, I also present an early pilot study that was ultimately not
included in any published works.

\section{Preface: Learning From an Early Pilot Study}\label{pilot-study}

The research proposal that kickstarted this project in 2021 set out a
plan to investigate the perception of correlation in scatterplots as a
function of screen size. This was prompted by recent research
demonstrating consistent perceptual biases in scatterplots due to
geometric scaling \cite{wei_2020}, the increasing prevalence of data
visualisations in lay people's daily lives due to the COVID-19 pandemic,
and the increasing adoption of wearable devices. The first experiment
conducted therefore examined how perceptions of correlation changed
according to the size of a scatterplot. Additionally, a very early
version of the opacity decay factor from experiment 2 was included. The
implementation of this factor was immature. In experiment 2 onwards, if
a scatterplot point resided in a particular place, it would always have
the same opacity or size. In the pilot study, the code that set the
opacity of each point always scaled the opacity values such that the
point with the highest residual had the lowest possible opacity, and
vice versa, resulting in the plots seen in
Figure~\ref{fig-pilot-study-examples}.

\begin{figure}

\centering{

\includegraphics[width=\textwidth]{../supplied_graphics/pilot_examples.png}

}

\caption{\label{fig-pilot-study-examples}Examples of the experimental
stimuli used in the pilot study (opacity decay factor). On the left, the
opacity decay function is visible. Note the linear scaling used.}

\end{figure}%

This provided no consistency between different experimental stimuli,
making it difficult to comment on the effects of changing levels of
opacity in various parts of a plot on correlation estimation. This was
later addressed by hardcoding residual size to a specific value of
opacity or size. The pilot also suffered extensively from poor data
quality. Of the 260 participants tested, data from only 118 was included
in the final analyses due to failed attention checks. It is for this
reason that the pre-screen requirements detailed in Section
\ref{recruitment} were implemented.

Participants viewed 180 experimental plots in a 3x2 factorial design.
The first independent variable, plot size, had three levels, 63\%,
100\%, and 252\% scales. The second was the presence or absence of the
opacity decay function (see Figure~\ref{fig-pilot-study-examples}). I
aimed to recruit 150 participants, but stopped after 118 due to data
quality issues. Nevertheless, the results provided were crucial in
informing the future direction of the research project. I present these
results in brief below.

\subsection{Pilot Study: Results}\label{pilot-study-results}

To investigate the effects of plot size and the presence or absence of
an opacity decay manipulation on participants' estimates of correlation,
a linear mixed effects model was built whereby participants' errors in
correlation estimation was predicted by plot size and the presence of
the opacity decay function. This model features random intercepts for
participants and items, as well as random slopes for both participants
and items relevant to the presence or absence of the opacity decay
function. A likelihood ratio test between the experimental model and a
null model with the fixed effects removed revealed that the experimental
model explained significantly more variance than the null (\(\chi^2\)(3)
= 26.38, \emph{p} \textless{} .001). There was no interaction between
plot size and the presence or absence of the opacity decay function. The
\textbf{emmeans} \cite{emmeans} package was used to explore estimated
marginal means (see Table~\ref{tbl-emmeans-pilot}) and contrasts (see
Table~\ref{tbl-contrasts-pilot}) separately for each factor.

\begin{table}

\caption{\label{tbl-emmeans-pilot}Estimated Marginal Means for plot size
(left) and the presence of the opacity decay function (right).}

\begin{minipage}{0.50\linewidth}

\begin{tabular}[t]{lrr}
\toprule
Size & Mean & Standard
Error\\
\midrule
Large (252\%) & 0.12 & 0.014\\
Medium (100\%) & 0.12 & 0.014\\
Small (62\%) & 0.13 & 0.014\\
\bottomrule
\end{tabular}

\end{minipage}%
%
\begin{minipage}{0.50\linewidth}

\begin{tabular}[t]{lrr}
\toprule
Decay & Mean & Standard
Error\\
\midrule
Absent & 0.13 & 0.015\\
Present & 0.11 & 0.013\\
\bottomrule
\end{tabular}

\end{minipage}%

\end{table}%

\begin{table}

\caption{\label{tbl-contrasts-pilot}Contrasts between levels of the size
factor (left) and opacity decay factor (right).}

\begin{minipage}{0.50\linewidth}

\begin{tabular}[t]{llrl}
\toprule
\multicolumn{2}{c}{Contrasts} & \multicolumn{2}{c}{Statistics} \\
\cmidrule(l{3pt}r{3pt}){1-2} \cmidrule(l{3pt}r{3pt}){3-4}
  &    & Z ratio & \textit{p}\\
\midrule
Large (252\%) & Medium (100\%) & -0.94 & 0.618\\
Large (252\%) & Small (52\%) & -3.56 & 0.001\\
Medium (100\%) & Small (52\%) & -2.63 & 0.023\\
\bottomrule
\end{tabular}

\end{minipage}%
%
\begin{minipage}{0.50\linewidth}

\begin{tabular}[t]{llrl}
\toprule
\multicolumn{2}{c}{Contrasts} & \multicolumn{2}{c}{Statistics} \\
\cmidrule(l{3pt}r{3pt}){1-2} \cmidrule(l{3pt}r{3pt}){3-4}
  &    & Z ratio & \textit{p}\\
\midrule
Absent & Present & 3.65 & <0.001\\
\bottomrule
\end{tabular}

\end{minipage}%

\end{table}%

\subsection{Pilot Study: Discussion}\label{pilot-study-discussion}

For the factor of plot size, the effect observed was driven by
significant differences in correlation estimation error between large
and small plots. Participants estimated more accurately when the plot
was large and when the decay function was present. Participants still
underestimated correlation in all conditions. The finding that
estimation error was lower for larger plots is in line with previous
evidence that geometrically scaling a scatterplot up can increase
perceptions of the strength of the correlation displayed
\cite{wei_2020}. Despite the statistical significance of this finding,
we elected at this point to abandon the plot size factor due to the
extremely small effect (see Table~\ref{tbl-emmeans-pilot}) and lack of
novelty.

The impact of even an immature point opacity decay function on
correlation estimation was a novel finding that we felt deserved
further, and more rigorous, study. Its implementation was based on
findings that changing the opacities of scatterplot points could bias
estimates of means \cite{hong_2021}, and on limited evidence for the
perception of correlation being based on the perceived width of a
probability distribution represented by the arrangement of scatterplot
points. I did not foresee the decay function, being novel, having a
greater effect on correlation estimation than size. Once evidence had
been found that changing the opacity of points in scatterplots could
have effects on correlation estimation, in opposition to previous
research \cite{rensink_2012, rensink_2014}, the door was opened for a
rigorous investigation into how this worked and how it could be used
systematically to correct for the historic underestimation bias.

\section{Introduction}\label{introduction-adjusting-opacity}

Findings from the pilot study suggest that changing the opacities of
points in scatterplots is able to change participants' estimates of the
correlation being displayed. The effect found in that study was too
small to make a real difference with a view to correcting for the
underestimation bias, and does not provide information on \emph{how}
changing opacity might change the percept (only that \emph{it does}).
Failing to understand the ways in which opacity is able to change the
perception of correlation prevents future work from tuning what was a
small effect in the pilot study into something with real potential for
producing more perceptually optimised scatterplots.

Previous work had found that

\subsection{Overview}\label{overview-adjusting-opacity}

In two experiments, the opacities of point in scatterplots was
manipulated while participants were asked to make judgements of
correlation. In the first, point opacity is changed in a uniform manner,
while in the second, point opacity is systematically altered as a
function of the size of a particular point's residual. By comparing
participants' performance on a correlation estimation task for
data-identical scatterplots that vary only in the opacities of their
points, it is demonstrated that; lower global point opacity results in
greater errors in the estimation of positive correlation (experiment 1);
and lowering point opacity as a function of the size of a point's
residual is able to bias estimates of positive correlation upwards to
partially correct for a historic underestimation bias.

\section{Related Work}\label{related-work-adjusting-opacity}

\subsection{Transparency, Contrast, Opacity, and Formal
Definitions}\label{transparency-contrast-opacity-and-formal-definitions}

\begin{itemize}
\tightlist
\item
  include the ``formalising contrast'' part of the original papers
  general methods section here
\item
  also include justification for referring to ``opacity'' instead of
  contrast
\end{itemize}

\subsection{Effects of Point Opacity on Correlation
Estimation}\label{effects-of-point-opacity-on-correlation-estimation}

\section{General Methods}\label{shared-methods-adjusting-opacity}

The experiments described in this chapter share multiple aspects of
their procedures. Both experiments were built using PsychoPy
\cite{peirce_2019} and are hosted on pavlovia.org. Both use 1-factor,
4-level designs. Ethical approval for both experiments was granted by
the University of Manchester's Computer Science Departmental Panel (Ref:
2022-14660-24397). In each experiment, participants were shown the
respective Participant Information Sheet (henceforth PIS) and provided
consent through key presses in response to consent statements.
Participants were asked to provide their age and gender identity, after
which they completed the 5-item Subjective Graph Literacy test described
by Garcia-Retamero et al.~\cite{garcia_2016} and discussed in Section
\ref{graph-literacy-lit-review} of the literature review. Early piloting
with a graduate student in humanities suggested the potential for
participants to be unfamiliar with the visual nature of different values
of Pearson's \emph{r}. Participants were therefore shown examples of
\emph{r} = 0.2, 0.5, 0.8, and 0.95 (see
Figure~\ref{fig-training-slide-adjusting-opacity}); a discussion of the
effects of this training is provided in Section
\ref{training-adjusting-opacity}. Participants were given two practice
trials to familiarise themselves with the response slider.

\begin{figure}

\centering{

\includegraphics[width=\textwidth]{../supplied_graphics/example-plots.png}

}

\caption{\label{fig-training-slide-adjusting-opacity}Participants viewed
these plots for at least eight seconds before being allowed to continue
to the practice trials.}

\end{figure}%

\begin{figure}

\centering{

\includegraphics[width=\textwidth]{../supplied_graphics/visual_mask.png}

}

\caption{\label{fig-mask-adjusting-opacity}An example of a visual mask
displayed for 2.5 seconds before each experimental trial.}

\end{figure}%

Each trial was preceded by text that either told the participant:

\begin{itemize}
\tightlist
\item
  Please look at the following plot and use the slider to estimate the
  correlation (n = 180).
\item
  Please IGNORE the correlation displayed and set the slider to 1 (n =
  3) or 0 (n = 3).
\end{itemize}

The latter instructions were attention checks, and were formatted with
red text to increase their visibility. Each experimental trial was
preceded by a visual mask (see Figure~\ref{fig-mask-adjusting-opacity})
that was displayed for 2.5 seconds. Participants were instructed to make
their judgements as quickly and accurately as possible, but there was no
time limit per trial. Both experiments described here use a fully
repeated-measures, within-participants design. Participants saw all 180
experimental items, corresponding to \textasciitilde27,000 individual
judgements per experiment, in a fully randomised order.

Both experiments were conducted according to principles of open and
reproducible research. All data and analysis code for the origin paper
is available on GitHub \footnote{https://github.com/gjpstrain/contrast\_and\_scatterplots}.
Experiment 1 \footnote{https://gitlab.pavlovia.org/Strain/exp\_uniform\_adjustments}
and 2 \footnote{https://gitlab.pavlovia.org/Strain/exp\_spatially\_dependent}
are hosted on Pavlovia.org, while the Open Science Framework hosts
pre-registrations \footnote{Experiment 1 - https://osf.io/tuexh.
  Experiment 2 - https://osf.io/6f5ev}. It is important to note at this
point that experiment 2 was conducted prior to experiment 1; when the
original paper was written, the order of presentation of the experiments
was swapped to make the narrative more cohesive. I preserve this order
in the present chapter.

\section{Experiment 1: Uniform Opacity
Adjustments}\label{experiment-1-uniform-opacity-adjustments}

\subsection{Introduction}\label{introduction-adjusting-opacity-e1}

Owing to the robust effects of altering stimulus opacity on perception
described above \cite{wehrhahn_1990, champion_2017}, it was hypothesised
that there would be a greater spread of estimates of correlation for
plots with lower global opacity compared to higher opacity plots.

\subsection{Method}\label{methods-adjusting-opacity-e1}

\subsubsection{Participants}\label{participants}

150 participants were recruited using the Prolific platform
\cite{prolific}. Normal to corrected-to-normal vision and English
fluency were required. Participants who had completed the pre-study were
prevented from participating. Data were collected from 158 participants.
8 failed more than 2 out of 6 attention check questions, and, as per the
pre-registration, had their submissions rejected from the study. The
data from the remaining 150 participants were included in the full
analysis (50.67\% male, 47.33 \% female, and 1.33\% non-binary).
Participants' mean age was 28.29 (\emph{SD} = 8.59). Mean graph literacy
score was 21.79 (\emph{SD} = 4.47). The mean time taken to complete the
experiment was 33 minutes (SD = 10 minutes).

\subsubsection{Design}\label{design}

For each of the 45 \emph{r} values, there were four versions of each
plot corresponding to the four levels of point opacity. Examples of each
of these can be seen in Figure~\ref{fig-exp1-examples-chap4},
demonstrated with an \emph{r} value of 0.6.

\begin{figure}

\centering{

\includegraphics[width=\textwidth]{4_adjusting_opacity_files/figure-latex/fig-exp1-examples-chap4-1.pdf}

}

\caption{\label{fig-exp1-examples-chap4}Examples of the stimuli used in
experiment 1, demonstrated with an \textit{r} value of 0.6. Here,
``opacity'' refers to the alpha value used by ggplot.}

\end{figure}%

\subsection{Analysis}\label{analysis-adjusting-opacity-e1}

To investigate the effects of opacity condition on participants'
estimates of correlation, a linear mixed effects model was built whereby
opacity condition is a predictor for the difference between objective
\emph{r} values for each plot and participants' estimates of \emph{r}.
This model has random intercepts for items and participants. A
likelihood ratio test revealed that the mode including global opacity as
a fixed effect explained significantly more variance than a null model
(\(\chi^2\)(3) = 223.13, \emph{p} \textless{} .001).
Figure~\ref{fig-e1-estimates} shows the mean errors in correlation
estimation for each opacity condition, along with 95\% confidence
intervals.

\begin{figure}

\centering{

\includegraphics[width=\textwidth]{4_adjusting_opacity_files/figure-latex/fig-e1-estimates-1.pdf}

}

\caption{\label{fig-e1-estimates}Estimated marginal means for the four
conditions tested in experiment 1. 95\% confidence intervals are shown.
The vertical dashed line represents no estimation error. The
overestimation zone is included to facilitate comparison to later work.}

\end{figure}%

\begin{table}

\caption{\label{tbl-contrasts-e1}Contrasts between different levels of
the opacity factor in experiment 1.}

\centering{

\begin{tabular}[t]{llrl}
\toprule
\multicolumn{2}{c}{Contrasts} & \multicolumn{2}{c}{Statistics} \\
\cmidrule(l{3pt}r{3pt}){1-2} \cmidrule(l{3pt}r{3pt}){3-4}
  &    & Z ratio & \textit{p}\\
\midrule
Full Opacity
(alpha = 1.0) & High Opacity
(alpha = 0.75) & -1.363 & 0.523\\
Full Opacity
(alpha = 1.0) & Medium Opacity
(alpha = 0.5) & -6.809 & <0.001\\
Full Opacity
(alpha = 1.0) & Low Opacity
(alpha = 0.25) & -13.439 & <0.001\\
High Opacity
(alpha = 0.75) & Medium Opacity
(alpha = 0.5) & -5.443 & <0.001\\
High Opacity
(alpha = 0.75) & Low Opacity
(alpha = 0.25) & -12.071 & <0.001\\
\addlinespace
Medium Opacity
(alpha = 0.5) & Low Opacity
(alpha = 0.25) & -6.631 & <0.001\\
\bottomrule
\end{tabular}

}

\end{table}%

This effect was driven by significant difference between means of
correlation estimation error between all conditions bar high and full
opacity. Statistical tests for contrasts were performed using the
\textbf{emmeans} package \cite{emmeans}, and are shown in
Table~\ref{tbl-contrasts-e1}. To test whether the observed results could
be explained by difference in participants' levels of graph literacy, an
additional model was built. This model is identical to the experimental
model, but also includes graph literacy as a fixed effect. Including
graph literacy as a fixed effect explained no additional variance
(\(\chi^2\)(1) = .002, \emph{p} = .962), indicating that the differences
observed in participants' correlation estimation performance were not as
a result of differences in levels of graph literacy.

\begin{table}

\caption{\label{tbl-efs-e1}Cohen's \textit{d} effect sizes for levels of
the opacity factor in experiment 1. Each effect size is compared to the
reference level, full contrast (alpha = 1).}

\centering{

\begin{tabular}[t]{lr}
\toprule
Effect & Cohen's \textit{d}\\
\midrule
Full Opacity (alpha = 1.0) & \\
High Opacity (alpha = 0.75) & 0.02\\
Medium Opacity (alpha = 0.5) & 0.08\\
Low Opacity (alpha = 0.25) & 0.16\\
\bottomrule
\end{tabular}

}

\end{table}%

A function from the now archived \textbf{EMAtools} package
\cite{ematools} was used to calculate an approximation of Cohen's
\emph{d} between the reference level (full contrast, alpha = 1.0) and
each other level of the opacity factor. These statistics can be seen in
Table~\ref{tbl-efs-e1}. The largest effect size observed (\emph{d}
\textasciitilde{} 0.16) is between the low and full opacity conditions,
and is small. This was unsurprising given the lack of previously
reported effects on correlation perception of global point opacity
\cite{rensink_2012}.

\subsection{Discussion}\label{discussion-adjusting-opacity-e1}

\section{Experiment 2: Spatially-Dependent Opacity
Adjustments}\label{experiment-2-spatially-dependent-opacity-adjustments}

\subsection{Introduction}\label{introduction-adjusting-opacity-e2}

\subsection{Methods}\label{methods-adjusting-opacity-e2}

\subsubsection{Participants}\label{participants-adjusting-opacity-e2}

150 participants were recruited using the Prolific platform
\cite{prolific}. Normal to corrected-to-normal vision and English
fluency were required. Participants who had completed the pre-study were
prevented from participating. Data were collected from 158 participants.
7 failed more than 2 out of 6 attention check questions, and, as per the
pre-registration, had their submissions rejected from the study. The
data from the remaining 150 participants were included in the full
analysis (51.33\% male, 46.00 \% female, and 2.67\% non-binary).
Participants' mean age was 27.05 (\emph{SD} = 7.37). Mean graph literacy
score was 21.71 (\emph{SD} = 4.06). The average time taken to complete
the experiment was 33 minutes (SD = 10 minutes).

\subsubsection{Design}\label{design-adjusting-opacity-e2}

For each of the 45 \emph{r} values in experiment 2, there were four
versions of each plot corresponding to the three levels of point opacity
decay function and the baseline global full opacity condition. Examples
of each of these can be seen in Figure~\ref{fig-exp2-examples-chap4},
demonstrated with an \emph{r} value of 0.6.

\begin{figure}

\centering{

\includegraphics[width=\textwidth]{4_adjusting_opacity_files/figure-latex/fig-exp2-examples-chap4-1.pdf}

}

\caption{\label{fig-exp2-examples-chap4}Examples of the stimuli used in
experiment 2, demonstrated with an \textit{r} value of 0.6. Here,
``opacity'' refers to the alpha value used by ggplot.}

\end{figure}%

\subsection{Analysis}\label{analysis-adjusting-opacity-e2}

To investigate the effects of the opacity decay functions on
participants' estimates of correlation, a linear mixed effects model was
built whereby decay function condition is a predictor for the difference
between objective \emph{r} values for each plot and participants'
estimates of \emph{r}. This model has random intercepts for items and
participants. A likelihood ratio test revealed that the mode including
global opacity as a fixed effect explained significantly more variance
than a null model (\(\chi^2\)(3) = 1,157.62, \emph{p} \textless{} .001).
Figure~\ref{fig-e1-estimates} shows the mean errors in correlation
estimation for each opacity decay function condition, along with 95\%
confidence intervals.

\begin{figure}

\centering{

\includegraphics[width=\textwidth]{4_adjusting_opacity_files/figure-latex/fig-e2-estimates-1.pdf}

}

\caption{\label{fig-e2-estimates}Estimated marginal means for the four
conditions tested in experiment 2. 95\% confidence intervals are shown.
The vertical dashed line represents no estimation error.}

\end{figure}%

\begin{table}

\caption{\label{tbl-contrasts-e2}Contrasts between different levels of
opacity decay function in experiment 2.}

\centering{

\begin{tabular}[t]{llrl}
\toprule
\multicolumn{2}{c}{Contrasts} & \multicolumn{2}{c}{Statistics} \\
\cmidrule(l{3pt}r{3pt}){1-2} \cmidrule(l{3pt}r{3pt}){3-4}
  &    & Z ratio & \textit{p}\\
\midrule
Full Opacity & Inverted Decay & -18.9 & <0.001\\
Full Opacity & Linear Decay & 1.6 & 0.405\\
Full Opacity & Non-Linear Decay & 15.3 & <0.001\\
Inverted Decay & Linear Decay & 20.4 & <0.001\\
Inverted Decay & Non-Linear Decay & 34.2 & <0.001\\
\addlinespace
Linear Decay & Non-Linear Decay & 13.7 & <0.001\\
\bottomrule
\end{tabular}

}

\end{table}%

The effect seen in experiment 2 was driven by significant differences in
means of correlation estimation error between all levels of opacity
decay function condition bar full opacity and linear decay. Statistical
testing for contrasts was performed using the \textbf{emmeans} package
\cite{emmeans}, and are shown in Table~\ref{tbl-contrasts-e2}. To test
whether the observed results could be explained by differences in graph
literacy, a model including participants' graph literacy scores as a
fixed effect was built. Including graph literacy as a fixed effect again
explained no additional variance (\(\chi^2\)(1) = .242, \emph{p} =
.623).

\begin{table}

\caption{\label{tbl-efs-e2}Cohen's \textit{d} effect sizes for levels of
the opacity decay function factor in experiment 2. Each effect size is
compared to the reference level, full contrast (alpha = 1).}

\centering{

\begin{tabular}[t]{lr}
\toprule
Effect & Cohen's \textit{d}\\
\midrule
Full Opacity & \\
Inverted Decay & 0.23\\
Linear Decay & -0.02\\
Non-Linear Decay & -0.19\\
\bottomrule
\end{tabular}

}

\end{table}%

Approximated Cohen's \emph{d} effect sizes between the baseline (global
full opacity) and each other condition can be seen in
Table~\ref{tbl-efs-e2}. The largest effect size observed
(\textasciitilde0.23 between full opacity and inverted decay conditions)
is small to moderate, and the effect size between the baseline and the
non-linear decay condition (\textasciitilde0.19) is small.

\subsection{Discussion}\label{discussion-adjusting-opacity-e2}

\section{General Discussion}\label{general-discussion-adjusting-opacity}

\subsection{Training}\label{training-adjusting-opacity}




\end{document}
