\documentclass[../main.tex]{subfiles}
\begin{document}


\section{Abstract}\label{abstract-adjusting-opacity}

Scatterplots are common data visualisations utilised for communication
with experts and lay people alike. Despite being widely studied, people
tend to underestimate the level of correlation displayed in them. The
weight of evidence points toward changes in the opacities of scatterplot
points being unable to change perceptions of correlation, however this
was not tested rigorously using systematic adjustments. Drawing on
evidence that the shape of a scatterplot's point cloud may drive
correlation perception, I conducted exploratory work addressing this
underestimation bias. In two experiments (total \emph{N} = 300),
evidence is provided that changing the opacities of scatterplot points
\emph{can} have small effects on participants' performance on a
correlation estimation task. The systematic adjustment of point opacity
as a function of residual distance is able to alter estimates
sufficiently to correct for the underestimation bias. In this chapter, I
also present an early pilot study that was ultimately not included in
any published works.

\section{Preface: Learning From an Early Pilot Study}\label{pilot-study}

The research proposal that kickstarted this project in 2021 set out a
plan to investigate the perception of correlation in scatterplots as a
function of screen size. This proposal is included in the supplemental
materials. This was prompted by recent research demonstrating consistent
perceptual biases in scatterplots due to geometric scaling
\cite{wei_2020}, the growing prevalence of data visualisations in lay
people's daily lives due to the COVID-19 pandemic, and the increasing
adoption of wearable devices \cite{shandhi_2024}. The first experiment
conducted therefore examined how perceptions of correlation changed
according to the size of a scatterplot. Additionally, a very early
version of the opacity decay condition from Experiment 2 was included,
however the implementation of this condition was immature. In Experiment
2 onwards, if a scatterplot point resided in a particular place on a
scatterplot, it would always have the same opacity or size. In the pilot
study, the code that set the opacity of each point always scaled the
opacity values such that the point with the highest residual had the
lowest possible opacity, and vice versa, resulting in the plots seen in
Figure~\ref{fig-pilot-study-examples}.

\begin{figure}

\centering{

\includegraphics[width=\textwidth]{../supplied_graphics/pilot_examples.png}

}

\caption{\label{fig-pilot-study-examples}Examples of the experimental
stimuli used in the pilot study (opacity decay factor). On the left, the
opacity decay function is visible. Note the linear scaling used.}

\end{figure}%

This provided no consistency between different experimental stimuli,
making it difficult to comment on the effects of changing levels of
opacity in various parts of a plot on correlation estimation. This was
later addressed by hardcoding residual size to a specific value of
opacity or size. The pilot also suffered extensively from poor data
quality. Of the 260 participants tested, data from only 118 was included
in the final analyses due to failed attention checks. It is for this
reason that the pre-screen requirements detailed in Section
\ref{recruitment} were implemented.

Participants viewed 180 experimental plots in a 3x2 factorial design.
The first independent variable, plot size, had three levels, 63\%,
100\%, and 252\% scales. The second was the presence or absence of the
opacity decay function (see Figure~\ref{fig-pilot-study-examples}). I
aimed to recruit 150 participants, but stopped after 118 due to ongoing
data quality issues. Nevertheless, the results provided were crucial in
informing the future direction of the research project. I present these
results in brief below.

\subsection{Pilot Study: Results}\label{pilot-study-results}

To investigate the effects of plot size and the presence or absence of
an opacity decay manipulation on participants' estimates of correlation,
a linear mixed effects model was built whereby participants' errors in
correlation estimation were predicted by plot size and the presence or
absence of the opacity decay function. This model features random
intercepts for participants and items, as well as random slopes for both
participants and items relevant to the presence or absence of the
opacity decay function. A likelihood ratio test between the experimental
model and a null model with the fixed effects removed revealed that the
experimental model explained significantly more variance than the null
(\(\chi^2\)(3) = 26.38, \emph{p} \textless{} .001). There was no
interaction between plot size and the presence or absence of the opacity
decay function. The \texttt{emmeans} \cite{lenth_2024} package was used
to explore estimated marginal means (see Table~\ref{tbl-emmeans-pilot})
and contrasts (see Table~\ref{tbl-contrasts-pilot}) separately for each
condition.

\begin{table}

\caption{\label{tbl-emmeans-pilot}Estimated Marginal Means (EMMs) of
correlation estimation error for plot size (a) and the presence of the
opacity decay function (b).}

\begin{minipage}{0.50\linewidth}

\subcaption{\label{tbl-emmeans-pilot-1}EMMs for plot size factor.}

\centering{

\begin{tabular}[t]{lrr}
\toprule
Size & Mean & Standard
Error\\
\midrule
Large (252\%) & 0.12 & 0.014\\
Medium (100\%) & 0.12 & 0.014\\
Small (62\%) & 0.13 & 0.014\\
\bottomrule
\end{tabular}

}

\end{minipage}%
%
\begin{minipage}{0.50\linewidth}

\subcaption{\label{tbl-emmeans-pilot-2}EMMs for presence or absence of
opacity decay function.}

\centering{

\begin{tabular}[t]{lrr}
\toprule
Decay & Mean & Standard
Error\\
\midrule
Absent & 0.13 & 0.015\\
Present & 0.11 & 0.013\\
\bottomrule
\end{tabular}

}

\end{minipage}%

\end{table}%

\begin{table}

\caption{\label{tbl-contrasts-pilot}Contrasts between levels of the size
factor (a) and opacity decay factor (b).}

\begin{minipage}{0.50\linewidth}

\subcaption{\label{tbl-contrasts-pilot-1}Contrasts for size decay
factor.}

\centering{

\begin{tabular}[t]{llrl}
\toprule
\multicolumn{2}{c}{Contrast} & \multicolumn{2}{c}{Statistics} \\
\cmidrule(l{3pt}r{3pt}){1-2} \cmidrule(l{3pt}r{3pt}){3-4}
  &    & Z ratio & \textit{p}\\
\midrule
Large (252\%) & Medium (100\%) & -0.94 & 0.618\\
Large (252\%) & Small (52\%) & -3.56 & 0.001\\
Medium (100\%) & Small (52\%) & -2.63 & 0.023\\
\bottomrule
\end{tabular}

}

\end{minipage}%
%
\begin{minipage}{0.50\linewidth}

\subcaption{\label{tbl-contrasts-pilot-2}Contrasts for opacity decay
factor.}

\centering{

\begin{tabular}[t]{llrl}
\toprule
\multicolumn{2}{c}{Contrast} & \multicolumn{2}{c}{Statistics} \\
\cmidrule(l{3pt}r{3pt}){1-2} \cmidrule(l{3pt}r{3pt}){3-4}
  &    & Z ratio & \textit{p}\\
\midrule
Absent & Present & 3.65 & <0.001\\
\bottomrule
\end{tabular}

}

\end{minipage}%

\end{table}%

\subsection{Pilot Study: Discussion}\label{pilot-study-discussion}

For the factor of plot size, the effect observed was driven by
significant differences in correlation estimation error between large
and small plots and between medium and small plots. There were no
significant differences in correlation estimation performance between
large and medium plots. Participants estimated more accurately when the
plot was large and when the decay function was present. Participants
still underestimated correlation in all conditions. The finding that
estimation error was lower for larger plots is in line with previous
evidence that geometrically scaling a scatterplot up can increase
perceptions of the strength of the correlation displayed
\cite{wei_2020}. Despite the statistical significance of this finding,
we elected at this point to abandon the plot size factor due to the
extremely small effect (see Table~\ref{tbl-emmeans-pilot}) and lack of
novelty compared to the effects of the opacity decay function.

The impact of even an immature point opacity decay function on
correlation estimation was a novel finding that I felt deserved further,
and more rigorous, study. Its implementation was based on findings that
changing the opacities of scatterplot points could bias estimates of
means \cite{hong_2022}, and on limited evidence for the perception of
correlation being based on the perceived width of a probability
distribution represented by the arrangement of scatterplot points. I did
not foresee the decay function, being novel, having a greater effect on
correlation estimation than the established effect of plot size. Once
evidence had been found that changing the opacity of points in
scatterplots could have effects on correlation estimation, in opposition
to previous research \cite{rensink_2012, rensink_2014}, the door was
opened for a rigorous investigation into how this worked and how it
could be used systematically to correct for the historic underestimation
bias.

\section{Introduction}\label{introduction-adjusting-opacity}

Findings from the pilot study suggest that changing the opacities of
points in scatterplots is able to change participants' estimates of the
correlation being displayed. The effect found in that study was too
small to make a real difference with regards to correcting for the
underestimation bias, and does not provide information on \emph{how}
changing opacity might change the percept (only that \emph{it does}).
Failing to understand the ways in which opacity is able to change the
perception of correlation prevents future work from tuning what was a
small effect in the pilot study into something with real potential for
producing more perceptually optimised scatterplots.

\subsection{Overview}\label{overview-adjusting-opacity}

In two experiments, the opacities of points in scatterplots were
manipulated while participants were asked to make judgements of
correlation. In the first, point opacity is changed in a uniform manner,
while in the second, point opacity is systematically altered as a
function of the size of a particular point's residual. By comparing
participants' performance on a correlation estimation task for
data-identical scatterplots that vary only in the opacities of their
points, it is demonstrated that; lower global point opacity results in
greater errors in the estimation of positive correlation (Experiment 1);
and lowering point opacity as a function of the size of a point's
residual is able to bias estimates of positive correlation upwards to
partially correct for a historic underestimation bias (Experiment 2).

\section{Related Work}\label{related-work-adjusting-opacity}

\subsection{Transparency, Contrast, Opacity, and Formal
Definitions}\label{transparency-contrast-opacity-and-formal-definitions}

The original paper that this chapter is based on is titled ``The Effects
of Contrast on Correlation Perception in Scatterplots''. In response to
reviewer comments to the paper that forms
\chap{chap:interactions_opacity_size}, the term ``contrast'' was changed
to ``opacity''. In order to maintain consistency throughout this thesis,
the more up-to-date wording (opacity) is used, although I discuss the
issue of terminology below.

Adjusting the opacity of points in scatterplots is an established
technique used to address issues of overplotting or clutter
\cite{bertini_2004, matejka_2015}, in which scatterplots with large
numbers of data points suffer from visibility issues caused by excessive
point density. Lowering the opacity of all scatterplot points using
alpha blending \cite{few_2008} addresses this, and makes data trends and
distributions easier to see and interpret for the reader.
Figure~\ref{fig-overplotting-examples} demonstrates the impact of
lowering global point opacity in a scatterplot with a very high number
of data points.

\begin{figure}

\centering{

\includegraphics[width=\textwidth]{4_adjusting_opacity_files/figure-latex/fig-overplotting-examples-1.pdf}

}

\caption{\label{fig-overplotting-examples}Adjusting point opacity to
address overplotting. Contrast between the points and the background is
full (alpha = 1, full opacity points, left) or low (alpha = .1, low
opacity points, right). The dataset used has 20,000 points.}

\end{figure}%

Lowering opacity leads to a reduction of the contrast of isolated points
with the background, and for regions with overlapping points, colour
intensities are summed. The stimuli used in the experiments throughout
this chapter had 128 small points, meaning the majority of points were
clearly visible at all times. For this reason, the effects of point
overlap were not taken into account when designing and analysing the
experiments described here. Due to this, the approach to opacity
described in this chapter would not be useful when dealing with much
larger datasets where clutter becomes an issue.

The \texttt{ggplot2} \cite{wickham_2016} package (version 3.4.1) was
used in R to create stimuli for this experiment. This package uses an
alpha parameter to set point opacity. Alpha here refers to the level of
linear interpolation \cite{stone_2008} between foreground and background
pixel values; alpha values of 0 (full transparency) and 1 (full opacity)
result in no interpolation and rendering of either the background or
foreground pixel values respectively. Alpha values between 0 and 1
correspond to different ratios of interpolation, and are illustrated in
Figure~\ref{fig-alpha-examples}.

\begin{figure}

\centering{

\includegraphics[width=\textwidth]{4_adjusting_opacity_files/figure-latex/fig-alpha-examples-1.pdf}

}

\caption{\label{fig-alpha-examples}The relationship between alpha values
and rendered point opacity. Higher alpha values result in greater
contrast between the foreground (scatterplot point) and background. When
alpha = 0, the foreground is ignored and the background is rendered.}

\end{figure}%

Definitions of contrast, opacity, and transparency are fuzzy. Often,
different works will use the terms interchangeably. As mentioned above,
I initially elected to use the term ``contrast'', given the
fundamentality of contrast as a feature of human visual perception
\cite{ginsburg_2003}, however later reviewer comments prompted the
change to ``opacity''. Nevertheless, we can consider \emph{opacity} as
it is used here when pertaining to scatterplot points on a white
background to be shorthand for ``the contrast between foreground and
background objects'', as visually, these concepts are the same. There
are numerous psychophysical definitions of perceived contrast
\cite{zuffi_2007} based on what is being presented, for example, models
that take into account visibility limits (CIELAB lightness), or contrast
in periodic patterns such as sinusoidal gratings (Michelson's contrast).
The common thread running through these definitions is the use of a
ratio between target and background luminances. The experiments
described here take place online, with participants completing
experiments on their personal laptop or desktop computers. Due to this,
the experimenter has no control over the exact luminances of stimuli,
only over the relative luminance between targets (scatterplot points)
and backgrounds. Given my interest in relative differences in
correlation perception averaged over a series of 180 single-plot trials,
this lack of control over the exact nature of the stimuli was not
problematic. It does mean that reporting the exact luminance values
would be pointless however, so where a value for opacity is referred to
in this chapter and beyond, it is the alpha value specified by
\texttt{ggplot2}.

\subsection{Effects of Point Opacity on Correlation
Estimation}\label{point-opacity-chap4}

Despite the popularity of adjusting opacity to address overplotting
issues, little investigation had taken place into the effects of
reducing point opacity on people's perceptions of correlation. In 2012,
Rensink \cite{rensink_2012} found correlation perception to be invariant
to changes in point opacity, although this work took place with a small
sample (\emph{N} = 12), and using bisection/JND methodologies (see
Section \ref{scatterplots-corr-viz} in \chap{chap:related_work}).

Changing the contrast between a stimulus and its background (lowering
its opacity) effectively reduces the strength of its signal. A likely
consequence of this is greater levels of uncertainty in aspects of that
stimulus, for example, the locations of points in scatterplots.
Consequently, one might anticipate that increased uncertainty could lead
to altered perceptions of correlation and/or the presence of greater
levels of noise in correlation estimates due to effects on the perceived
position of points within a scatterplot point cloud. While there is
evidence \cite{wehrhahn_1990} that perception of stimulus position
becomes exponentially worse as contrast is reduced (as measured by
vernier acuity tasks), this is only true for a narrow range of low
contrast stimuli just above the detection threshold. For stimuli that
feature higher contrasts between them and their backgrounds, vernier
acuity appears robust to such changes. Nevertheless, there is evidence
that other perceptual estimates become more uncertain with reduced
contrast, such as speed perception \cite{champion_2017}. With this in
mind, I argue that the effects of stimulus opacity on perceived
correlation in scatterplots warrants further investigation.

In 2022, Hong et al.~\cite{hong_2022} used point opacity and size to
encode a third variable in trivariate scatterplots while asking
participants to judge the average position of all the points displayed.
It was found that participants' estimates of average point position were
biased towards areas of larger or darker points; this was termed the
\emph{weighted average illusion}. Together with evidence that darker
(more opaque) and larger points are more salient \cite{healey_2012},
this work suggested that there was the scope to use point opacity to
systematically lower the salience of the points representing the widest
parts of the probability distribution; if participants perceived a
narrower distribution, one might expect this to be able to (at least
partially) correct for the underestimation bias.

One way to correct for an underestimation of correlation in scatterplots
would be to simply remove outer data points until correlation perception
is aligned with the actual correlation value. However, this would
necessitate hiding data and thus changing the information presented to
the viewer. An alternative approach is to manipulate the opacity of only
some of the points; it would seem most sensible to do so for the points
that are more extreme relative to the underlying regression line. In the
present study these questions are explored in two online experiments
with large sample sizes. In the first, the effects of point opacity over
the entire scatterplot on correlation estimates is investigated. The
second experiment examines how changing contrast as a function of
distance to the regression line affects perceived correlation. To
pre-empt the results, clear effects of both manipulations are found.

\section{General Methods}\label{general-methods-adjusting-opacity}

The experiments described in this chapter share multiple aspects of
their procedures. Both experiments were built using PsychoPy
\cite{peirce_2019} and are hosted on Pavlovia.org. Both use 1-factor,
4-level designs. Ethical approval for both experiments was granted by
the University of Manchester's Computer Science Departmental Panel (Ref:
2022-14660-24397). In each experiment, participants were shown the
respective Participant Information Sheet (henceforth PIS) and provided
consent through key presses in response to consent statements.
Participants were asked to provide their age and gender identity, after
which they completed the 5-item Subjective Graph Literacy test described
by Garcia-Retamero et al.~\cite{garcia_2016} and discussed in Section
\ref{graph-literacy-related-work} of \chap{chap:related_work}. Early
piloting with a graduate student in humanities suggested the potential
for participants to be unfamiliar with the visual nature of different
values of Pearson's \emph{r}. Participants were therefore shown examples
of \emph{r} = 0.2, 0.5, 0.8, and 0.95 (see
Figure~\ref{fig-training-slide-adjusting-opacity}); a discussion of the
effects of this training is provided in Section
\ref{training-adjusting-opacity}. Participants were given two practice
trials to familiarise themselves with the response slider.

\begin{figure}

\centering{

\includegraphics[width=\textwidth]{../supplied_graphics/example-plots.png}

}

\caption{\label{fig-training-slide-adjusting-opacity}Participants viewed
these plots for at least eight seconds before being allowed to continue
to the practice trials.}

\end{figure}%

\begin{figure}

\centering{

\includegraphics[width=\textwidth]{../supplied_graphics/visual_mask.png}

}

\caption{\label{fig-mask-adjusting-opacity}An example of a visual mask
displayed for 2.5 seconds before each experimental trial.}

\end{figure}%

Each experimental trial was preceded by text that either told the
participant:

\begin{itemize}
\tightlist
\item
  Please look at the following plot and use the slider to estimate the
  correlation (\emph{n} = 180).
\item
  Please IGNORE the correlation displayed and set the slider to 1
  (\emph{n} = 3) or 0 (\emph{n} = 3).
\end{itemize}

The latter instructions were attention checks, and were formatted with
red text to increase their visibility. Each experimental trial was
preceded by a visual mask (see Figure~\ref{fig-mask-adjusting-opacity})
that was displayed for 2.5 seconds. Participants were instructed to make
their judgements as quickly and accurately as possible, but there was no
time limit per trial. Both experiments described here use a fully
repeated-measures, within-participants design. All 150 participants saw
all 180 experimental items, corresponding to \textasciitilde27,000
individual judgements per experiment, in a fully randomised order.

\subsection{Open Research}\label{open-research-chap4}

Both experiments were conducted according to principles of open and
reproducible research \cite{ayris_2018}. All data and analysis code for
the original paper are available on GitHub \footnote{https://github.com/gjpstrain/contrast\_and\_scatterplots}.
This repository also includes a Docker implementation to reproduce the
original computational environment the paper was written in. Experiment
1 \footnote{https://gitlab.pavlovia.org/Strain/exp\_uniform\_adjustments}
and 2 \footnote{https://gitlab.pavlovia.org/Strain/exp\_spatially\_dependent}
are hosted on Pavlovia.org, while the Open Science Framework hosts
pre-registrations \footnote{Experiment 1 - https://osf.io/tuexh.
  Experiment 2 - https://osf.io/6f5ev}. It is important to note at this
point that Experiment 2 was conducted prior to Experiment 1; when the
original paper was written, the order of presentation of the experiments
was swapped to make the narrative more cohesive. I preserve this order
in the present chapter.

\section{Experiment 1: Uniform Opacity
Adjustments}\label{experiment-1-uniform-opacity-adjustments}

\subsection{Introduction}\label{introduction-e1}

Previous literature had described correlation perception as being
resistant to changes in opacity \cite{rensink_2012, rensink_2014}.
Findings from the pre-study described above provided evidence against
this conclusion, so before proceeding with the fine-tuning of the
immature point opacity decay function, I felt that gaining an
understanding of how point opacity and correlation estimation interact
more generally was important. Owing to the robust effects of altering
stimulus opacity on perception described above
\cite{wehrhahn_1990, champion_2017}, it was hypothesised that:

\begin{itemize}
\tightlist
\item
  H1: A greater spread of estimates of correlation for plots with lower
  global opacity compared to higher opacity plots will be observed.
\end{itemize}

\subsection{Method}\label{method-e1}

\subsubsection{Participants}\label{participants-e1}

150 participants were recruited using the Prolific platform
\cite{prolific}. Normal or corrected-to-normal vision and English
fluency were required. Participants who had completed the pilot study
were prevented from participating. Data were collected from 158
participants. 8 failed more than 2 out of 6 attention check questions,
and, as per the pre-registration, had their submissions rejected from
the study. The data from the remaining 150 participants were included in
the full analysis (76 male, 71 female, and 2 non-binary). Participants'
mean age was 28.29 (\emph{SD} = 8.59). Mean graph literacy score was
21.79 (\emph{SD} = 4.47). The mean time taken to complete the experiment
was 33 minutes (SD = 10 minutes).

\subsubsection{Design}\label{design-e1}

For each of the 45 \emph{r} values, there were four versions of each
plot corresponding to the four levels of point opacity. Examples of each
of these can be seen in Figure~\ref{fig-exp1-examples-chap4},
demonstrated with an \emph{r} value of 0.6.

\begin{figure}

\centering{

\includegraphics[width=\textwidth]{4_adjusting_opacity_files/figure-latex/fig-exp1-examples-chap4-1.pdf}

}

\caption{\label{fig-exp1-examples-chap4}Examples of the stimuli used in
Experiment 1, demonstrated with an \textit{r} value of 0.6. Here,
``opacity'' refers to the alpha value used by \texttt{ggplot2}.}

\end{figure}%

\subsection{Results}\label{results-e1}

To investigate the effects of opacity condition on participants'
estimates of correlation, a linear mixed effects model was built whereby
opacity condition is a predictor for the difference between objective
\emph{r} values for each plot and participants' estimates of \emph{r}.
This model has random intercepts for items and participants. A
likelihood ratio test revealed that the model including global opacity
as a fixed effect explained significantly more variance than a null
model (\(\chi^2\)(3) = 223.13, \emph{p} \textless{} .001).
Figure~\ref{fig-e1-estimates} shows the mean errors in correlation
estimation for each opacity condition, along with 95\% confidence
intervals.

\begin{figure}

\centering{

\includegraphics[width=\textwidth]{4_adjusting_opacity_files/figure-latex/fig-e1-estimates-1.pdf}

}

\caption{\label{fig-e1-estimates}Estimated marginal means for the four
conditions tested in Experiment 1. 95\% confidence intervals are shown.
The vertical dashed line represents no estimation error. The
overestimation zone is included to facilitate comparison to later work.}

\end{figure}%

\begin{table}

\caption{\label{tbl-contrasts-e1}Contrasts between different levels of
the opacity factor in Experiment 1.}

\centering{

\begin{tabular}[t]{llrl}
\toprule
\multicolumn{2}{c}{Contrast} & \multicolumn{2}{c}{Statistics} \\
\cmidrule(l{3pt}r{3pt}){1-2} \cmidrule(l{3pt}r{3pt}){3-4}
  &    & Z ratio & \textit{p}\\
\midrule
Full Opacity
(alpha = 1.0) & High Opacity
(alpha = 0.75) & -1.363 & 0.523\\
Full Opacity
(alpha = 1.0) & Medium Opacity
(alpha = 0.5) & -6.809 & <0.001\\
Full Opacity
(alpha = 1.0) & Low Opacity
(alpha = 0.25) & -13.439 & <0.001\\
High Opacity
(alpha = 0.75) & Medium Opacity
(alpha = 0.5) & -5.443 & <0.001\\
High Opacity
(alpha = 0.75) & Low Opacity
(alpha = 0.25) & -12.071 & <0.001\\
\addlinespace
Medium Opacity
(alpha = 0.5) & Low Opacity
(alpha = 0.25) & -6.631 & <0.001\\
\bottomrule
\end{tabular}

}

\end{table}%

This effect was driven by significant differences between means of
correlation estimation error between all conditions bar high and full
opacity. Statistical tests for contrasts were performed using the
\texttt{emmeans} package \cite{lenth_2024}, and are shown in
Table~\ref{tbl-contrasts-e1}. To test whether the observed results could
be explained by difference in participants' levels of graph literacy, an
additional model was built. This model is identical to the experimental
model, but also includes graph literacy as a fixed effect. Including
graph literacy as a fixed effect explained no additional variance
(\(\chi^2\)(1) = .002, \emph{p} = .962), indicating that the differences
observed in participants' correlation estimation performance were not as
a result of differences in levels of graph literacy.

\begin{table}

\caption{\label{tbl-efs-e1}Cohen's \textit{d} effect sizes (a) and
summary statistics (b) for levels of the opacity factor in Experiment 1.
Each effect size is compared to the reference level, full contrast
(alpha = 1).}

\begin{minipage}{0.50\linewidth}

\subcaption{\label{tbl-efs-e1-1}Effect sizes for opacity factor.}

\centering{

\begin{tabular}[t]{lr}
\toprule
Effect & Cohen's \textit{d}\\
\midrule
Full Opacity (alpha = 1.0) & \\
High Opacity (alpha = 0.75) & 0.02\\
Medium Opacity (alpha = 0.5) & 0.08\\
Low Opacity (alpha = 0.25) & 0.16\\
\bottomrule
\end{tabular}

}

\end{minipage}%
%
\begin{minipage}{0.50\linewidth}

\subcaption{\label{tbl-efs-e1-2}Summary statistics.}

\centering{

\begin{tabular}[t]{lrr}
\toprule
Opacity & Mean & Standard
Error\\
\midrule
Full Opacity
(alpha = 1.0) & 0.15 & 0.016\\
High Opacity
(alpha = 0.75) & 0.15 & 0.016\\
Medium Opacity
(alpha = 0.5) & 0.17 & 0.016\\
Low Opacity (alpha = 0.25) & 0.18 & 0.016\\
\bottomrule
\end{tabular}

}

\end{minipage}%

\end{table}%

\begin{figure}

\centering{

\includegraphics[width=\textwidth]{4_adjusting_opacity_files/figure-latex/fig-estimates-by-r-e1-1.pdf}

}

\caption{\label{fig-estimates-by-r-e1}Participants' mean errors in
correlation estimates grouped by condition and by \textit{r} value. The
dashed horizontal line represents perfect estimation. Participants were
most accurate when presented with the plots featuring higher global
point opacity. Error bars show standard deviations of estimates.}

\end{figure}%

A function from the now archived \texttt{EMAtools} package
\cite{ematools} was used to calculate an approximation of Cohen's
\emph{d} between the reference level (full contrast, alpha = 1.0) and
each other level of the opacity factor. These statistics can be seen in
Table~\ref{tbl-efs-e1} (a) along with means and standard deviations (b).
The largest effect size observed (\emph{d} \textasciitilde{} 0.16) is
between the low and full opacity conditions, and is small. This was
unsurprising given the lack of previously reported effects on
correlation perception of global point opacity \cite{rensink_2012}.
Figure~\ref{fig-estimates-by-r-e1} shows how participants' estimates of
correlation change with the objective \emph{r} value in the plot. The
line represent mean errors in correlation estimation, and standard
deviations of error are provided as error bars. The dashed horizontal
line represents hypothetical perfect estimation. As reported in previous
literature (see Section \ref{underestimation-related-work} in
\chap{chap:related_work}), participants underestimated \emph{r} in
nearly all cases.

\subsection{Discussion}\label{discussion-e1}

The hypothesis, that there would be a greater spread of correlation
estimates for plots with lower global opacity compared to those with
higher global opacity, was not supported. As seen in
Table~\ref{tbl-efs-e1} (b), standard errors for each opacity condition
are identical to 3 decimal places. Participants' errors in correlation
estimation were significantly greater when the opacity of all
scatterplot points was lower compared when it was higher. This held true
up until alpha was set to 0.75, implying a threshold around this value
past which there is little variation in the perception of opacity, at
least as far as it is associated with correlation estimation. This lack
of significant difference in correlation estimation between the two
highest global opacity conditions is congruent with the logarithmic
nature of contrast/brightness perception
\cite{fechner_1948, varshney_2013}; despite there being equal linear
distance between the opacity values used, the perceptual distance
between them is minimal.

As mentioned previously, Rensink \cite{rensink_2012, rensink_2014}
presents the only other account of experiments that directly test
correlation perception as it pertains to the opacities of scatterplot
points, and report no difference in either bias (error) or variability
(spread) in correlation perception regarding point opacity
manipulations. In comparison, the results observed here do report an
effect. This effect may be explained by differences in experimental
power, as it is a small effect, although I argue that methodological
differences may have also played a small role. Given the small effect
size (Cohen's \emph{d} = 0.16), the small sample in Rensink (2014)
\cite{rensink_2014} may have caused the experiment to be insufficiently
powered. With the large sample size in the present work (\emph{N} =
150), evidence for an effect has been found. The experimental
methodology utilised here is more representative of the use of
scatterplots in the wild, and is therefore more suited to informing
design as opposed to investigating the mathematical relationship between
real and perceived correlation. While the effect is small, it
demonstrates definitively that differences in the opacities of
scatterplot points \emph{can} affect estimates of correlation in
positively correlated scatterplots. From the results it is unclear why
lowering global opacity causes greater errors in correlation estimation
while causing no difference in spread.

I suggest that correlation perception functions similarly to speed
perception \cite{champion_2017} with regards to changes in the contrast
between foreground targets (in this case scatterplot points) and the
background; the greater spatial uncertainty brought on by reduced point
opacity, while not eliciting greater spread in correlation estimates,
might be responsible for the effects observed via an increase in the
perceived width of the probability distribution displayed by the
scatterplot

From the results it is clear that a scatterplot optimised for
correlation perception should have contrast between the foreground
(points) and background in a range corresponding to alpha values of
between 0.75 and 1. That there are significant differences in
correlation estimation between data-identical scatterplots with
different global point opacities however, suggests that this effect may
be leveraged to further improve participants' performances on a
correlation estimation task.

\section{Experiment 2: Spatially-Dependent Opacity
Adjustments}\label{experiment-2-spatially-dependent-opacity-adjustments}

\subsection{Introduction}\label{introduction-e2}

Experiment 1 found that point opacity in positively correlated
scatterplots has an effect on the perception of correlation such that
those scatterplots with higher levels of global point opacity are rated
as being more strongly correlated. Given this finding, the question
arises of whether additional changes in correlation perception may be
observed as a function of the spatial arrangement of point opacity.
Given the previously reported effects of changing scatterplot point
opacity on other perceptual metrics (see Section
\ref{point-opacity-chap4}), and with the findings from Experiment 1 in
mind, it was hypothesised that:

\begin{itemize}
\tightlist
\item
  H1: the non-linear decay parameter in which point opacity falls with
  residual distance will result in lower mean errors in correlation
  estimation compared to linear decay and full global opacity
  conditions.
\item
  H2: the use of the inverted non-linear decay parameter, in which point
  opacity becomes greater with residual distance, will result in higher
  mean errors in correlation estimation than for all other conditions.
\end{itemize}

\subsection{Method}\label{method-e2}

\subsubsection{Participants}\label{participants-e2}

150 participants were recruited using the Prolific platform
\cite{prolific}. Normal to corrected-to-normal vision and English
fluency were required. Participants who had completed the pilot study or
Experiment 1 were prevented from participating. Data were collected from
158 participants. 8 failed more than 2 out of 6 attention check
questions, and, as per the pre-registration, had their submissions
rejected from the study. The data from the remaining 150 participants
were included in the full analysis (77 male, 69 female, and 4
non-binary). Participants' mean age was 27.05 (\emph{SD} = 7.37). Mean
graph literacy score was 21.71 (\emph{SD} = 4.06). The average time
taken to complete the experiment was 33 minutes (SD = 10 minutes).

\subsubsection{Design}\label{design-e2}

For each of the 45 \emph{r} values in Experiment 2, there were four
versions of each plot corresponding to the three levels of point opacity
decay function and the baseline global full opacity condition. Examples
of each of these can be seen in Figure~\ref{fig-exp2-examples-chap4},
demonstrated with an \emph{r} value of 0.6. Given the shape of the
underestimation curve found in previous work (see Figure
\ref{fig-underestimation-curves}, \chap{chap:related_work}), intuition
suggested employing a symmetrically opposing curve (see the non-linear
decay curve in Figure~\ref{fig-opposing-curve}) to relate point opacity
to residuals.

\begin{figure}

\centering{

\includegraphics[width=\textwidth]{4_adjusting_opacity_files/figure-latex/fig-opposing-curve-1.pdf}

}

\caption{\label{fig-opposing-curve}Using an \textit{r} value of 0.2 to
demonstrate the relationship between the size of a point's residual and
the alpha value (opacity) rendered.}

\end{figure}%

Equation 4.1 was used to non-linearly map residuals to \texttt{ggplot2}
alpha values. 0.25 was chosen as the value of \emph{b}, as it was felt
at the time that this rendered plots that maintained point visibility
while also allowing a large enough point opacity range that, if an
effect was present, it was likely to be found.

\begin{equation}
  point_{size/opacity} = 1 - b^{residual}
\end{equation}

\begin{figure}

\centering{

\includegraphics[width=\textwidth]{4_adjusting_opacity_files/figure-latex/fig-exp2-examples-chap4-1.pdf}

}

\caption{\label{fig-exp2-examples-chap4}Examples of the stimuli used in
Experiment 2, demonstrated with an \textit{r} value of 0.6. Here,
``opacity'' refers to the alpha value used by \texttt{ggplot2}.}

\end{figure}%

\subsection{Results}\label{results-e2}

To investigate the effects of the opacity decay functions on
participants' estimates of correlation, a linear mixed effects model was
built with decay function condition as a predictor for the difference
between objective \emph{r} values for each plot and participants'
estimates of \emph{r}. This model has random intercepts for items and
participants. A likelihood ratio test revealed that the model including
opacity decay function as a fixed effect explained significantly more
variance than a null model (\(\chi^2\)(3) = 1,157.62, \emph{p}
\textless{} .001). Figure~\ref{fig-e2-estimates} shows the mean errors
in correlation estimation for each opacity decay function condition,
along with 95\% confidence intervals.

\begin{figure}

\centering{

\includegraphics[width=\textwidth]{4_adjusting_opacity_files/figure-latex/fig-e2-estimates-1.pdf}

}

\caption{\label{fig-e2-estimates}Estimated marginal means for the four
conditions tested in Experiment 2. 95\% confidence intervals are shown.
The vertical dashed line represents no estimation error.}

\end{figure}%

\begin{table}

\caption{\label{tbl-contrasts-e2}Contrasts between different levels of
opacity decay function in Experiment 2.}

\centering{

\begin{tabular}[t]{llrl}
\toprule
\multicolumn{2}{c}{Contrast} & \multicolumn{2}{c}{Statistics} \\
\cmidrule(l{3pt}r{3pt}){1-2} \cmidrule(l{3pt}r{3pt}){3-4}
  &    & Z ratio & \textit{p}\\
\midrule
Full Opacity & Inverted Decay & -18.9 & <0.001\\
Full Opacity & Linear Decay & 1.6 & 0.405\\
Full Opacity & Non-Linear Decay & 15.3 & <0.001\\
Inverted Decay & Linear Decay & 20.4 & <0.001\\
Inverted Decay & Non-Linear Decay & 34.2 & <0.001\\
\addlinespace
Linear Decay & Non-Linear Decay & 13.7 & <0.001\\
\bottomrule
\end{tabular}

}

\end{table}%

The effect seen in Experiment 2 was driven by significant differences in
means of correlation estimation error between all levels of opacity
decay function condition bar full opacity and linear decay. Statistical
testing for contrasts was performed using the \texttt{emmeans} package
\cite{lenth_2024}, and are shown in Table~\ref{tbl-contrasts-e2}. To
test whether the observed results could be explained by differences in
graph literacy, a model including participants' graph literacy scores as
a fixed effect was built. Including graph literacy as a fixed effect
again explained no additional variance (\(\chi^2\)(1) = .242, \emph{p} =
.623).

\begin{table}

\caption{\label{tbl-efs-e2}Cohen's \textit{d} effect sizes (a) and
summary statistics (b) for the opacity decay function condition in
Experiment 2. Each effect size is compared to the reference level, full
contrast (alpha = 1).}

\begin{minipage}{0.50\linewidth}

\subcaption{\label{tbl-efs-e2-1}Effect sizes for opacity decay factor.}

\centering{

\begin{tabular}[t]{lr}
\toprule
Effect & Cohen's \textit{d}\\
\midrule
Full Opacity & \\
Inverted Decay & 0.23\\
Linear Decay & -0.02\\
Non-Linear Decay & -0.19\\
\bottomrule
\end{tabular}

}

\end{minipage}%
%
\begin{minipage}{0.50\linewidth}

\subcaption{\label{tbl-efs-e2-2}Summary statistics for opacity decay
factor.}

\centering{

\begin{tabular}[t]{lrr}
\toprule
Opacity & Mean & Standard
Error\\
\midrule
Full Opacity & 0.12 & 0.012\\
Inverted Decay & 0.17 & 0.012\\
Linear Decay & 0.12 & 0.012\\
Non-Linear Decay & 0.09 & 0.012\\
\bottomrule
\end{tabular}

}

\end{minipage}%

\end{table}%

Approximated Cohen's \emph{d} effect sizes between the baseline (global
full opacity) and each other condition can be seen in
Table~\ref{tbl-efs-e2}. The largest effect size observed
(\textasciitilde0.23 between full opacity and inverted decay conditions)
is small to moderate, and the effect size between the baseline and the
non-linear decay condition (\textasciitilde0.19) is small.
Figure~\ref{fig-estimates-by-r-e2} illustrates the effects of each
manipulation on participants' correlation estimation performance
separately for each value of \emph{r} used. The dashed horizontal line
represents perfect estimation, and standard deviations of estimation
error are provided by way of error bars. Participants still
underestimated correlation in all conditions, although the use of the
non-linear decay function biased participants' estimates upwards to
partially correct for the underestimation.

\begin{figure}

\centering{

\includegraphics[width=\textwidth]{4_adjusting_opacity_files/figure-latex/fig-estimates-by-r-e2-1.pdf}

}

\caption{\label{fig-estimates-by-r-e2}Participants' mean errors in
correlation estimates grouped by condition and by \textit{r} value. The
dashed horizontal line represents perfect estimation. Participants were
most accurate when presented with the plots featuring the non-linear
opacity decay function. Error bars show standard deviations of
estimates.}

\end{figure}%

\subsection{Discussion}\label{discussion-e2}

Both hypotheses received support in Experiment 2. Participants' errors
in correlation estimation were lowest when the non-linear decay function
was used, and were highest when scatterplots employed the non-linear
inverted decay function. There was no significant difference in
correlation estimation errors between the linear decay function and full
contrast conditions. This result was surprising, however on closer
inspection of the scatterplots in the linear decay function condition,
it is clear that the logarithmic nature of contrast perception
\cite{fechner_1948,
varshney_2013} means there was little perceptual distance between points
with high opacity values (\emph{alpha} \textgreater{} 0.75). This
resulted in no significant perceived differences between full opacity
and linear decay function scatterplots. A similar threshold for high
opacity values was found in Experiment 1. Selecting only lower \emph{r}
values, those with naturally higher residuals (arbitrarily \emph{r}
\textless{} 0.6), still results in no difference between correlation
estimation errors for linear decay parameters and full opacity
conditions (\(\chi^2\)(1) = 0.09, \emph{p} = .769). Effect sizes were
small, with the largest being between the baseline full opacity and
inverted non-linear decay conditions. This suggests that it is easier to
induce further bias in correlation estimates through a reduction in the
salience of a point cloud's centre than it is to correct for the
underestimation bias.

Looking at the standard deviations of correlation estimates plotted
separately by opacity decay function and \emph{r} value in
Figure~\ref{fig-estimates-by-r-e2}, it can be observed that as in
Experiment 1 (see Figure~\ref{fig-estimates-by-r-e1}), and aside from
the inverted non-linear decay function condition, precision in \emph{r}
estimation increased with the objective \emph{r} value. This finding
corroborates previous work. In the inverted non-linear decay condition,
as \emph{r} approaches 1 (and point residuals accordingly approach 0),
the opacity of points diminishes. Just as the standard deviation of
correlation estimates was higher for the low global opacity condition in
Experiment 1, having lower opacity points at the high \emph{r} end of
the non-linear inverted condition in Experiment 2 resulted in fairly
constant standard deviations across \emph{r} values, as the usual
reduction towards \emph{r} = 1 was not observed.

The non-linear inverted opacity decay condition produced significantly
lower estimates of correlation than all other conditions. This adds
perspective to suggestions \cite{yang_2019} that, among other visual
features, the area of a hypothetical prediction ellipse
\cite{cleveland_1982, yang_2019}, a region used to predict new
observations assuming a bivariate normal distribution (see
Figure~\ref{fig-prediction-ellipse}) is a better predictor of
correlation estimation performance than the objective \emph{r} value
itself. In the inverted non-linear decay condition, the area of this
ellipse remained the same, yet estimates of correlation differed
significantly. These results suggest that the apparent density of the
scatterplot point cloud also has effects on estimates of correlation.
Prior work has found that more dense scatterplots are rated as having
higher correlations \cite{lauer_1989, rensink_2014}, although the effect
found was weak. To explore this effect further, work investigating what
people attend to when making correlation judgements must be completed.
Eye-tracking offers an elegant solution to this problem, yet at the time
of writing, had only been used for simpler scatterplot-related tasks,
such as those asking participants to identify the number of, or distance
between, points \cite{netzel_2017}.

\begin{figure}

\centering{

\includegraphics[width=\textwidth]{4_adjusting_opacity_files/figure-latex/fig-prediction-ellipse-1.pdf}

}

\caption{\label{fig-prediction-ellipse}Plot showing a 95\% prediction
ellipse over a scatterplot with an \textit{r} value of 0.6.}

\end{figure}%

\section{General Discussion}\label{general-discussion-adjusting-opacity}

To summarise the results, evidence was found that changing the opacities
of points in scatterplots can significantly alter participants'
estimates of correlation, that lower global point opacity is associated
with greater correlation underestimation error, that lowering point
opacity as a function of residual error can partially correct for this
underestimation, and that raising opacity as a function of residual
error can increase the underestimation bias. These findings pave the way
for using changes in point opacity to produce perceptually-optimised
scatterplots that do not rely on data removal. As the focus of this work
is on designing data visualisations that lay people are more easily able
to interpret and understand, we used a large, representative sample,
including people from a range of nationalities and educational
backgrounds. This chapter demonstrates that a simple framework can be
employed with these groups to gather high quality data, and, by design,
produce conclusions that may generalise better to in-the-wild data
visualisation usage.

In agreement with previous research
\cite{pollack_1960, rensink_2010, rensink_2012,
rensink_2014, rensink_2017}, participants were more accurate and precise
in their estimates of correlation when the \emph{r} value shown in the
scatterplot was higher. Figure~\ref{fig-estimates-by-r-e1} and
Figure~\ref{fig-estimates-by-r-e2} plot objective \emph{r} value against
participants' errors in correlation estimation separately for each level
of the respective independent variable. These plots illustrate, as in
much previous work, the lower levels of precision and accuracy in
correlation estimation that are seen for \emph{r} values further from 0
or 1.

The results here contribute to a body of evidence that suggests
participants are attending to the width of the probability distribution
displayed in a scatterplot
\cite{cleveland_1982, meyer_1997, rensink_2017, yang_2019} when making
judgements of correlation. Further evidence is provided for the
systematic underestimation bias, and a potential correction strategy is
offered. This work does not attempt to redesign the scatterplot as a
medium, but to provide a set of recommendations for designers based on
the evidence; when designing positively correlated scatterplots to
support correlation perception:

\begin{itemize}
\tightlist
\item
  Lowering the total opacity in a scatterplot can cause people to
  underestimate correlation compared to when contrast is maximal between
  the points and the plot background (when point opacity is high).
\item
  The use of a non-linear opacity decay function, in which point opacity
  falls as a function of residual size, can be used to counteract the
  underestimation seen in correlation estimation in positively
  correlated scatterplots.
\end{itemize}

Scatterplots are widely used, and are often designed with a number of
communicative concepts in mind. When one of these concepts is
illustrating the degree of positive association between two variables,
the findings presented suggest that designers should utilise the
techniques described to give viewers the best chance of interpreting the
correlation displayed as accurately as possible.

\subsection{Training}\label{training-adjusting-opacity}

Both experiments tested lay participants with varying levels of graph
literacy. Due to concerns about participants' familiarity with
scatterplots, each saw four scatterplots depicting correlations of
\emph{r} = 0.2, 0.5, 0.8, and 0.95 (see
Figure~\ref{fig-training-slide-adjusting-opacity}) to familiarise
themselves with the concept. To test if the patterns of results seen in
correlation estimation could be attributed to this training, models were
built that included the half of the session (first or second) as a
predictor for participants' judgements of correlation. Comparing these
models to the original revealed a significant effect in Experiment 1
(\(\chi^2\)(1) = 7.60, \emph{p} = 0.01), but not Experiment 2
(\(\chi^2\)(1) = 2.20, \emph{p} = 0.14). In Experiment 1, participants'
errors in correlation estimation were higher in the second half of the
experiment, suggesting that having more recently viewed the training
material may have made participants more accurate. That this was not
observed in Experiment 2 implies that the point opacity manipulation had
a greater effect on estimates than any training effects.

\subsection{Limitations}\label{limitations-adjusting-opacity}

The results in Experiment 2 provide evidence that reducing the salience
of points as they move further from the regression line can increase
people's estimates of correlation, at least when plots like these are
presented alongside conventional ones. Testing whether this phenomenon
would exist with a plot in isolation would present a number of
difficulties. As can be seen in Figure~\ref{fig-estimates-by-r-e1} and
Figure~\ref{fig-estimates-by-r-e2}, participants' estimates of
correlation, especially between 0.2 and 0.7, suffer from high variance.
High numbers of trials and participants ameliorate this to an extent,
but this does prevent commentary on single plot judgements of
correlation.

An important first step in the utilisation of point opacity changes to
optimise perception of correlation in scatterplots has been made,
however there remains much to do. Quantitative determination of whether
0.25 is indeed the most optimal value for \(b\) in equation 1, for
example, is impossible from the present results. It may well be the case
that changing the value of \(b\) as a function of the objective
Pearson's \emph{r} value could produce more accurate correlation
estimation in participants; findings that participants were more
accurate in correlation estimation when \emph{r} was nearer 0 or 1 would
suggest that the use of a decay parameter for these correlations is
unnecessary.

The simplicity of the direct estimation task employed confers some
limitations on the conclusions that can be drawn, although these
limitations do not prevent the data gathered from being practically
useful. The methods used do not allow for the investigation of absolute
correlation perception, as JND or bisection methodologies might. The
finding that mean errors in judgements of correlation for full opacity
conditions were different between Experiments 1 (0.149) and 2 (0.123)
makes this limitation evident. In the experimental paradigm,
participants indirectly made comparative correlation judgements, which
may have resulted in this discrepancy. Nevertheless, the results found
are promising with regards to design research.

\subsection{Future Work}\label{future-work-adjusting-opacity}

Future work may wish to investigate negative values of \emph{r}, given
findings that people may overestimate the correlation of negatively
correlated scatterplots in a similar way \cite{sher_2017}. The
techniques presented here could be adapted to bias perceptions of
correlation down and correct for this overestimation. The non-linear
inverted opacity decay function demonstrated here may be able to
accomplish this.

The opacity decay function conditions in Experiment 2 used the vertical
distance between a particular point and the regression line to set that
point's opacity. Previous work \cite{meyer_1997} has suggested that the
perpendicular distance between a point and the regression line may be a
more accurate predictor of performance on correlation estimation tasks
\cite{cleveland_1982, rensink_2017, yang_2019}. Future work exploring
these manipulations further may wish to investigate whether there are
differences between using perpendicular and vertical residual distance
as bases for setting point opacity with regard to correlation
estimation.

Finally, the most exciting avenue for related future work is the
possibility of combining the techniques developed and tested here with
other novel scatterplot visualisation techniques that have a basis in
the literature. \chap{chap:adjusting_size} investigates the use of point
size in place of opacity with the techniques described, while
\chap{chap:belief_change} combines the two to explore how these
different modalities work together with regards to the estimation of
positive correlation.

\section{Conclusion}\label{conclusion-chap4}

In a pair of experiments, I varied the opacity of point in scatterplots
both uniformly (Experiment 1) and using functions relating point opacity
to the size of a particular point's residual (Experiment 2). In
Experiment 1, I showed that, in contradiction with previous work,
changing the opacity of all points in a scatterplot can effect
participants' performance on a correlation estimation task. In
Experiment 2, I showed that varying point opacity as a function of a
point's residual distance is able to significantly change participants'
correlation estimates and partially correct for a long-standing
underestimation bias.




\end{document}
