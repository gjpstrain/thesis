\documentclass[../main.tex]{subfiles}
\begin{document}


Changing the opacities and sizes of points in positively correlated
scatterplots has clear effects on the estimation of correlation. This
thesis set out to investigate whether visual manipulations of
scatterplots based on aspects of human perception might be able to
change estimates of correlation, and, in particular, whether these
visual manipulations could be used to correct for a historical
underestimation bias observed since the 1960s.

\section{Main Findings}\label{main-findings}

Each published experimental chapter in this work builds on the previous
and provides a unique contribution to knowledge about how changing
visual features may affect thoughts and perceptions about correlations.

In \chap{chap:adjusting_opacity}, the opacities of scatterplot points
were altered while participants were tested on a correlation estimation
task. The two published experiments in this chapter were partially based
on an earlier pilot study that ultimately remains unpublished due to
data quality issues (see Section \ref{pilot-study} for a full account of
this). In the first experiment in this chapter, the opacities of
scatterplot points were changed in a uniform manner; this was done with
the intent of discovering any general effects of point opacity on
correlation estimation. Evidence was found for an increase in
underestimation error when global point opacity was reduced. This result
provided evidence that, in disagreement with previous work, changing the
opacities of scatterplot points could be used to alter estimates of the
correlation therein. Based on these findings, data from the pilot study,
and evidence that the arrangement of points in scatterplots may be a
good proxy for correlation judgements, Experiment 2 varied point opacity
using functions related to each point's distance from a regression line.
In this experiment, evidence was found that reducing the opacity of a
point as it moved further from the regression line could bias
participants' estimates of correlation upwards to partially correct for
the underestimation bias. While this effect was small, it showed that
estimates of correlation could be changed to correct for bias
\emph{without} removing any data from the plot.

\chap{chap:adjusting_size} expanded these findings to a second visual
modality (point size). Building on findings from
\chap{chap:adjusting_opacity} and evidence that the size of a stimulus
may have stronger effects on perception than its opacity, a single
experiment took the functions relating opacity and residual magnitude in
the previous chapter and applied them to point size. The effect on
correlation estimation observed here was much stronger than that
observed in \chap{chap:adjusting_opacity}. Additionally, several changes
in the nature of the estimation curve were observed in this experiment.
Firstly, while the shape of the estimation curve when using point
opacity manipulations was identical to that when using no visual
adjustments, the use of size decay manipulations such that point size
decreased with residual magnitude resulted in an estimation curve with a
visibly altered shape. Secondly, in previous work, and in all
experiments undertaken in \chap{chap:adjusting_opacity}, precision in
correlation estimation increased as the objective \emph{r} value itself
increases; this behaviour was not observed when employing manipulations
that alter point size. The non-linear decay condition in this experiment
remains the most promising condition with regards to correcting for the
underestimation bias.

In a final implementation of the same experimental paradigm,
\chap{chap:interactions_opacity_size} combined point opacity and size
manipulations from the two previous chapters to further explore the
potential for biasing estimates of correlation and the potential
mechanisms behind the effects observed. Using point opacity and size
manipulations together in both congruent (same direction) and
incongruent (opposing directions) combinations, this single experiment
showed that employing both manipulations was able to bias correlation
estimates further than either manipulation in isolation, resulting in a
marked overestimation being introduced. This combination is not,
however, linearly additive, adding support to the idea that there are
different mechanisms of action behind the effects of each visual
manipulation on correlation estimation. This experiment provided
evidence that, with regards to using point opacity and size
manipulations to bias correlation estimates, there are few limits.
Figure~\ref{fig-mean-all-exp} summarises the mean estimation errors from
all 4 experiments described in \chap{chap:adjusting_opacity},
\chap{chap:adjusting_size}, and \chap{chap:interactions_opacity_size}.

\begin{figure}

\centering{

\includegraphics[width=\textwidth]{8_conclusion_files/figure-latex/fig-mean-all-exp-1.pdf}

}

\caption{\label{fig-mean-all-exp}Estimated Marginal Mean errors in
correlation estimation for Experiments 1 to 4. This figure is an amalgam
of Figures \ref{fig-e1-estimates}, \ref{fig-e2-estimates},
\ref{fig-e3-estimates}, and \ref{fig-e4-estimates}. 95\% confidence
intervals are shown as error bars. The vertical dashed line represents
no estimation error.}

\end{figure}%

Extending these findings into a cognitive space,
\chap{chap:belief_change} utilised the congruent typical orientation
size and opacity decay condition from Experiment 4, contextualised as
part of news items, to explore whether the perceptual effects found in
the previous three chapters might also be able to bias people's beliefs
about the levels of relatedness between variable pairs. In this
experiment, people's beliefs about a pair of variables were tested
before and after viewing a series of data-identical scatterplots with
either the congruent typical orientation opacity and size decay
condition, or with no opacity or size decay present. While beliefs about
relatedness were changed in each condition, they changed to a greater
degree when participants viewed the scatterplots with opacity and size
decay present. This finding provided evidence that the perceptual
effects of systematically altering the opacities and sizes of
scatterplot points can be extended into a cognitive space, and are able
to change what people believe about the levels of relatedness between
variable pairs.

\section{Relationship to Prior Work}\label{relationship-to-prior-work}

Overall, the work presented in this thesis contributes to wider evidence
that simple changes in visual features can have perceptual
\cite{hong_2022} and cognitive \cite{karduni_2020} effects on those who
view such visualisations. The finding that changing the opacities and
sizes of scatterplot points can affect viewers' estimates of correlation
is in direct contradiction to earlier work
\cite{rensink_2012, rensink_2014}, which found no effects of changing
point opacities and sizes in different ways. The novelty of the findings
in this thesis necessarily means that commenting on the potential
mechanisms behind them is difficult. Taking the findings from
\chap{chap:adjusting_opacity}, \chap{chap:adjusting_size}, and
\chap{chap:interactions_opacity_size} together, the balance of evidence
suggests that decreasing the opacity or size of more exterior
scatterplot points is able to bias estimates of correlation upwards by
reducing the influence of those points when viewers make their
estimates. The results presented in \chap{chap:belief_change} provide
evidence that cognitions about data visualisations can also be altered
by small changes in visual presentation. This is in line with a large
body of work demonstrating that small changes in visual presentation -
often of the kind that may be considered only as aesthetic preference by
designers - can lead to large changes in perceptions and the conclusions
that viewers of visualisations draw; see Fygenson et al., 2023
\cite{fygenson_2023} for examples with bar chart spacing and
arrangement, Moritz et al., 2023 \cite{moritz_2023} for examples with
line graphs, and Bradley et al., 2025 \cite{bradley_2025} for examples
with axis limits and truncation. Franconeri et al.~(2021
\cite{franconeri_2021}) and Szafir (2018 \cite{szafir_2018}) both
provide excellent wider reviews on how poor design can lead to
visualisation-driven miscommunication and bias.

\section{Reproducibility}\label{reproducibility}

As detailed in \chap{chap:gen_methods}, this project has been conducted
according to the principles of open and reproducible science
\cite{ayris_2018}. Doing so effectively facilitates future replication
and authentication of the published results, in addition to enabling
others to learn from where I have succeeded and failed. Materials are
provided throughout for the re-creation of each individual experimental
paper, including Docker containers to reproduce the specific
computational environments used. This thesis also features an all-in-one
Docker implementation, details of which can be found in the repository
associated with this paper. Owing to the accessibility issues present
with the PDF format \cite{kumar_2024}, this thesis is also hosted as a
website\footnote{placeholder} using GitHub Pages.

\section{Contributions}\label{contributions-conclusion}

This thesis addresses a long-standing bias in the interpretation of
correlation in positively correlated scatterplots by drawing on
influences from perceptual psychology and data visualisation design. A
thorough exploration of the effects of two salient visual features
(opacity and size) on correlation estimation was carried out over a
series of well-powered experiments in \chap{chap:adjusting_opacity},
\chap{chap:adjusting_size}, and \chap{chap:interactions_opacity_size}.
Furthermore, it was shown in \chap{chap:belief_change} that the effects
of point opacity and size with regards to the estimation of positive
correlation could be extended into a cognitive space to change peoples'
beliefs about the levels of relatedness between a pair of variables.

The most important contribution this thesis makes is to the
visualisation design and HCI audiences; secondarily, I add to a growing
body of work that, taken together, may one day be able to reveal
precisely how viewers' interpret correlation from scatterplots.

Additionally, I hope to have provided an example of a PhD project
conducted from the ground up in an open and reproducible manner. A PhD
project is as much a story as it is a piece of research, and through
acknowledgement of the successes and failures of the work I have done,
along with the employment of a host of reproducibility-enabling tools, I
hope to contribute an honest account of what one PhD project may look
like.

\section{Implications}\label{implications}

\subsection{For Design}\label{for-design}

It is clear from the results presented in this thesis that designers
must be aware of the potential effects on perception and cognition of
aesthetic choices in data visualisation design. This is particularly
salient given that changes in point opacity and size are both common
techniques in scatterplot design. The former is often used to reduce
overplotting issues, while the latter is used to create trivariate
bubble charts. Given that many data visualisations are designed with a
range of communicative intentions, designers may be inadvertently
sacrificing effectiveness in one modality (correlation perception), for
clarity in another (reduction in overplotting, for example); while this
trade-off may be acceptable, it is important for designers to be aware
of the effects their designs have on people's thoughts and feelings
about the data stories they are being told.

On the other hand, this work also provides marked opportunities. Where
appropriate, using some form of the point opacity and/or size
manipulations described in this thesis could enhance the communicative
efficacy of positively correlated scatterplots. I would not recommend
this until further work has been done on the potential for
misinterpretation from scatterplots such as those described here (see
Section \ref{future-work-chap8}).

\subsection{For Society}\label{for-society}

Given the ubiquity of scatterplots and the fundamental nature of
correlation as a mathematical concept, the findings here have the
potential to change the way that people engage with visualisations
depicting relatedness. This is especially pertinent given the findings
detailed in \chap{chap:belief_change} that suggest that the techniques
described here are also able to bias viewers' beliefs about
correlations. Whether or not such techniques would be appropriate is a
question that requires further work to answer.

Additionally, I hope that this project is demonstrative of the need for
the re-examination of common data visualisations from the ground up. In
an age where everything is measured, and the complexity of information
increases year on year, data visualisations are crucial for the
communication of difficult and complex topics to the general public.
Many of the most popular, widely-used data visualisations are not
designed to be accurately interpreted by lay people; given that this is
the group that represents the greatest viewership, and that has the most
to gain from engaging with data visualisations, more design must take
place that acknowledges and addresses the ways in which perceptions and
cognitions around data visualisations are biased. Designing in this way
requires humility; as designers, we must recognise that biases exist in
common data visualisations and work thoroughly to address them.

\section{Limitations}\label{limitations}

For \chap{chap:adjusting_opacity}, \chap{chap:adjusting_size}, and
\chap{chap:interactions_opacity_size}, the shared experimental method
confers the largest limitation in this thesis. As opposed to previous
work that employed more psychophysical designs, such as correlation
discrimination or magnitude estimation via bisection tasks
\cite{rensink_2010, rensink_2017}, Experiments 1 to 4 in this thesis
utilised a direct estimation paradigm more akin to much older
correlation perception work
\cite{strahan_1978, bobko_1979, cleveland_1982, lane_1985, lauer_1989, collyer_1990,
meyer_1992}. While this technique supplies robust conclusions for the
HCI and design audiences, it means that making claims about the precise
mathematical relationship between perceived and actual correlation is
difficult based on the data gathered. Similarly, the correlation
judgements that participants made in these experiments must be
considered in context; it would not be expected that people may view 180
data visualisations in sequence during the course of normal, day-to-day
visualisation viewing.

All experiments presented in this thesis were conducted entirely online.
Doing so allowed for the rapid collection of large amounts of high
quality data, however necessarily limits the amount of additional data
that can be collected. A method for determining the dot pitch of
participants' monitors was employed from \chap{chap:adjusting_size}
onwards and was included in statistical modelling paradigms. Generally,
this explained little to no variance in participants' responses. Online
experimentation does not allow for the determination of other factors
that may explain variance, such as head-to-monitor distance or monitor
specifications that may influence the visual quality of scatterplots
with altered point opacities or sizes.

\section{Future Directions}\label{future-work-chap8}

There are numerous avenues for future work based on this thesis, many of
which are relatively simple extensions of the techniques described.
Firstly, all experiments in this thesis took place using positively
correlated scatterplots. There is evidence that, similarly to how
viewers \emph{underestimate} correlation in positively correlated
scatterplots, they may also \emph{overestimate} correlation in
negatively correlated scatterplots \cite{sher_2017}. Uniquely among
correlation visualisations, the visual forms of positively and
negatively correlated scatterplots are symmetrical \cite{harrison_2014},
suggesting that the techniques modifying point opacity and size
described in this thesis may be used (in a symmetrical manner) to
address the overestimation bias. Doing so would require a relatively
simple replication of a subset of the experiments described here using
negatively correlated scatterplots.

This thesis represents a large amount of quantitative data on how
viewers perceive correlation in scatterplots, and how their beliefs
about relatedness may be changed following scatterplot viewing. Before
recommendations can be firmly made about the potential use of these
techniques, however, qualitative work should take place investigating
users' thoughts and feelings about novel visualisations that have been
designed based on perceptual metrics. It may be the case that any
potential advantages of the novel scatterplots described herein are
outweighed by distrust or unfamiliarity with novel designs.

Finally, to further investigate precisely \emph{why} the techniques
described in this thesis work, lab-based experiments should be
conducted. While more costly and time-consuming, this would allow for
far more control over the precise experimental set-up, and would
additionally allow for the employment of eye-tracking to understand how
changing point opacities and sizes may influence perceptions of
correlation.

\section{Closing Remarks}\label{closing-remarks}

The design of data visualisations should be based, at least partially,
on the ways in which those data visualisations are perceived by the
people that use them. The opacities and sizes of points in positively
correlated scatterplots have clear, strong effects on the perceptions of
correlation based on them. The extent to which viewers update their
beliefs following scatterplot-viewing can also depend on the opacities
and sizes of scatterplot points. These findings highlight that small
changes in the visual features of common data visualisations may produce
large changes in the perceptions and cognitions of viewers.




\end{document}
