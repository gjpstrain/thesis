\documentclass[../main.tex]{subfiles}
\begin{document}


Changing the opacities and sizes of point in positively correlated
scatterplots has clear effects on the estimation of correlation. This
thesis set out to investigate whether visual manipulations of
scatterplots based on aspects of human perception might be able to
change estimates of correlation, and, in particular, whether these
visual manipulations could be used to correct for a historic
underestimation bias present since the 1960s.

\section{Main Findings}\label{main-findings}

Each published experimental chapter in this work builds on the previous
and provides a unique contribution to knowledge about how changing
visual features may affect thoughts and perceptions about correlations.

In \chap{chap:adjusting_opacity}, the opacities of scatterplot points
were altered while participants were tested on a correlation estimation
task. The two published experiments in this chapter were partially based
on an earlier pilot study that ultimately remains unpublished due to
data quality issues (see \ref{pilot-study} for a full account of this).
In the first experiment in this chapter, the opacities of scatterplot
points were changed in a uniform manner; this was done with the intent
of discovering any general effects of point opacity on correlation
estimation. Evidence was found for an increase in underestimation error
when global point opacity was reduced. This result provided evidence
that, in disagreement with previous work, changing the opacities of
scatterplot points could be used to alter estimates of the correlation
therein. Based on these findings, data from the pilot study, and
evidence that the arrangement of point in scatterplots may be a good
proxy for correlation judgements, experiment 2 varied point opacity
using functions related to each point's distance from a regression line.
In this experiment, evidence was found that reducing the opacity of a
point as it moved further from the regression line could bias
participant's estimates of correlation upwards to partially correct for
the underestimation bias. While this effect was small, it showed that
estimates of correlation could be changed to correct for bias
\emph{without} removing any data from the plot.

\chap{chap:adjusting_size} expanded these findings to a second visual
modality. Building on findings from \chap{chap:adjusting_opacity} and
evidence that the size of a stimulus may have stronger effects on
perception than opacity, a single experiment took the functions relating
opacity and residual distance in the previous chapter and applied them
to point size. The effect on correlation estimation observed here was
much stronger than that observed in \chap{chap:adjusting_opacity}.
Additionally, several changes in the nature of the estimation curve were
observed in this experiment. Firstly, while the shape of the estimation
curve when using point opacity manipulations was identical to that when
using no visual adjustments, the use of size decay manipulations such
that point size decreased with residual error resulted in an estimation
curve with a visibly altered shape. Secondly, in previous work, and in
all experiments undertaken in \chap{chap:adjusting_opacity}, precision
in correlation estimation has been observed to increase as the objective
\emph{r} value itself increases; this behaviour was not observed when
employing manipulations that alter point size. The non-linear decay
condition in this experiment remains the most promising condition with
regards to correcting for the underestimation bias.

In a final implementation of the same experimental paradigm,
\chap{chap:interactions_opacity_size} combines point opacity and size
manipulations from the two previous chapters to further explore the
potential for biasing estimates of correlation and the potential
mechanisms behind the effects observed. Using point opacity and size
manipulations together in both congruous (same direction) and
incongruous (opposing directions) combinations, this single experiment
showed that employing both manipulations was able to bias correlation
estimates further than either manipulation in isolation, resulting in a
marked overestimation being introduced. This combination is not,
however, linearly additive, suggesting different mechanisms of action
behind the effects of each visual manipulation on correlation
estimation. This experiment provided evidence that with regards to using
point opacity and size manipulations to bias correlation estimates,
there are few limits. Figure~\ref{fig-mean-all-exp} summarises the mean
estimation errors from all 4 experiments described in
\chap{chap:adjusting_opacity}, \chap{chap:adjusting_size}, and
\chap{chap:interactions_opacity_size}.

\begin{figure}

\centering{

\includegraphics[width=\textwidth]{8_conclusion_files/figure-latex/fig-mean-all-exp-1.pdf}

}

\caption{\label{fig-mean-all-exp}Estimated Marginal Mean errors in
correlation estimation for experiments 1 to 4. This figure is an
amalgamation of Figures \ref{fig-e1-estimates}, \ref{fig-e2-estimates},
\ref{fig-e3-estimates}, and \ref{fig-e4-estimates}. 95\% confidence
intervals are shown as error bars. The vertical dashed line represents
no estimation error.}

\end{figure}%

\section{Relationship to Prior Work}\label{relationship-to-prior-work}

\section{Reproducibility}\label{reproducibility}

As detailed in \chap{chap:gen_methods}, this project has been conducted
according to the principles of open and reproducible science
\cite{ayris_2018}. Doing so effectively facilitates future replication
and authentication of my published results, in addition to enabling
others to learn from where I have succeeded and failed. Materials are
provided throughout for the re-creation of each individual experimental
paper, including Docker containers to reproduce the specific
computational environments used. This thesis also features an all-in-one
Docker implementation, details of which can be found in the repository
associated with this paper. Owing to the accessibility issues present
with the PDF format \cite{kumar_2024}, this thesis is also hosted as a
website\footnote{placeholder}.

\section{Contributions}\label{contributions-conclusion}

\section{Implications}\label{implications}

\subsection{For Design}\label{for-design}

\subsection{For Society}\label{for-society}

\section{Limitations}\label{limitations}

\section{Future Directions}\label{future-work-chap8}

\section{Closing Remarks}\label{closing-remarks}




\end{document}
