\documentclass[../main.tex]{subfiles}
\begin{document}


Changing the opacities and sizes of point in positively correlated
scatterplots has clear effects on the estimation of correlation. This
thesis set out to investigate whether visual manipulations of
scatterplots based on aspects of human perception might be able to
change estimates of correlation, and, in particular, whether these
visual manipulations could be used to correct for a historic
underestimation bias present since the 1960s.

\section{Main Findings}\label{main-findings}

Each published experimental chapter in this work builds on the previous
and provides a unique contribution to knowledge about how changing
visual features may affect thoughts and perceptions about correlations.

In \chap{chap:adjusting_opacity}, the opacities of scatterplot points
were altered while participants were tested on a correlation estimation
task. The two published experiments in this chapter were partially based
on an earlier pilot study that ultimately remains unpublished due to
data quality issues (see \ref{pilot-study} for a full account of this).
In the first experiment in this chapter, the opacities of scatterplot
points were changed in a uniform manner; this was done with the intent
of discovering any general effects of point opacity on correlation
estimation. Evidence was found for an increase in underestimation error
when global point opacity was reduced. This result provided evidence
that, in disagreement with previous work, changing the opacities of
scatterplot points could be used to alter estimates of the correlation
therein. Based on these findings, data from the pilot study, and
evidence that the arrangement of point in scatterplots may be a good
proxy for correlation judgements, experiment 2 varied point opacity
using functions related to each point's distance from a regression line.
In this experiment, evidence was found that reducing the opacity of a
point as it moved further from the regression line could bias
participant's estimates of correlation upwards to partially correct for
the underestimation bias. While this effect was small, it showed that
estimates of correlation could be changed to correct for bias
\emph{without} removing any data from the plot.

\chap{chap:adjusting_size} expanded these findings to a second visual
modality. Building on findings from \chap{chap:adjusting_opacity} and
evidence that the size of a stimulus may have stronger effects on
perception than opacity, a single experiment took the functions relating
opacity and residual distance in the previous chapter and applied them
to point size. The effect on correlation estimation observed here was
much stronger than that observed in \chap{chap:adjusting_opacity}.
Additionally, several changes in the nature of the estimation curve were
observed in this experiment. Firstly, while the shape of the estimation
curve when using point opacity manipulations was identical to that when
using no visual adjustments, the use of size decay manipulations such
that point size decreased with residual error resulted in an estimation
curve with a visibly altered shape. Secondly, in previous work, and in
all experiments undertaken in \chap{chap:adjusting_opacity}, precision
in correlation estimation has been observed to increase as the objective
\emph{r} value itself increases; this behaviour was not observed when
employing manipulations that alter point size. The non-linear decay
condition in this experiment remains the most promising condition with
regards to correcting for the underestimation bias.

In a final implementation of the same experimental paradigm,
\chap{chap:interactions_opacity_size} combines point opacity and size
manipulations from the two previous chapters to further explore the
potential for biasing estimates of correlation and the potential
mechanisms behind the effects observed. Using point opacity and size
manipulations together in both congruous (same direction) and
incongruous (opposing directions) combinations, this single experiment
showed that employing both manipulations was able to bias correlation
estimates further than either manipulation in isolation, resulting in a
marked overestimation being introduced. This combination is not,
however, linearly additive, suggesting different mechanisms of action
behind the effects of each visual manipulation on correlation
estimation. This experiment provided evidence that with regards to using
point opacity and size manipulations to bias correlation estimates,
there are few limits. Figure~\ref{fig-mean-all-exp} summarises the mean
estimation errors from all 4 experiments described in
\chap{chap:adjusting_opacity}, \chap{chap:adjusting_size}, and
\chap{chap:interactions_opacity_size}.

\begin{figure}

\centering{

\includegraphics[width=\textwidth]{8_conclusion_files/figure-latex/fig-mean-all-exp-1.pdf}

}

\caption{\label{fig-mean-all-exp}Estimated Marginal Mean errors in
correlation estimation for experiments 1 to 4. This figure is an
amalgamation of Figures \ref{fig-e1-estimates}, \ref{fig-e2-estimates},
\ref{fig-e3-estimates}, and \ref{fig-e4-estimates}. 95\% confidence
intervals are shown as error bars. The vertical dashed line represents
no estimation error.}

\end{figure}%

Extending these findings into a cognitive space,
\chap{chap:belief_change} utilised the congruous typical orientation
size and opacity decay condition from experiment 4, contextualised as
part of news items, to explore whether the perceptual effects found in
the previous three chapter might also be able to bias people's beliefs
about the levels of relatedness between variable pairs. In this
experiment, people's beliefs about a pair of variables were tested
before and after viewing a series of data-identical scatterplots with
either the congruous typical orientation opacity and size decay
condition, or with no opacity or size decay present. While beliefs about
relatedness were changed in each condition, they changed to a greater
degree when participants viewed the scatterplots with opacity and size
decay present. This finding provided evidence that the perceptual
effects of systematically altering the opacities and sizes of
scatterplot points can be extended into a cognitive space, and are able
to change what people believe about the levels of relatedness between
variable pairs.

\section{Relationship to Prior Work}\label{relationship-to-prior-work}

Overall, the work presented in this thesis contributes to wider evidence
that simple changes in visual features can have perceptual
\cite{hong_2022} and cognitive \cite{karduni_2020} effects on those who
view such visualisations. The finding that changing the opacities and
sizes of scatterplot points can affect viewers' estimates of correlation
is in direct contradiction to earlier work
\cite{rensink_2012, rensink_2014}, which found no effects whatsoever of
changing point opacities and sizes in different ways. The novelty of the
findings in this thesis necessarily means that commenting on the
potential mechanisms behind them is difficult. Taking the findings from
\chap{chap:adjusting_opacity}, \chap{chap:adjusting_size}, and
\chap{chap:interactions_opacity_size} together, the balance of evidence
suggests that decreasing the opacity or size of more exterior
scatterplot points is able to bias estimates of correlation upwards by
reducing the influence of those points when viewers make their
estimates. The results presented in \chap{chap:belief_change} provide
evidence that cognitions about data visualisations can also be altered
by small changes in visual presentation. This is in line with a large
body of work demonstrating that small changes in visual presentation -
often of the kind that may be considered only as aesthetic preference by
designers - can lead to large changes in perceptions and the conclusions
that viewers of visualisations draw; see Fygenson et al.~(2023
\cite{fygenson_2023}) for examples with bar chart spacing and
arrangement, Moritz et al.~(2023 \cite{moritz_2023}) for examples with
line graphs, and Bradley et al.~(2025 \cite{bradley_2025}) for examples
with axis limits and truncation. Franconeri et al.~(2021
\cite{franconeri_2021}) and Szafir (2018 \cite{szafir_2018}) both
provide excellent wider reviews on how poor design can lead to
visualisation-driven miscommunication and bias.

\section{Reproducibility}\label{reproducibility}

As detailed in \chap{chap:gen_methods}, this project has been conducted
according to the principles of open and reproducible science
\cite{ayris_2018}. Doing so effectively facilitates future replication
and authentication of my published results, in addition to enabling
others to learn from where I have succeeded and failed. Materials are
provided throughout for the re-creation of each individual experimental
paper, including Docker containers to reproduce the specific
computational environments used. This thesis also features an all-in-one
Docker implementation, details of which can be found in the repository
associated with this paper. Owing to the accessibility issues present
with the PDF format \cite{kumar_2024}, this thesis is also hosted as a
website\footnote{placeholder}.

\section{Contributions}\label{contributions-conclusion}

This thesis addresses a long-standing bias in the interpretation of
correlation in positively correlated scatterplots by drawing on
influences from perceptual psychology and data visualisation design. A
thorough exploration of the effects of two salient visual features
(opacity and size) on correlation estimation was carried out over a
series of well-powered experiments in \chap{chap:adjusting_opacity},
\chap{chap:adjusting_size}, and \chap{chap:interactions_opacity_size}.
Furthermore, it was shown that the effects of point opacity and size
with regards to the estimation of positive correlation could be extended
into a cognitive space to change peoples' thoughts about the levels of
relatedness between a pair of variables.

The most important contribution this thesis makes is to the
visualisation design and HCI audiences; secondarily, I add to a growing
body of work that, taken together, may one day be able to reveal
precisely how viewer's interpret correlation from scatterplots.

Additionally, I hope to have provided an example of a PhD project
conducted from the ground up in an open and reproducible manner. A PhD
project is as much a story as it is a piece of research, and through
acknowledgement of the successes and failures of the work I have done,
along with the employment of a host of reproducibility-enabling tools, I
hope to contribute an honest account of what one PhD project may look
like.

\section{Implications}\label{implications}

\subsection{For Design}\label{for-design}

It is clear from the results presented in this thesis that designers
must be aware of the potential effects on perception and cognition of
aesthetic choices in data visualisation design. This is particularly
salient given that point opacity changes and point size changes are both
common techniques in scatterplot design. The former is often used to
reduce overplotting issues, while the latter is used to create
trivariate bubble charts. Given that many data visualisations are
designed with a range of communicative intentions, designers may be
inadvertently sacrificing effectiveness in one modality (correlation
perception), for clarity in another (reduction in overplotting, for
example); while this trade-off may be acceptable, it is important for
designers to be aware of the effects their designs have on people's
thoughts and feelings about the data stories they are being told.

On the other hand, this work also provides marked opportunities for
designers. Where appropriate, using some form of the point opacity
and/or size manipulations described in this thesis could enhance the
communicative efficiency of positively correlated scatterplots.

\subsection{For Society}\label{for-society}

\section{Limitations}\label{limitations}

\section{Future Directions}\label{future-work-chap8}

\section{Closing Remarks}\label{closing-remarks}




\end{document}
