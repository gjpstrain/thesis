\documentclass[../main.tex]{subfiles}
\begin{document}


\section{Abstract}\label{abstract-beliefs}

\chap{chap:adjusting_opacity}, \chap{chap:adjusting_size}, and
\chap{chap:interactions_opacity_size} show through four experiments that
point opacity and size manipulations can have powerful effects on
participants' estimates of correlation in positively correlated
scatterplots. In \chap{chap:adjusting_opacity}, global and
spatially-dependent adjustments in point opacity were employed, and a
small, but statistically significant level of correction for the
underestimation of positive correlation was found. Spatially-dependent
adjustment of point size, in which size is reduced as a function of
residual magnitude, was found in \chap{chap:adjusting_size} to produce
much stronger effects on estimates of correlation; the non-linear decay
function used in that experiment produced higher levels of correction
and resulted in highly accurate correlation estimates (see Figure
\ref{fig-estimates-by-r-e3}, \chap{chap:adjusting_size}). In
\chap{chap:interactions_opacity_size}, these point opacity and size
manipulations were combined. Their combination was found to produce
stronger effects than would be expected if they were linearly additive.
While my efforts at correcting for the underestimation bias have seen
success, my work has not yet attempted to investigate whether any of the
techniques developed in this thesis may be used to change people's
cognitions about data. Therefore, for my final experimental chapter, I
show that scatterplot manipulations that are able to correct for a
historical correlation underestimation bias are also able to induce
stronger levels of belief change in viewers compared to conventional
plots showing identical data. Through a pre-study and main experiment, I
provide evidence that adjusting visual features in scatterplots can go
beyond simple perceptual effects to influence beliefs about information
from trusted news sources.

\section{Introduction}\label{intro-beliefs}

Research consistently finds that the correlation displayed in positively
correlated scatterplots is underestimated
\cite{strahan_1978,bobko_1979, cleveland_1982,
collyer_1990, lane_1985, lauer_1989, meyer_1992, rensink_2017}. This
underestimation is particularly pronounced for Pearson's \emph{r} values
of 0.2 \textless{} \emph{r} \textless{} 0.6, and has been replicated
extensively in the current thesis. If scatterplots were solely used for
communication between experts, then the presence of this bias would not
be especially problematic; those trained in statistics and data
visualisation are more likely to be aware of, and make allowances for,
their biases. Unfortunately, this is not the case; lay people are
expected to be able to use and interpret data visualisations on an
almost daily basis. It is therefore the duty of those who design
visualisations to design with the naive, inexperienced viewer in mind.
Doing so requires us to understand \emph{how} visualisations work, and
to gain an appreciation for the hidden processes that allow pictorial
representations to convey more than words and numbers alone ever could.

In this thesis, \chap{chap:adjusting_opacity},
\chap{chap:adjusting_size}, and \chap{chap:interactions_opacity_size}
demonstrate how changing the opacities and sizes of points in
scatterplots is able to significantly alter participants' estimates of
correlation in positively correlated scatterplots. Substantial progress
has been made in correcting for the underestimation bias, however these
efforts have only provided evidence about perceptual effects using a
simple direct estimation paradigm. While successful, this work has not
yet investigated whether, and to what extent, these techniques can
influence cognition in the context of real-word data visualisations and
the relatedness between variables.

Visualisation is a powerful tool. After all, if numerical data were
sufficient for understanding, there would be no need to visualise beyond
aesthetic preference. Pattern recognition, attention, and familiarity
are aspects of human perception and cognition that can be exploited by
visualisation designers to facilitate more efficient, enjoyable, and
effective communication \cite{franconeri_2021}. This, however, is a
double-edged sword; poor design, be it malevolent or misguided, can
cause distrust, confusion, and misunderstanding amongst viewers. It is
for these reasons that belief change in scatterplots as a consequence of
alternative designs is the next logical research direction for this
project. Scatterplots, like many other data visualisations, have been
submitted as evidence in court cases \cite{bobko_1979}, and play key
roles in organisational decision-making, including in healthcare
\cite{poly_2019}. It is reasonable to assume that data visualisations
are used to make decisions that result in positive or negative outcomes
with regard to health and policy more generally, especially given
findings that in certain contexts, they are more persuasive than textual
information \cite{pandey_2014}. Studying the potential for new designs
to alter beliefs about relatedness facilitates better visualisation
techniques, but also effects an understanding about how these designs
might be used by malevolent actors with a view to inoculating those who
engage with them. To this end, I present a two-experiment study. First,
crowdsourcing is used to select part of the experimental stimuli. The
propensity for previously established alternative scatterplot designs to
alter beliefs about relatedness is then tested, taking into account the
emotional content of the statement and the graph literacy and defensive
confidence of participants.

\section{Related Work}\label{related-work-beliefs}

\subsection{Scatterplots: Developments in This
Thesis}\label{scatterplots-developments-in-this-thesis}

In contradiction to previous work \cite{rensink_2014, rensink_2017},
this thesis has found clear and powerful effects of systematically
changing the opacities and sizes of scatterplot points on participants'
estimates of correlation in positively correlated scatterplots. While
\chap{chap:adjusting_size}, in which point sizes were lowered as a
function of residual magnitude, provided the best level of correction
seen so far, \chap{chap:interactions_opacity_size} featured the most
dramatic level of correction. In that particular condition, both point
opacity and size were lowered as a function of residual magnitude using
equation 7.1:

\begin{equation}
  point_{size/opacity} = 1 - b^{residual}
\end{equation}

\begin{figure}[H]

\centering{

\includegraphics[width=\textwidth]{7_belief_change_files/figure-latex/fig-previous-manipulations-1.pdf}

}

\caption{\label{fig-previous-manipulations}Top row: Examples of
scatterplot manipulations from previous work using an \textit{r} value
of 0.6. Bottom row: the corresponding correlation estimation behaviour
across values of \textit{r} between 0.2 and 0.99. The dashed diagonal
line represents hypothetically accurate estimation, while the solid line
is what is observed when participants are asked to estimate
correlation.}

\end{figure}%

Figure~\ref{fig-previous-manipulations} contains a summary of the point
opacity and size manipulations from the previous three chapters in this
thesis, along with their effects on performance on a correlation
estimation task. Each manipulation specified employs the labelled
non-linear decay function(s). The present experiment investigates the
potential for alternative scatterplot designs to have effects on
cognition. For this reason, and to facilitate comparison to the work
carried out in Experiments 1 to 4, the same design protocols have been
utilised here, including the number of points (\emph{n} = 128), the
value of \emph{b} (0.25), and the size scaling factor, additional
constant, and opacity floor. For the alternative scatterplot condition
here, the manipulation which has been previously shown to cause the most
dramatic change in participants' estimates of correlation was used: the
combination of non-linear opacity and size decay functions described in
\chap{chap:interactions_opacity_size} (see the right-hand pair of plots
in Figure~\ref{fig-previous-manipulations}).

\subsection{Perception \& Cognition in Data
Visualisation}\label{perception-cognition-in-data-visualisation}

Interacting with data visualisation is a complex process involving
bottom-up and top-down mechanisms
\cite{shah_2011, franconeri_2021, xiong_2023}. My previous work
investigating alternative scatterplot designs has focused on perceptual
factors and mechanisms; here the potential for top-down effects to bias
participants is introduced. Recent work has established that
scatterplots are able to induce different levels of belief change in
viewers \cite{karduni_2021, markant_2023}; this may depend on factors
such as prior belief strength, attitudes, and the presence or absence of
uncertainty visualisations. Accordingly, the pre-study is essential for
isolating the variable of interest, the alternative scatterplot design.
Data visualisation does not take place without context, and so the
investigation of top-down effects is critical for providing designers
with the tools to design visualisations that work as intended in the
field. To this end, I present a two-experiment study investigating the
propensity for established scatterplot visualisation techniques to bias
participants' beliefs about the levels of relatedness between variables.

\section{Open Research}\label{open-research-chap7}

All experiments in this chapter were conducted according to the
principles of open and reproducible research \cite{ayris_2018}. All
experimental code, materials, and instructions are hosted on GitLab for
the pretest\footnote{https://gitlab.pavlovia.org/Strain/beliefs\_scatterplots\_pretest}
and for the main test as a pair of separate
experiments\footnote{https://gitlab.pavlovia.org/Strain/atypical\_scatterplots\_main\_t}\textsuperscript{,}\footnote{https://gitlab.pavlovia.org/Strain/atypical\_scatterplots\_main\_a}.
The original paper is maintained as a GitHub repository\footnote{https://github.com/gjpstrain/beliefs\_alternative\_scatterplots}.
This repository contains all code, analysis, and visualisation. The
repository also contains instructions for building a Docker container to
recreate the computational environment the paper was written in.
Pre-registrations for both the pre-test\footnote{https://osf.io/xuf4d}
and the main experiment\footnote{https://osf.io/anmez} are hosted on the
Open Science Framework (OSF), and any deviations are noted where
relevant in this chapter.

\section{Pre-Study: Investigating Beliefs About Relatedness
Statements}\label{beliefs-e5a}

The goal of the present study is to investigate the extent to which a
novel scatterplot design, incorporating the findings of
\chap{chap:adjusting_opacity}, \chap{chap:adjusting_size}, and
\chap{chap:interactions_opacity_size}, is able to alter participants'
beliefs about correlations. Due to the targeting of lay populations, and
my previous experience with lay participants failing to understand the
term ``correlation'' (although not the concept), I elected to
operationalise correlation as ``strength of relatedness''. There is
evidence that belief change can be affected by prior beliefs and
attitudes \cite{markant_2023, xiong_2023}, and that emotion, including
the content of a visualisation \cite{phelps_2006, harrison_2013} and the
emotional state of a participant \cite{thoresen_2016} can have
perceptual effects on participants and their performance. I was unable
to find resources for correlative statements that included ratings for
belief strength and emotional valence, so elected to create my own. To
control for these factors as much as possible, the pre-study was run
with the intent of finding a correlative statement that was matched on
emotional valence and level of belief strength. As opposed to manually
creating a list of candidate statements, I used the ChatGPT4 Large
Language Model (LLM) \cite{chatGPT}. On the 9th April, 2024, ChatGPT was
asked:

\begin{quotation}
    ``Generate 100 statements that describe the correlation between two variables, such as:
     ``X is associated with a higher level of Y" or
     ``As X increases, Y increases".
    Try to match all the statements on emotionality.''
\end{quotation}

The full list of statements can be found in Appendix
\ref{chap:appendix_A}. A co-author and I rated each statement on
emotional valence and belief about strength of relatedness using Likert
scales from 1 to 7. Both statement emotional valence and strength of
relatedness were anchored at points 1 and 7: \emph{Very Negative} and
\emph{Very Positive} for the former, and \emph{Not Related At All} and
\emph{Strongly Related} for the latter. All other points were
unlabelled. The \texttt{irr} (version 0.84.1 \cite{irr}) package was
used to calculate a quadratic weighted Cohen's Kappa between the raters,
which penalises larger magnitude disagreements more harshly. Following
each statement, a pair of Likert scales were presented labelled
``Statement Emotionality'' and ``Strength of Relatedness''. We agreed
above chance for both emotional valence (\(\kappa\) = 0.49, \emph{p}
\textless{} .001) and strength of relatedness (\(\kappa\) = 0.51,
\emph{p} \textless{} .001), indicating moderate levels of agreement in
both cases \cite{cohen_1968, fleiss_1969}. Together, we selected
strongly and weakly correlated statements with the highest levels of
absolute agreement, resulting in 14 strongly and 11 weakly correlated
statements. These statements are reproduced in
Table~\ref{tbl-all-statements}.

\begin{table}

\caption{\label{tbl-all-statements}Statements tested with participants
in the pre-study.}

\centering{

\begin{tabular}[t]{rl}
\toprule
Item & Statement\\
\midrule
1 & Increased exposure to sunlight is correlated with higher vitamin D levels.\\
2 & As caffeine consumption increases, so does the average heart rate.\\
3 & Greater frequency of exercise is linked to a lower risk of depression.\\
4 & Greater use of helmets is associated with a lower incidence of head injuries in cyclists.\\
5 & As the quality of healthcare improves, life expectancy tends to increase.\\
\addlinespace
6 & As access to clean water improves, the incidence of waterborne diseases decreases.\\
7 & Higher levels of empathy are linked to stronger interpersonal relationships.\\
8 & As soil quality degrades, agricultural productivity tends to decrease.\\
9 & Higher levels of civic engagement are linked to a stronger sense of community.\\
10 & Higher sugar consumption is associated with an increased risk of dental cavities.\\
\addlinespace
11 & Higher attendance at preventive health screenings is linked to earlier detection of diseases.\\
12 & Increased use of energy-efficient appliances is associated with lower electricity bills.\\
13 & As pedestrian-friendly infrastructure improves, urban walkability tends to increase.\\
14 & Greater regularity in sleep patterns is associated with improved mental health.\\
15 & Greater water consumption is linked to improved kidney function.\\
\addlinespace
16 & As the amount of sleep decreases, the risk of obesity increases.\\
17 & Greater intake of omega-3 fatty acids is associated with lower inflammation levels.\\
18 & Greater exposure to music education is linked to enhanced cognitive development in children.\\
19 & Higher exposure to air conditioning is associated with increased respiratory issues.\\
20 & Higher frequency of family meals is linked to better eating habits in children.\\
\addlinespace
21 & As participation in community arts programs increases, local cultural engagement tends to rise.\\
22 & Higher consumption of spicy foods is associated with a lower risk of certain types of cancer.\\
23 & Greater adherence to a Mediterranean diet is linked to a lower risk of neurodegenerative diseases.\\
24 & Higher consumption of nuts and seeds is associated with reduced risk of cardiovascular diseases.\\
25 & As cultural preservation efforts increase, community identity and cohesion tend to strengthen.\\
\bottomrule
\end{tabular}

}

\end{table}%

\subsection{Hypotheses}\label{hypotheses-e5a}

In an attempt to control for the potential effects of belief about
strength of relatedness and emotional valence in the main study, the 25
candidate statements selected by myself and a co-author were then tested
with a representative UK population in order to ascertain consensus. It
was hypothesised that:

\begin{itemize}
\tightlist
\item
  H1: there will be a significant difference in the mean ratings of
  emotional valence between statements.\footnote{The pre-registration
    for this hypothesis refers to ``emotionality'', as did an earlier
    draft of the paper corresponding to this chapter. In response to
    reviewer comments, I clarify that it is really emotional valence
    that is being tested, and therefore this wording is used here.}
\item
  H2: there will be a significant difference in the mean ratings of
  strength of relatedness between statements.
\end{itemize}

\subsection{Method}\label{method-e5a}

\subsubsection{Design}\label{design-e5a}

The pre-study featured a within-subjects design. Each participant saw
all 25 survey items (see Table~\ref{tbl-all-statements}), along with the
six attention check items, in a fully randomised order.

\subsubsection{Procedure}\label{procedure-e5a}

Ethical approval for this experiment was granted by the University of
Manchester's Computer Science Departmental Panel (Ref:
2024-19426-33939). Participants viewed the PIS and were asked to provide
consent through key presses in response to consent statements. They were
prompted to provide their age in a free text box and their gender
identity. Participants were told that they would be asked to read
statements about the relatedness between a pair of variables, after
which they would be asked to answer some questions. To familiarise
themselves with the sliders used to collect responses, they were asked
to complete a practice trial in response to the statement: ``As
participation in online experiments increases, society becomes
happier''.

\subsubsection{Participants}\label{participants-e5a}

100 participants were recruited using the Prolific platform
\cite{prolific}. English fluency and UK residency were required for
participation, as the main experiment relied on familiarity with data
visualisations from a popular British news source. In addition to 25
experimental items, six attention check items were included that
instructed participants to ignore the statement and provide specific
answers to the Likert scale sliders. No participants failed more than 2
out of 6 attention check items, and therefore data from all 100 were
included in the full analysis (52 male and 48 female). Participants'
mean age was 41.1 (\emph{SD} = 12.3). The mean time taken to complete
the survey was 14.2 minutes (\emph{SD} = 2.9 minutes).

\subsection{Results}\label{results-e5a}

As before, the \texttt{irr} package (version 0.84.1 \cite{irr}) was used
to measure interrater agreement on statement emotional valence and
strength of relatedness for the 25 experimental items. This analysis
revealed that participants agreed above chance on statement emotional
valence (\(\kappa\) = 0.07, \emph{p} \textless{} .001) and strength of
relatedness (\(\kappa\) = 0.06, \emph{p} \textless{} .001).

\subsection{Selecting Statements for the Main
Experiment}\label{selecting-statements-e5a}

\begin{table}

\caption{\label{tbl-candidate-statements}Statements with neutral mean
emotional valence ratings.}

\centering{

\begin{tabular}[t]{rl}
\toprule
Item & Statement\\
\midrule
2 & As caffeine consumption increases, so does the average heart rate.\\
10 & Higher sugar consumption is associated with an increased risk of dental cavities.\\
16 & As the amount of sleep decreases, the risk of obesity increases.\\
22 & Higher consumption of spicy foods is associated with a lower risk of certain types of cancer.\\
23 & Greater adherence to a Mediterranean diet is linked to a lower risk of neurodegenerative diseases.\\
\bottomrule
\end{tabular}

}

\end{table}%

Statements representing neutral emotional valence were selected to
control for the potential effects of statement emotionality in the main
experiment. Statements with mean emotionality ratings between 3 and 5
are statements 2, 10, 22, 16, and 23, which can be seen in
Table~\ref{tbl-candidate-statements}. To ascertain which statements
represent the greatest consensus, standard deviations of ratings for
statement emotional valence and strength of relatedness were summed. Due
to concerns about experimental power, and in line with evidence that
propensity for belief change is highest when prior beliefs are not
strongly held \cite{xiong_2023, markant_2023}, at this point the
decision was made to only test a statement corresponding to weak beliefs
about the strength of relatedness between the variables in question.
Statement number 22 was therefore selected: ``Higher consumption of
spicy foods is associated with a lower risk of certain types of
cancer'', however the wording was modified such that the variables (food
consumption and cancer risk) were positively correlated. While the work
carried out in \chap{chap:adjusting_opacity},
\chap{chap:adjusting_size}, and \chap{chap:interactions_opacity_size}
show that opacity and size adjustments in scatterplots can affect
estimates of positive correlation, no work regarding the effects of
these manipulations in negatively correlated scatterplots has been
completed.

\subsection{Discussion}\label{discussion-e5a}

Fleiss' Kappa values for interrater agreement on both statement
emotional valence and strength of relatedness scales are low (\(\kappa\)
= 0.07 and \(\kappa\) = 0.06 respectively), however do exceed that which
would be expected by chance. In light of this, decisions regarding which
statement to use are not based on the values of Fleiss' Kappa observed,
but rather on the standard deviations of ratings across all raters.
Statement emotionality and strength of relatedness are tested with
participants in the main study and included as fixed effects as part of
the analyses.

\section{Main Experiment: Alternative Scatterplot Designs and
Beliefs}\label{beliefs-main-e5}

The statement selected exhibits the lowest mean level of belief about
strength of relatedness and the 2\textsuperscript{nd} highest level of
consensus. Modified for directionality, the statement reads:

\begin{quotation}
``Higher consumption of plain (non-spicy) foods is associated with a higher risk of certain types of cancer.''
\end{quotation}

To maximise the likelihood of finding an effect of viewing alternative
scatterplots, the stimuli were designed based on a popular British news
source and the data were falsely credited as being supplied by the
British National Health Service (NHS). Participants were told that the
news source had requested their identity be obscured. They were
debriefed that this was not the case, and in fact the data were false,
at the end of the experiment. Based on evidence that beliefs can change
after viewing visualisations \cite{karduni_2021, markant_2023}, and that
scatterplots employing point opacity and size manipulations described in
Section \ref{related-work-beliefs} are able to affect perceptual
estimates, the following hypotheses were made.

\subsection{Hypotheses}\label{hypotheses-e5}

\begin{itemize}
\tightlist
\item
  H1: there will be a significant difference between ratings of strength
  of relatedness made before and after participants viewed scatterplots
  in either the standard or alternative conditions.
\item
  H2: this difference will be greatest when participants are exposed to
  scatterplots in the alternative scatterplot condition.
\end{itemize}

Exploratory investigations also took place taking into account
participants' scores on a defensive confidence test, their scores on a
graph literacy test, and each participant's rating of the emotional
valence of the correlative statement used. Analysis including each of
these factors can be found in Section \ref{add-analyses-e5}, and their
inclusion is justified below.

\subsubsection{Defensive Confidence}\label{def-con-e5}

In line with evidence that those who are more confident in their ability
to defend their own positions are more susceptible to having those
positions changed \cite{albarracin_2004}, participants' defensive
confidence was measured using Albarracín and Mitchell's
\cite{albarracin_2004} 12-item scale. This scale is replicated from
previous work in Table~\ref{tbl-def-con-scale}, and has been utilised
more recently \cite{markant_2023} to explore the potential for attitude
change specifically with regard to correlations in scatterplots.
Participants provide answers to the 12 scale items using a 5-point
Likert scale anchored at points 1 (\emph{not at all characteristic of
me}) and 5 (\emph{extremely characteristic of me}), with all other
points being unlabelled.

\begin{table}

\caption{\label{tbl-def-con-scale}The 12-item Defensive Confidence scale
from using Albarracín and Mitchell \cite{albarracin_2004}}

\centering{

\begin{tabular}[t]{r>{\raggedright\arraybackslash}p{40em}}
\toprule
Item & Statement\\
\midrule
1 & During discussions of issues I care about I can successfully defend my ideas.\\
2 & I have many resources to defend my point of view when I feel my ideas are under attack.\\
3 & When I pay attention to the arguments proposed by people who disagree with me I feel confused and cannot think. (reverse-scored)\\
4 & When trying to defend my point of view I am not at all articulate. (reverse-scored)\\
5 & I have developed ways of ’winning’ when I debate issues I care about.\\
\addlinespace
6 & I could stand by my ideas in front of anybody.\\
7 & No matter what I read or hear I am always capable of defending my feelings and opinions.\\
8 & I think of myself as somebody who has enough information to defend his or her points of view.\\
9 & Compared to most people, I am able to maintain my own opinions regardless of what conflicting information I receive.\\
10 & Compared to people I know who are very successful at maintaining their point of view, I have somewhat weak, underdeveloped opinions. (reverse-scored)\\
\addlinespace
11 & I can defend my points of view when I want to.\\
12 & I am unable to defend my own opinions successfully. (reverse-scored)\\
\bottomrule
\end{tabular}

}

\end{table}%

\subsubsection{Graph Literacy}\label{graph-literacy-e5}

No effect of graph literacy was found in Experiments 1 to 4 (see
\chap{chap:adjusting_opacity}, \chap{chap:adjusting_size}, and
\chap{chap:interactions_opacity_size}). Despite this, the scale was
included here due to the higher predicted cognitive load of the current
task, as there is evidence that graph literacy may affect performance on
more cognitively demanding visualisation tasks
\cite{canham_2010, okan_2012}. Additionally, the graph literacy test
used \cite{garcia_2016} is extremely short; in the present study, this
took participants an average of 27 seconds (\emph{SD} = 16 seconds).

\subsubsection{Emotionality}\label{emotionality}

The emotional content of a visualisation and the emotional state of a
participant may have cognitive and perceptual effects on performance in
visualisation tasks \cite{phelps_2006, harrison_2013, thoresen_2016};
this was the primary motivation behind performing the pre-study.
Nevertheless, it is not guaranteed that each participant considers the
emotional content of the correlative statement to be the same. To
account for these individual differences, ratings of emotional valence
are also collected during the main study.

\subsection{Method}\label{method-e5}

\subsubsection{Stimuli}\label{stimuli-e5}

\begin{figure}

\centering{

\includegraphics[width=\textwidth]{7_belief_change_files/figure-latex/fig-exp5-examples-chap7-1.pdf}

}

\caption{\label{fig-exp5-examples-chap7}Examples of the experimental
stimuli for Experiment 5 using an \textit{r} value of 0.6. Group A saw
the alternative scatterplot presented on the left, while group B saw the
standard design on the right. The labels below the plots are included
for the reader's convenience, and were not a part of the experimental
stimuli.}

\end{figure}%

After selecting a correlative statement describing a weak relationship
and with a high level of consensus between participants, the
\texttt{ggplot2} package (version 3.5.1) \cite{wickham_2016} was used to
create the stimuli for the main experiment. As the statement was rated
as describing a low level of relatedness, scatterplots describing a
strong relationship (0.6 \textless{} \emph{r} \textless{} 0.99) were
used with the intent of inducing belief change. Plots in the alternative
scatterplot condition were created using a combination of non-linear
opacity and size decay, as this particular condition was shown to bias
correlation estimates to a greater degree than either point opacity or
size decay alone (see Experiment 4 in
\chap{chap:interactions_opacity_size}). 45 \emph{r} values uniformly
distributed between 0.6 and 0.99 were used to create 45 scatterplots for
each condition. Examples of stimuli using an \emph{r} value of 0.6 for
both the standard and alternative scatterplot conditions can be seen in
Figure~\ref{fig-exp5-examples-chap7}.

\subsubsection{Design}\label{design-e5}

Unlike all previous experiments, a between-participants design was
employed here. Each participant was randomly assigned either to group A,
in which they viewed alternative scatterplots designed deliberately to
elicit higher levels of belief change, or group B, in which they viewed
standard scatterplots. Participants saw all 45 experimental items for
their group, along with 4 attention check items, in a fully randomised
order. The dependent variable was the level of belief change induced by
viewing the scatterplot visualisations, so participants were tested on
how strongly related they believed the variables described by the
correlative statement were both \textbf{before} and \textbf{after}
viewing the experimental items.

\subsubsection{Procedure}\label{procedure-e5}

Ethical approval for this experiment was granted by the University of
Manchester's Computer Science Departmental Panel (Ref:
2024-19426-33939). Participants viewed the PIS and provided consent
through key presses in response to displayed consent statements.
Participants were then asked to provide their age and gender identity.
Following this, participants completed the 5-item Subjective Graph
Literacy scale as in previous experiments \cite{garcia_2016}, and
Albarracín and Mitchell's \cite{albarracin_2004} 12-item defensive
confidence scale. To give legitimacy to the data visualisations with the
hope of maximising any potential belief change, participants were told
that the graphs they would see were taken from a well-known British news
source, but that the identity of this source had been obscured at their
request. To promote engagement with the visualisations, participants
were instructed to use a slider to estimate the correlation displayed in
each scatterplot; no hypotheses were made based on these data, and
therefore they were not analysed further. Following instructions, which
included textual descriptions of scatterplots and Pearson's \emph{r},
participants were given two practice trials; these trials took the form
of a full opacity trial from Experiment 1. Participants were then asked
to indicate their beliefs about emotional valence and strength of
relatedness described in the chosen correlative statement; these data
were captured using Likert scales identical to those described
previously. After completing 45 experimental trials, participants were
then asked again, using the same Likert scales, to indicate their belief
about the strength of relatedness described in the correlative
statement. Interspersed among the experimental items were four attention
check trials which explicitly asked participants to set the slider to 0
or 1.

\subsubsection{Participants}\label{participants-e5}

150 participants were recruited using the Prolific platform
\cite{prolific}. Normal or corrected-to-normal vision and English
fluency were required. As in the pre-study, UK residency was required of
participants, as the experiment relied on familiarity with the visual
style of a British news source. Participants who took part in the
pre-study, or in any of the experiments described in
\chap{chap:adjusting_opacity}, \chap{chap:adjusting_size}, or
\chap{chap:interactions_opacity_size} were prevented from completing
this experiment. Data were collected from 77 participants in each
condition. 2 participants failed more than 2 out of 4 attention check
questions for each condition, meaning their data were excluded per
pre-registration stipulations. Data from the remaining 150 participants
were included in the full analysis (73 male, 73 female, and 4
non-binary). Participants' mean age was 39.3 (\emph{SD} = 11.5).
Participants' mean graph literacy score was 21.3 (\emph{SD} = 4.3) out
of 30, their mean defensive confidence score was 43 (\emph{SD} = 6.8)
out of 60, and their mean rating of statement emotional valence was 2.9
(\emph{SD} = 1.3) on a 7-point Likert scale. On average, participants
took 14.2 minutes to complete the experiment (\emph{SD} = 6.41).

\subsection{Results}\label{results-e5}

Likert scales capture whether one rating is higher or lower than
another, however they do not quantify the differences between levels of
rating. Metric modelling assuming equal levels of difference between
ratings, such as linear regression, is therefore inappropriate
\cite{liddell_2018}. In light of this, the \texttt{ordinal} package
(version 2023.12-4.1 \cite{ordinal}) was used to build cumulative link
mixed effects models to analyse Likert scale data\footnote{The linked
  pre-registration (see Section \ref{open-research-chap7}) specifies
  linear mixed effects models. This was an oversight; conclusions are
  identical when using said models.}. As in previous chapters, the
\texttt{buildmer} (version 2.12 \cite{buildmer}) package is used to
automate the selection of the random effects structure (see Section
\ref{model-construction} for further details). Odds ratios and
equivalent Cohen's \emph{d} effect size values are calculated using the
\texttt{effectsize} package (version 1.0.1 \cite{effectsize}).

To test the first hypothesis, that ratings of strength of relatedness
would be different before and after participants viewed experimental
items, a model is built whereby the rating of strength of relatedness
the participant made is predicted by whether it was made \textbf{before}
or \textbf{after} viewing the experimental items. The first hypothesis
was supported; there was a significant difference in ratings of strength
of relatedness made before and after participants viewed the
experimental plots. A likelihood ratio test revealed that the model
including time of rating as a predictor explained significantly more
variance than the null (\(\chi^2\)(1) = 8,046.95, \emph{p} \textless{}
.001). This model has random intercepts for participants. Statistical
testing providing support for this hypothesis is shown in
Table~\ref{tbl-rating-time}. Figure~\ref{fig-descriptives-e5} shows
means and dot plots for ratings of strength of relatedness made before
and after viewing scatterplots in either the standard or alternative
condition.

\begin{table}

\caption{\label{tbl-rating-time}Statistics for the significant main
effect of rating time. Odds ratio and the equivalent Cohen's \textit{d}
value is also supplied.}

\centering{

\begin{tabular}[t]{lrrrlrr}
\toprule
  & Estimate & Standard Error & Z-value & \textit{p} & Odds Ratio & Cohen's \textit{d}\\
\midrule
Rating Time & 3.77 & 0.049 & 76.63 & <0.001 & 43.2 & 2.08\\
\bottomrule
\end{tabular}

}

\end{table}%

\begin{table}

\caption{\label{tbl-condition-interact}Statistics for the significant
main effect of rating time and the significant interaction between
rating time and condition on the difference between pre- and
post-scatterplot viewing ratings for standard and alternative plots.
Odds ratios and equivalent Cohen's \emph{d} effect sizes are also
shown.}

\centering{

\begin{tabular}[t]{lrrrlrr}
\toprule
  & Estimate & Standard Error & Z-value & \textit{p} & Odds Ratio & Cohen's \textit{d}\\
\midrule
Rating Time & 4.15 & 0.063 & 66.34 & <0.001 & 63.33 & 2.29\\
Condition & 0.49 & 0.390 & 1.25 & 0.211 & 1.63 & 0.27\\
Rating Time $x$ Condition & -0.72 & 0.071 & -10.22 & <0.001 & 2.06 & 0.40\\
\bottomrule
\end{tabular}

}

\end{table}%

\begin{figure}

\centering{

\includegraphics[width=\textwidth]{7_belief_change_files/figure-latex/fig-descriptives-e5-1.pdf}

}

\caption{\label{fig-descriptives-e5}Dot plots for pre- and post-plot
viewing ratings of strength of relatedness for standard and alternative
scatterplot conditions. Mean ratings are also shown as points.}

\end{figure}%

The second hypothesis, that the difference between ratings of strength
of relatedness made before and after participants viewed the
experimental plots would be greater when they were assigned to the
alternative scatterplot condition, also received support. Treatment
coding was used for each of the experimental factors of rating time
(pre- or post-) and scatterplot condition, which facilitates direct
comparisons between means of ratings made before and after plot viewing.
A cumulative link mixed effects model, whereby the rating of strength of
relatedness the participant made was predicted by the condition they
were assigned to \emph{and} the time they made the rating was built. A
likelihood ratio test revealed that the model including condition and
rating time as predictors explained significantly more variance than the
null (\(F\)(3) = 8,151.94, \emph{p} \textless{} .001). This model had
random intercepts for participants. There was a main effect of rating
time, no main effect of condition, and an interaction between rating
time and condition. Test statistics, along with odds ratios and
equivalent Cohen's \emph{d} effect sizes can be seen in
Table~\ref{tbl-condition-interact}. The estimate for the interaction
corresponds to the difference-in-difference between ratings made pre-
and post-viewing for standard and alternative scatterplots. This
difference-in-difference is visualised in
Figure~\ref{fig-difference-descriptive}; the difference in ratings
between pre- and post-plot viewing times is greater for the participants
who were exposed to the alternative scatterplot condition.

\begin{figure*}

\centering{

\includegraphics[width=\textwidth]{7_belief_change_files/figure-latex/fig-difference-descriptive-1.pdf}

}

\caption{\label{fig-difference-descriptive}Histograms illustrating the
magnitudes of the difference between pre- and post-plot viewing ratings
of strength of relatedness for standard and alternative scatterplots.
Median values are plotted as points.}

\end{figure*}%

\subsubsection{Additional Analyses}\label{add-analyses-e5}

Effects were also found of participants' scores on the defensive
confidence test (\(F\)(4) = 69.73, \emph{p} \textless{} .001),
participants' scores on the graph literacy test (\(F\)(4) = 42.66,
\emph{p} \textless{} .001), and of how emotionally valent participants
rated the chosen correlative statement before beginning the block of
trials (\(F\)(4) = 43.51, \emph{p} \textless{} .001). The interactions
between the main effect and graph literacy, defensive confidence, and
statement emotional valence are discussed in Section
\ref{add-analyses-discussion}.

\subsection{Discussion}\label{discussion-e5}

Both hypotheses were supported by the results of the main experiment.
Participants reliably updated their beliefs after viewing scatterplots,
and the difference between pre- and post-viewing beliefs was greater for
those participants who viewed scatterplots in the alternative condition.
These results suggest that the perceptual effects found in
\chap{chap:adjusting_opacity}, \chap{chap:adjusting_size}, and
\chap{chap:interactions_opacity_size} can be extended into a higher
level cognitive space to change people's beliefs about the strength of
relatedness between a pair of variables. These findings are encouraging
for data visualisation designers who wish to design scatterplots such
that correlation perception more closely matches the underlying
statistics, however further work is required before developing
guidelines for the use of alternative scatterplot designs with regards
to producing more persuasive visualisations.

\subsubsection{Graph Literacy, Defensive Confidence, and Statement
Emotional Valence}\label{add-analyses-discussion}

\begin{figure}

\centering{

\includegraphics[width=\textwidth]{7_belief_change_files/figure-latex/fig-add-analyses-plots-1.pdf}

}

\caption{\label{fig-add-analyses-plots}Illustrating how differences in
beliefs about strength of relatedness change as a function of
participants' scores on the graph literacy test (left), their scores on
the defensive confidence test (centre), and their ratings of statement
emotional valence (right). Locally smoothed curves with 95\% CI ribbons
are shown separately for standard and alternative scatterplot viewing
conditions. Lower ratings of Difference in Beliefs (\(y\) axis)
corresponds to lower levels of belief change between pre- and
post-scatterplot viewing times.}

\end{figure}%

Mean differences in pre- and post-plot viewing ratings of strength of
relatedness by Subjective Graph Literacy score can be seen in
Figure~\ref{fig-add-analyses-plots}. Generally, participants with higher
scores on a graph literacy test experienced smaller changes in their
ratings of strength of relatedness. This is in line with previous work
suggesting that those with higher levels of graph or visualisation
literacy show better performance in inference tasks related to
visualisations \cite{canham_2010}, are more capable of describing
effects that visualisations aim to communicate \cite{shah_2011}, and can
preferentially attend to relevant features of visualisations to a
greater degree \cite{okan_2016}, than those with lower levels of graph
literacy. In the present study, evidence is provided that those with
greater levels of graph literacy are \emph{less susceptible} to having
their beliefs changed by visualisations. The use of the alternative plot
manipulation largely removes this effect, suggesting that there is less
systematic reliance on graph literacy when participants are faced with
an unfamiliar data visualisation.

An opposing pattern of results is observed when examining the effects of
defensive confidence on participants' propensity for belief change.
Generally, participants with higher scores on the defensive confidence
test experienced greater levels of belief change. This is in line with
evidence that those who are more confident in their ability to defend
their own beliefs are more liable to having those beliefs changed in
light of evidence \cite{albarracin_2004}. This effect has previously
been explained as being due to those with a greater degree of confidence
in their own ability to defend their ideas engaging with information
with lower levels of attention to the fact it opposes their beliefs. The
present study provides additional evidence in favour of this phenomenon.
While the general pattern of results is expected based on previous work,
the interaction present between defensive confidence and scatterplot
condition is novel. Despite following the normal pattern of results for
low to moderate levels of defensive confidence, the relationship between
scatterplot type, defensive confidence, and belief change diverges past
\textasciitilde{} 36/60 on the defensive confidence scale. It is unclear
why the alternative scatterplot condition is associated with high levels
of belief change in those with high defensive confidence. Further work
would be required to explore the interaction between novel visualisation
types (such as the alternative scatterplot in the present experiment)
and defensive confidence.

The effect of statement emotional valence on belief change is also
illustrated in Figure~\ref{fig-add-analyses-plots}. There is a broad
research space regarding emotionality and data visualisation
\cite{lan_2024}, and it is clear from previous work that emotion may
affect perception, cognition, and behaviour
\cite{phelps_2006, harrison_2013, thoresen_2016} pertaining to data
visualisation. Harrison et al.~\cite{harrison_2013} found that
participants who were positively primed performed better on a low-level
visual judgement task compared to those who were negatively primed.
Comparison of this work to the current is difficult, as \emph{success}
is hard to define in the present experimental paradigm.

Further experimental work is required to provide more comprehensive
explanations for the interactive effects of graph literacy, defensive
confidence, and statement emotional valence as they pertain to belief
change after scatterplot viewing.

\section{General Discussion}\label{general-discussion-e5}

The most parsimonious explanation for the results observed in the
present study is as follows; things that \emph{look} more related will
be \emph{judged} as being more related, and are therefore \emph{more}
able to change beliefs about the levels of relatedness between
variables. Given the frequent real-world usage of scatterplots, and the
role of data visualisations in decision-making, it is particularly
important to test empirically whether perceptual effects may be extended
into a cognitive space to influence beliefs. Doing so is a necessary
step in broadening the data visualisation design space and bringing
novel designs closer towards use cases with the potential for real-world
consequences while maintaining a strong foundation of experimental
evidence. Having controlled as far as possible for factors such as
emotional content, the consensus on how related the variables in
question were, and the general design (bar the points themselves) of the
scatterplot, I can conclude with strong evidence that alternative
scatterplot design was responsible for increasing the level of belief
change amongst participants.

An alternative (although not competing) explanation for the results have
seen here comes from recent work on the incorporation of uncertainty
visualisations in scatterplots. Karduni et al.~\cite{karduni_2021} found
in 2021 that visualisations that encode uncertainty produce lower levels
of belief change compared to those that do not. The alternative
scatterplot design employed in the main experiment can be thought of as
masking some of the uncertainty inherent in a scatterplot point cloud by
reducing the salience of the most exterior points.

Previous work has provided support for the idea that it is the shape of
the point cloud, more specifically, the width of the probability
distribution it represents, that drives correlation perception in
scatterplots. If this mechanism were valid, the observed results would
be expected. These results are broadly consequential. For data
visualisation designers, they provide strong evidence that utilising the
alternative scatterplot designs described in this thesis can affect
beliefs about levels of relatedness without requiring the removal of
data. For researchers, these results pave the way for work in a number
of directions.

\section{Future Work}\label{future-work-e5}

Because alternative scatterplot designs have not been tested before with
regard to belief change, the current study was designed with the
intention of capturing effects, should they exist. To this end, the
design was simple; multiple correlative statements were not
investigated, nor was the propensity for strongly held beliefs to be
changed or the effect of topics with strong or polarised emotional
components. A simple, blunt measure of belief about relatedness was
utilised, and only one of the alternative scatterplot visualisations
described in this thesis was used. Each of these components deserves
study, and each is ripe for future work to investigate the contributions
of each factor to the effects have seen here.

Section \ref{add-analyses-discussion} describes the effects that graph
literacy, defensive confidence, and participants' ratings of the
emotional valence of the correlative statement have on the propensity
for belief change. Future work may wish to investigate these factors,
along with others that affect perceptions of correlation, such as
educational background or spatial abilities \cite{tandon_2024}. Xiong et
al.~\cite{xiong_2023} describe how correlation estimation may differ
according to the context the data are presented in; this could be
extended to instead investigate statements with differing emotional
contents and how alternative scatterplot designs might interact with
emotional valence. Similarly, selecting matched participant groups with
low or high graph literacy or defensive confidence would facilitate
understanding of how alternative designs may be employed to cater for
people with different levels of experience, or who differ in terms of
their faith in their own ideas and abilities.

Previous works investigating beliefs with regard to correlation
estimation have made distinctions between beliefs and attitudes
\cite{xiong_2023, markant_2023}. No such distinction was made here due
to the novel utilisation of alternative designs. Markant et
al.~\cite{markant_2023} found that while beliefs about correlations
changed in participants as a result of interaction with scatterplots,
attitudes did not. Future work may wish to investigate whether this
finding would persist with scatterplots utilising the alternative
designs described here. Finally, while changing perceptions, beliefs,
and attitudes are promising early steps, changing people's behaviours
would be the real test of the power of alternative visualisation
techniques; while this may be difficult to study, future work should
investigate whether what has been found here and throughout this thesis
may be used to induce behaviour change.

\section{Limitations}\label{limitations-e5}

The commitment to finding an effect, should one exist, is also the
biggest limitation of the present chapter. The exploratory nature of the
work means I cannot comment specifically on how different forms of size
and opacity manipulation in scatterplots may change beliefs in different
ways, although addressing this using the framework presented here would
be simple to accomplish. To date, there has been no qualitative work
performed on alternative scatterplot designs such as those utilised
here; it may be that any perceptual or cognitive benefits are outweighed
by distrust or unfamiliarity with novel designs. The dependent variable
in the main experiment was represented to participants as a simple,
blunt, 7-point Likert scale. While I argue that this is not particularly
problematic given that the intention was to find an effect, future work
may wish to use techniques that provide further scope for analysis, such
as the graphical elicitation method developed by Karduni et al.
\cite{karduni_2021, karduni_2023}.

\section{Conclusion}\label{conclusion-e5}

In a single, final experiment testing whether previously established
perceptual techniques could be extended into a cognitive space to
influence participants' beliefs about the level of relatedness between
variables, evidence is provided that using a combination of non-linear,
typical orientation point opacity and size decay functions is able to
change beliefs to a greater extent than data-identical scatterplots that
do not use these techniques. Scatterplots using these techniques were
presented as news items, and featured a variable pair that had been
selected by the target population as describing a weakly held,
emotionally neutral correlation. Participants who viewed such
scatterplots experienced greater levels of belief change compared to
participants who only viewed standard scatterplots. Additionally,
interaction effects were found of a number of participant
characteristics. These results suggest that visualisation techniques
that have previously been employed to improve perception amongst
participants are deserving of study with regards to their potential to
change beliefs.




\end{document}
