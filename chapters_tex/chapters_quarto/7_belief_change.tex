\documentclass[../main.tex]{subfiles}
\begin{document}


\section{Abstract}\label{abstract-beliefs}

\chap{chap:adjusting_opacity}, \chap{chap:adjusting_size}, and
\chap{chap:interactions_opacity_size} show through four experiments that
point opacity and size changes can have powerful effects on
participants' estimates of correlation in positively correlated
scatterplots. In \chap{chap:adjusting_opacity}, global and
spatially-dependent adjustments in point opacity were employed, and a
small, but statistically significant level of correction for a historic
underestimation bias of positive correlation was found.
Spatially-dependent adjustment of point size, in which size is reduced
as a function of residual error, was found in \chap{chap:adjusting_size}
to produce much stronger effects on estimates of correlation; the
non-linear decay function used in that experiment produced higher levels
of correction and resulted in highly accurate correlation estimates. In
\chap{chap:interactions_opacity_size}, these point opacity and size
manipulations were combined. Their combination was found to produce
stronger effects than would be expected if they were linearly additive.
While my efforts at correcting for the underestimation bias were
successful, my work has not yet attempted to investigate whether any of
the techniques developed in this thesis may be used to change people's
cognitions about data. Therefore, for my final experimental chapter, I
show that scatterplot manipulations that are able to correct for a
historic correlation underestimation bias are also able to induce
stronger levels of belief change in viewers compared to conventional
plots showing identical data. In a pre-study and main experiment, I
provide evidence that adjusting visual features in scatterplots can go
beyond simple perceptual effects to influence beliefs about information
from trusted news sources.

\section{Introduction}\label{intro-beliefs}

\section{Related Work}\label{related-work-beliefs}

\subsection{From Perception to
Cognition}\label{from-perception-to-cognition}

\subsection{From Cognition to Belief}\label{from-cognition-to-belief}

\section{Open Research}\label{open-research-chap7}

\section{Pre-Study: Investigating Beliefs About Relatedness
Statements}\label{beliefs-e5a}

\subsection{Introduction}\label{intro-e5a}

\subsection{Method}\label{method-e5a}

\subsubsection{Participants}\label{participants-e5a}

100 participants were recruited using the Prolific platform
\cite{prolific}. English fluency and UK residency were required for
participation, as the main experiment relied on familiarity with data
visualisations from a popular British news source. In addition to 25
experimental items, six attention check items were included that
instructed participants to ignore the statement and provide specific
answers to the Likert scale sliders. No participants failed more than 2
out of 6 attention check items, and therefore data from all 100 were
included in the full analysis (52 male and 48 female). Participants'
mean age was 41.1 (\emph{SD} = 12.3). The average time taken to complete
the survey was 7.6 minutes (\emph{SD} = 2.9 minutes).

\subsubsection{Design}\label{design-e5a}

Each participant saw all survey items (see Appendix A), along with the
six attention check items, in a fully randomised order. All experimental
code, materials, and instructions are hosted on GitLab \footnote{https://gitlab.pavlovia.org/Strain/beliefs\_scatterplots\_pretest}.

\subsubsection{Procedure}\label{procedure-e5a}

Participants viewed the PIS and were asked to provide through key
presses in response to consent statements. They were prompted to provide
their age in a free text box and their gender identity. Participants
were told that they would be asked to read statements about the
relatedness between a pair of variables, after which they would be asked
to answer some questions. To familiarise themselves with the sliders
used to collect responses, they were asked to complete a practice trial
in response to the statement: ``As participation in online experiments
increases, society becomes happier''. Following each statement, a pair
of Likert scales were presented labelled ``Statement Emotionality'' and
``Strength of Relatedness; these scales were identical to those used by
myself a co-author in Section \ref{beliefs-e5a}.

\subsection{Results}\label{results-e5a}

As before, the \texttt{irr} package \cite{irr} to measure interrater
agreement on statement emotional valence and strength of relatedness for
the 25 experimental items. This analysis revealed that participants
agreed above chance on statement emotional valence (\(\kappa\) = 0.07,
\emph{p} \textless{} .001) and strength of relatedness (\(\kappa\) =
0.06, \emph{p} \textless{} .001).

\subsection{Selecting Statements for the Main
Experiment}\label{selecting-statements-e5a}

\begin{table}

\caption{\label{tbl-candidate-statements}Statements with neutral average
emotional valence ratings.}

\centering{

\begin{tabular}[t]{rl}
\toprule
Item & Statement\\
\midrule
2 & As caffeine consumption increases, so does the average heart rate.\\
10 & Higher sugar consumption is associated with an increased risk of dental cavities.\\
16 & As the amount of sleep decreases, the risk of obesity increases.\\
22 & Higher consumption of spicy foods is associated with a lower risk of certain types of cancer.\\
23 & Greater adherence to a Mediterranean diet is linked to a lower risk of neurodegenerative diseases.\\
\bottomrule
\end{tabular}

}

\end{table}%

Statements represents neutral emotional valence were selected to control
for the potential effects of statement emotionality in the main
experiment. Statements with average emotionality ratings between 3 and 5
are statements 2, 10, 22, 16, and 23, which can be seen in
Table~\ref{tbl-candidate-statements}. To ascertain which statements
represent the greatest consensus, standard deviations of ratings for
statement emotional valence and strength of relatedness were summed. Due
to concerns about experimental power, and in line with evidence that
propensity for belief change is highest when prior beliefs are not
strongly held \cite{xiong_2022, markant_2023}, at this point the
decision was made to test only the statement corresponding to weak
beliefs about the strength of relatedness between the variables in
question. Statement number 22 was therefore selected: ``Higher
consumption of spicy foods is associated with a lower risk of certain
types of cancer'', however the wording was modified such that the
variables (food consumption and cancer risk) are positively correlated.
While the work carried out in \chap{chap:adjusting_opacity},
\chap{chap:adjusting_size}, and \chap{chap:interactions_opacity_size}
show that opacity and size adjustments in scatterplots can affect
estimates of positive correlation, no work regarding the effects of
these manipulations in negatively correlated scatterplots has been
completed.

\subsection{Discussion}\label{discussion-e5a}

Fleiss' Kappa values for interrater agreement on both statement
emotional valence and strength of correlation scales are low (\(\kappa\)
= 0.07 and \(\kappa\) = 0.06 respectively), however do exceed that which
would be expected by chance. In light of this, decisions regarding which
statement to use are not based on the values of Fleiss' Kappa observed,
but rather on the standard deviations of ratings across all raters.
Statement emotionality and strength of relatedness are tested with
participants in the main study and included as fixed effects as part of
the analyses.

\section{Main Experiment: Alternative Scatterplot Designs and Beliefs
about Relatedness}\label{beliefs-main-e5}

The statement selected exhibits the lowest average level of belief about
strength of relatedness and the 2\textsuperscript{nd} highest level of
consensus. Modified for directionality, the statement reads:

\begin{quotation}
    ``Higher consumption of plain (non-spicy) foods
is associated with a higher risk of certain types of cancer.''
    
\end{quotation}

To maximise the likelihood of finding an effect of viewing alternative
scatterplots, the stimuli were designed based on a popular British news
source and the data were falsely credited as being supplied by the
British National Health Service (NHS). Participants were told that the
news source had requested their identity be obscured. They were
debriefed that this was not the case, and in fact the data were false,
at the end of the experiment. Based on evidence that beliefs can change
after viewing visualisations \cite{karduni_2020, markant_2023}, and that
scatterplots employing point opacity and size manipulations described in
\ref{related-work-beliefs} are able to affect perceptual estimates, the
following hypotheses were made:

\subsection{Hypotheses}\label{hypotheses-e5}

\begin{itemize}
\tightlist
\item
  H1: there will be a significant difference between ratings of strength
  of relatedness made before and after participants viewed scatterplots
  in either the standard or alternative conditions.
\item
  H2: this difference will be greatest when participants are exposed to
  scatterplots in the alternative scatterplot condition.
\end{itemize}

Exploratory investigations taking into account participants' scores on a
defensive confidence test, their scores on a graph literacy test, and
each participant's rating of the emotional valence of the correlative
statement used. Analysis including each of these factors can be found in
Section \ref{add-analyses-e5}, and their inclusion is justified below.

\subsubsection{Defensive Confidence}\label{def-con-e5}

In line with evidence that those who are more confident in their ability
to defend their own positions are more susceptible to having those
positions changed \cite{albarracin_2004}, participant's defensive
confidence was measured using Albarracín and Mitchell's
\cite{albarracin_2004} 12-item scale. This scale is replicated from
previous work in Appendix B, and has been utilised more recently
\cite{markant_2023} to explore the potential for attitude change
specifically with regard to correlations in scatterplots. Participants
provide answers to the 12 scale items using a 5-point Likert scale
anchored at points 1 (\emph{not at all characteristic of me}) and 5
(\emph{extremely characteristic of me}), with all other points being
unlabelled.

\subsubsection{Graph Literacy
\{graph-literacy-e5\}}\label{graph-literacy-graph-literacy-e5}

No effect of graph literacy was found in experiments 1 to 4 (see
\chap{chap:adjusting_opacity}, \chap{chap:adjusting_size}, and
\chap{chap:interactions_opacity_size}). Despite this, the scale was
included here due to the higher predicted cognitive load of the current
tasks, along with evidence that graph literacy may affect performance on
more cognitively demanding visualisation tasks
\cite{canham_2010, okan_2012}. Additionally, the graph literacy test
used \cite{garcia_2016} is extremely short; in the present study, this
took participants an average of 27 seconds (\emph{SD} = 16 seconds).

\subsubsection{Emotionality}\label{emotionality}

The emotional content of a visualisation and the emotional state of a
participant may have cognitive and perceptual effects on performance in
visualisation tasks \cite{phelps_2006, harrison_2013, thoreson_2016};
this was the primary motivation behind performing the pre-study.
Nevertheless, it is not guaranteed that each participant considers the
emotional content of the correlative statement to be the same. To
account for these individual differences, ratings of emotional valence
are also collected during the main study.

\subsection{Method}\label{method-e5}

\subsubsection{Stimuli}\label{stimuli-e5}

\begin{figure}

\centering{

\includegraphics[width=\textwidth]{7_belief_change_files/figure-latex/fig-exp5-examples-chap7-1.pdf}

}

\caption{\label{fig-exp5-examples-chap7}Examples of the stimuli used in
experiment 4, demonstrated with an \textit{r} value of 0.6.}

\end{figure}%

\subsection{Results}\label{results-e5}

\subsubsection{Additional Analyses}\label{add-analyses-e5}

\subsection{Discussion}\label{discussion-e5}

\subsubsection{Graph Literacy, Defensive Confidence, and Statement
Emotional
Valence}\label{graph-literacy-defensive-confidence-and-statement-emotional-valence}

\section{General Discussion}\label{general-discussion-e5}

\section{Limitations}\label{limitations-e5}

\section{Future Work}\label{future-work-e5}




\end{document}
