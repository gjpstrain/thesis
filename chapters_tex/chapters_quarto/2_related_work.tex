\documentclass[../main.tex]{subfiles}
\begin{document}


\section{Data Visualisation: A Brief History}\label{brief-history}

Data visualisation, which can be thought of as the practice representing
information in a visual modality \cite{hinterberger_2009}, is difficult
to concretely define, classify, and categorise. With the primacy of
vision with regards to our interactions with and interpretations of the
world around us, data visualisation may be thought of as an extension of
art and the written word. Both art and writing are ancient phenomena,
with evidence for the former being found in the prehistoric period some
66,000 years ago \cite{standish_2025}, and evidence for the latter
emerging as Mesopotamian cuneiform around 3200 B.C.E
\cite{schmandt_2014}. Broadly, the literature agrees that art emerged
prior to the written word; this speaks volumes of the human instinct to
represent our thoughts, feelings, emotions, and that which we interact
with in the world around us pictorially.

When, then, should we consider to be the emergence of data visualisation
as a human practice? Of course, answering this question requires the
provision of a definition for the practice itself first.
Schmandt-Besserat \cite{schmandt_1978, schmandt_2014} considers clay
counting tokens to be the direct precursor of the written word. While
the evidence for this direct link is controversial, the existence of
such tokens is not. With each shape of token representing a certain
amount of a certain good (measures of grain, jars of oil, etc.), this
system could be considered a very early, very simple form of data
visualisation. Similarly, there is limited evidence of ancient
cartographic symbols \cite{muhly_1978}, which may also be considered a
form of, or related to, data visualisation. While I am not asserting
that data visualisation is older than writing, or that ancient map
drawings are forms of data visualisation, the existence of these
representations emphasises the attractive convenience that symbols and
signs represent for humans; making sense of our world and the
relationships therein is often easier through pictures as opposed to
words and numbers, a principle which I consider key for this thesis.

Note: much of the rest of this section is heavily inspired by Michael
Friendly's \emph{A Brief History of Data Visualization}
\cite{friendly_2008}. Moving on, then, to the kind of pictorial
representation that modern students and scientists would firmly
recognise as a ``data visualisation''. Tufte and Graves-Morris, in
1983's seminal \emph{The Visual Display of Quantitative Information}
\cite{tufte_1983}, describe an unattributed time series illustration
from the 10th or 11th century, itself described by Funkhouser in 1936
\cite{funkhouser_1936} as being discovered by Sigmund Günther in 1877.
This illustration is included here in
Figure~\ref{fig-early-time-series}.

\begin{figure}

\centering{

\includegraphics[width=\textwidth]{../supplied_graphics/tufte_1983.png}

}

\caption{\label{fig-early-time-series}Reproduced in Tufte and
Graves-Morris, 1983 \cite{tufte_1983} from Funkhouser, 1936
\cite{funkhouser_1936}\}}

\end{figure}%

\section{Measuring Relatedness}\label{measuring-relatedness}

\section{Conceptions of Correlation}\label{conceptions-of-correlation}

\section{Visualising Correlation}\label{visualising-correlation}

\subsection{History}\label{history-corr-viz}

\subsection{Present Landscape}\label{present-landscape-corr-viz}

\subsection{Scatterplots}\label{scatterplots-corr-viz}

\section{Correlation Perception}\label{corr-percept-related-work}

\section{Correlation Cognition}\label{corr-cognition}

\section{Underestimation: What's Really Going
On?}\label{underestimation-whats-going-on}

\section{Underestimation: Potential
Consequences}\label{underestimation-consequences}

\section{Data Visualisation Literacy}\label{graph-literacy-related-work}

\section{Objectives and Contributions}\label{objectives-contributions}




\end{document}
