\documentclass[../main.tex]{subfiles}
\begin{document}


\section{Data Visualisation: A Brief History}\label{brief-history}

Data visualisation, which can be thought of as the practice of
representing information in a visual modality \cite{hinterberger_2009},
is difficult to concretely define, classify, and categorise. With the
primacy of vision with regards to our interactions with and
interpretations of the world around us, data visualisation may be
thought of as an extension of art and the written word. Both art and
writing are ancient phenomena, with evidence for the former being found
in the prehistoric period some 66,000 years ago \cite{standish_2025},
and evidence for the latter emerging as Mesopotamian cuneiform around
3200 B.C.E \cite{schmandt_2014}. Broadly, the literature agrees that art
emerged prior to the written word; this speaks volumes of the human
instinct to represent our thoughts, feelings, emotions, and that which
we interact with in the world around us pictorially.

When, then, should we consider to be the emergence of data visualisation
as a human practice? Of course, answering this question requires the
provision of a definition for the practice itself first.
Schmandt-Besserat \cite{schmandt_1978, schmandt_2014} considers clay
counting tokens to be the direct precursor of the written word. While
the evidence for this link is controversial, the existence of such
tokens is not. With each shape of token representing a certain amount of
a certain good (measures of grain, jars of oil, etc.), this system could
be considered a very early, very simple form of data visualisation.
Similarly, there is limited evidence of prehistoric cartographic
drawings \cite{muhly_1978}, which may also be considered a form of, or
related to, data visualisation. While I am not asserting that data
visualisation is older than writing, or that ancient map drawings are
equivalent to data visualisation, the existence of these representations
emphasises the attractive convenience that symbols and signs represent
for humans; making sense of our world and the relationships therein is
often easier through pictures as opposed to words and numbers, a
principle which I consider key for this thesis.

Note: much of the rest of this section is heavily inspired by Michael
Friendly's \emph{A Brief History of Data Visualization}
\cite{friendly_2008}. Moving on, then, to the kind of pictorial
representation that modern students and scientists would firmly
recognise as a ``data visualisation''. Tufte and Graves-Morris, in
1983's seminal \emph{The Visual Display of Quantitative Information}
\cite{tufte_1983}, describe an unattributed time series illustration
from the 10th or 11th century, itself described by Funkhouser in 1936
\cite{funkhouser_1936} as being discovered by Sigmund Günther in 1877.
This illustration is included here in
Figure~\ref{fig-early-time-series}.

\begin{figure}

\centering{

\includegraphics[width=\textwidth]{../supplied_graphics/tufte_1983.png}

}

\caption{\label{fig-early-time-series}Reproduced in Tufte and
Graves-Morris, 1983 \cite{tufte_1983} from Funkhouser, 1936
\cite{funkhouser_1936}.}

\end{figure}%

This purports to show the movements of planetary bodies as a function of
time, although Funkhouser considered it little more than a ``schematic
diagram\ldots for illustrative purposes'' \cite{funkhouser_1936}.
Regardless, the recognisable grid lines and sinusoidal variation in the
curves are ideas that would not appear again for another 600-700 years,
after which they would become mainstays of visualisation. In the
mid-14th century, French philosopher Nicole Oresme demonstrated an
understanding of graphing by plotting proto-bar charts, and by the 16th
century, advances in cartography, photography, and mathematics laid the
ground for an explosion in data visualisation.

The 17th century saw the birth of geometry and coordinate systems, error
measurement, probability, and demographic statistics. With these
scientific advancements came the advancements in data visualisation
needed to communicate these concepts. For example, in 1626, Scheiner
used what Tufte would later term the ``principle of small multiples''
\cite{tufte_1983} to illustrate how configurations of sunspots change
over time (see Figure~\ref{fig-sunspots}).

\begin{figure}

\centering{

\includegraphics[width=\textwidth]{../supplied_graphics/sunspots.png}

}

\caption{\label{fig-sunspots}Reproduced in Tufte and Graves-Morris, 1983
\cite{tufte_1983} from Funkhouser, 1936 \cite{funkhouser_1936}.}

\end{figure}%

\begin{figure}

\centering{

\includegraphics[width=\textwidth]{../supplied_graphics/snow_cholera.jpg}

}

\caption{\label{fig-cholera}John Snow's (1854) map of cholera cases in
Soho, London. Using this data visualisation, Snow was able to
demonstrate a link between cholera cases and a contaminated water
supply.}

\end{figure}%

\begin{figure}

\centering{

\includegraphics[width=\textwidth]{../supplied_graphics/minard_napoleon.jpg}

}

\caption{\label{fig-napoleon}Charles Joseph Minard's (1869) flow diagram
of Napoleon's botched invasion of Russia in 1812-1813. This diagram
shows Napoleon's advance and retreat on Moscow. The width of the orange
and black columns encodes the size of the Grande Armée. The temperature
scale on the lower portion of the graph illustrates the weather
conditions during the retreat, with a freezing Russian winter causing
high rates of attrition.}

\end{figure}%

\begin{figure}

\centering{

\includegraphics[width=\textwidth]{../supplied_graphics/minard_napoleon.jpg}

}

\caption{\label{fig-nightingale}Florence Nightingale's (1858) polar area
chart illustrates the causes for mortality among British soldiers during
the Crimean War. Data visualisations of this type were used to
illustrate that in reality, more British soldiers died from preventable
disease than were killed by the enemy, and were used as part of a
campaign to improve sanitation among soldiers.}

\end{figure}%

The latter half of the 19th century, the so-called ``Golden Age of
Statistical Graphics'' \cite{friendly_2008} sees the rise of forms of
data visualisation that begin to look remarkably similar to the graphs
and informatics seen in mass media and scientific publications today.
The most notable examples of these are John Snow's cholera map, which
was able to link the incidence of cholera to a contaminated water pump
in London (Figure~\ref{fig-cholera}), Charles Joseph Minard's flow chart
of the Napoleonic invasion of Russia (Figure~\ref{fig-napoleon}), and
Florence Nightingale's rose diagrams (polar area charts in the modern
parlance, see Figure~\ref{fig-nightingale}). In each of these graphs,
visualisation is used with different intent. In John Snow's cholera map,
visualisation was used to track cases of a deadly disease, and
facilitated a novel linkage between cholera and contaminated drinking
water. In Charles Joseph Minard's flow chart of Napoleon's failed 1812
invasion of Russia, a total of six variables are displayed to tell the
data story, allowing the viewer to appreciate the movements of the
Grande Armée, it's diminishing size owing to attrition, and the freezing
temperatures that largely caused that attrition. In Florence
Nightingale's polar area chart depicting the causes of mortality amongst
British troops in the Crimean War; charts such as this were used to
successfully campaign for better sanitation in hospitals and the
frontlines.

In all of these visualisations, data is used to accentuate storytelling.
In some cases, this may lead to critical discoveries that save lives,
and in others, it may simply facilitate a greater understanding and
appreciation of the data. In either case, visualisation is used
effectively appeal to our affinity for visual storytelling.

In both, we see an encoding of variables that facilitates understanding
beyond prose or tabulated data; this is the magic of data visualisation
which has ensured its longevity and popularity among scientific
communicators, journalists, and advertisers alike.

\section{Measuring Relatedness}\label{measuring-relatedness}

\section{Conceptions of Correlation}\label{conceptions-of-correlation}

\section{Visualising Correlation}\label{visualising-correlation}

\subsection{History}\label{history-corr-viz}

\subsection{Present Landscape}\label{present-landscape-corr-viz}

\subsection{Scatterplots}\label{scatterplots-corr-viz}

\section{Correlation Perception}\label{corr-percept-related-work}

\section{Correlation Cognition}\label{corr-cognition}

\section{Underestimation: What's Really Going
On?}\label{underestimation-whats-going-on}

\section{Underestimation: Potential
Consequences}\label{underestimation-consequences}

\section{Data Visualisation Literacy}\label{graph-literacy-related-work}

\section{Objectives and Contributions}\label{objectives-contributions}




\end{document}
