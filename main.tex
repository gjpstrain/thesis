%%%%%%%%%%%%%%%%%% USAGE INSTRUCTIONS - from original template %%%%%%%%%%%%%%%%%%
% - Compile using LuaLaTeX and biber, unless there is a particular reason not to. Do not use the older LaTex/PDFLaTeX or BibTeX (the fonts won't work correctly.)
% - Font and the report 'year' must be specified when all \documentclass or the template won't work correctly. (There's no error checking/default cases!)
% - Options for fonts are: calibri, times, palatino, garamond, arial, tahoma, verdana, trebuchet. Note however that not all these are installed on Overleaf, you need to install to your project 
% - Options for 'year' are: first, second, thesis.
% - If not a thesis, should probably remove the COVID-19 impact statement page
% - As many further packages as wanted can be loaded. Below are just an example set. Note that template itself loads a number of packages, including hyperref.
% - References are handed using biblatex.
% - Don't need to include a \uomdeclarations unless this is a thesis
% - Note there is more documentation in the header of uom_these_casson.cls file if you need more help
% - Link to the presentation of theses policy: http://www.regulations.manchester.ac.uk/pgr-presentation-theses/

%%%%%%%%%%%%%%%%%% META DATA SETUP %%%%%%%%%%%%%%%%%%
% This is where the document title and author are set. Other details for the title page are set later
\begin{filecontents*}{\jobname.xmpdata}
  \Title{The Effects of Visual and Design Features on the Perception of Correlation in Scatterplots} 
  \Author{Gabriel Strain}
  \Language{en-GB}
  \Copyrighted{True}
  % More meta-data fielda can be added here if wanted, see https://ctan.org/pkg/pdfx?lang=en for fields
\end{filecontents*}

%%%%%%%%%%%%%%%%%% DOCUMENT SETUP %%%%%%%%%%%%%%%%%%
\documentclass[calibri,thesis]{uom_thesis_casson} % See above for font options Year can be: first, second or thesis.

%%%%%%%%%%%%%%%%%% PACKAGES AND COMMANDS %%%%%%%%%%%%%%%%%%

% Packages - some useful examples
\usepackage{amsmath}               % assumes amsmath package installed
  \allowdisplaybreaks[1]           % allow eqnarrays to break across pages
\usepackage{amssymb}               % assumes amsmath package installed 
\usepackage{url}                   % format hyperlinks correctly
\usepackage{rotating}              % allow portrait figures and tables
\usepackage{multirow}              % allows merging of rows in tables
\usepackage{lscape}                % allows pages to be typeset in landscape mode
\usepackage{tabularx}              % allows fixed width tables
\usepackage{verbatim}              % enhanced version of built-in verbatim environment
\usepackage{footnote}              % allows more control over footnote environments
%\usepackage{float}                 % allows H option on floats to force here placement
\usepackage{booktabs}              % improve table line spacing
\usepackage[base]{babel}           % required for lisum package
\usepackage{lipsum}                % for adding dummy text here
\usepackage{subcaption}
\captionsetup[subfigure]{justification=centering}
\captionsetup[subtable]{justification=centering}

% Force top alignment of subtables
\usepackage{floatrow}
\floatsetup[table]{style=Plaintop}

\usepackage{siunitx}               % add SI units
% Add your packages here
\usepackage{subfiles}
\usepackage{bibentry}
\usepackage{longtable}			   % more control of tables 

%\usepackage[printonlyused]{acronym}
\captionsetup{justification   = raggedright,
              singlelinecheck = false}
\usepackage{fancyref}

\usepackage{xcolor}

% Custom commands
\newcommand{\degree}{\ensuremath{^\circ}}
\newcommand{\sus}[1]{$^{\mbox{\scriptsize #1}}$} % superscript in text (e.g. 1st can be 1\sus{st})
\newcommand{\sub}[1]{$_{\mbox{\scriptsize #1}}$} % subscript in text
\newcommand{\chap}[1]{Chapter~\ref{#1}}
\newcommand{\sect}[1]{Section~\ref{#1}}
\newcommand{\fig}[1]{Figure~\ref{#1}}
\newcommand{\tab}[1]{Table~\ref{#1}}
\newcommand{\equ}[1]{Equation~(\ref{#1})}
\newcommand{\appx}[1]{Appendix~\ref{#1}}
% Add your commands here

%%%%%%%%%%%%%%%%%% REFERENCES SETUP %%%%%%%%%%%%%%%%%%

% Setup your references here. Change the reference style here if wanted
\usepackage[style=numeric-comp,backend=biber, maxbibnames = 99, maxcitenames=3, minnames=1, backref=true,hyperref=auto,natbib=true]{biblatex}
% Note backref=true adds a page number (and hyperlink) to each reference so you can easily go back from the references to the main document. You may prefer backref=false if you need to stick strictly to a given reference style

% Fixes which can't be applied in the .cls file
\DefineBibliographyStrings{english}{backrefpage = {cited on p\adddot},  backrefpages = {cited on pp\adddot}}
  \renewcommand*{\bibfont}{\large}

% Add more .bib files here if wanted
\addbibresource{thesis.bib}

% trying to address pandoc issue (12.05.2025)
\newcommand{\pandocbounded}[1]{#1}

%%%%%%%%%%%%%%%%%% START DOCUMENT %%%%%%%%%%%%%%%%%%
\begin{document}

%%%%%%%%%%%%%%%%%% TITLE PAGE %%%%%%%%%%%%%%%%%%

% Title and author are automatically taken from the document meta-data defined above
\makeatletter
\title{\xmp@Title}
\author{\xmp@Author}
\makeatother

% Set the below yourself
\faculty{Science and Engineering}                  % "Faculty of" is added automatically
\department{Department of Computer Science} % regulations allow School, Division, or Department to be put here
\submitdate{2025}                                  % regulations ask only for the year, not month
\wordcount{36,969}		                           % use \wordcount{} to set the count, \thewordcount to print in the text
\maketitle

%%%%%%%%%%%%%%%%%% LISTS OF CONTENT %%%%%%%%%%%%%%%%%%

% Probably don't need all of these unless final thesis
\uomtoc % contents 
\uomlof % figures
\uomlot % tables

%%%%%%%%%%%%%%%%%% ABSTRACT %%%%%%%%%%%%%%%%%%
\begin{abstract} % put abstract here. Limit is 1 page.
Data visualisations are a crucial part of making data accessible and interpretable. By leveraging human perceptual and cognitive systems, data visualisations can be designed that communicate information more accurately and efficiently than text and numbers. Despite their ubiquity, biases exist which cause viewers to make incorrect judgements about the levels of relatedness displayed in positively correlated scatterplots. This thesis investigates this underestimation bias, revealing insights into how changing simple visual features in scatterplots is able to affect perceptions and cognitions about correlations.

A set of four experiments in this paper explore the effects of changing the opacities and sizes of scatterplot points on participants' estimates of correlation. To summarise, it was found that uniformly reducing the opacity of scatterplot points could increase the correlation underestimation bias, that reducing opacity as a function of residual magnitude could increase estimates of correlation and partially correct for the bias, that reducing size using the same function could correct for the underestimation to a greater degree, and that combining both manipulations produced an overcorrection and consequent overestimation of correlation. The final experiment investigated whether the effects seen in the previous experiments could be extended into a cognitive space. By contextualising scatterplots as part of news items, evidence was provided that the previously established perceptual effects could also increase the degree to which participants changed their beliefs about a variable pair.

The work presented was conducted with a focus on reproducibility and open science. This includes open and public sharing of all data and code, and the facilitation of containerised environments to enable computational reproducibility.

The work in this thesis reveals how changing the opacities and sizes of points on scatterplots is able to affect viewers' perceptions of, and beliefs about, the levels of correlation between a pair of variables. These results provide insights into the nature of correlation perception, and provide guidance for those designing with the perception of positive correlation in mind.
\vspace{-1cm}
\end{abstract}%
%%%%%%%%%%%%%%%%%% DECLARATIONS %%%%%%%%%%%%%%%%%%
\uomdeclarations % Don't need unless final thesis. No options are needed. Having this command will add the required declarations

% %%%%%%%%%%%%%%%%%% LIST OF THESIS REVISIONS %%%%%%%%%%%%%%%%%%
% \begin{uomlotr} % Only required for resubmitted theses
% Put list of revisions here. Only required for resubmitted theses. Delete if not needed
% \end{uomlotr} 

%%%%%%%%%%%%%%%%%% ACKNOWLEDGEMENTS %%%%%%%%%%%%%%%%%%
\begin{uomacknowledgements} % probably don't need unless final thesis
None of the work described in this thesis took place in isolation, and none of it would have been possible without my family, friends, and colleagues; in many instances, happily, these identities intersect. The academic journey I took to arrive at this point was equal parts meandering and exhilarating. It required a strong foundation of support, which has come, over the years, from places too numerous to fully recount here; these acknowledgements are by no means exhaustive.

First and foremost I thank my wife, Hannah. Your love, support, and companionship are all I ever really needed. I thank my supervisors, Andrew Stewart, Caroline Jay, and Paul Warren. Your encouragement to always continue learning has been inspirational, and I consider myself extremely lucky to have received such a high level of personal, professional, and academic support; I look forward to many more years of collaboration and fun. I would like to thank everyone I worked with in the lab over the years, in particular David, Chris (\#1), Chris (\#2), Duncan, and Hamila. You made the lab a place I looked forward to being in, which in turn made my PhD a joy to complete. Special thanks goes to David; meeting one of my closest friends through postgraduate study has been the unexpected icing on an already-delicious cake. I would like to thank Jamie and David for their biweekly squash sessions; there are few things I look forward to more after spending the day staring at computer screens and thinking about scatterplots.

Finally, I would like to thank my family. From you I learnt curiosity, an appreciation of vigorous debate, and the need to not take anything in life too seriously.
\end{uomacknowledgements}

%%%%%%%% CHAPTERS %%%%%%%%%%%%%%%%%%
\chapter{Introduction} \label{chap:introduction}
\subfile{chapters_tex/chapters_quarto/1_introduction.tex}
\chapter{Related Work}\label{chap:related_work}
\subfile{chapters_tex/chapters_quarto/2_related_work.tex}
\chapter{General Methodology}\label{chap:gen_methods}
\subfile{chapters_tex/chapters_quarto/3_general_methodology.tex}
\chapter{Adjusting the Opacities of Scatterplot Points Can Affect Correlation Estimates}\label{chap:adjusting_opacity}
\subfile{chapters_tex/chapters_quarto/4_adjusting_opacity.tex}
\chapter{Adjusting the Sizes of Scatterplot Points Can Correct for a Historical Correlation Underestimation Bias}\label{chap:adjusting_size}
\subfile{chapters_tex/chapters_quarto/5_adjusting_size.tex}
\chapter{Interactions of Opacity and Size Adjustments}\label{chap:interactions_opacity_size}
\subfile{chapters_tex/chapters_quarto/6_interactions_opacity_size.tex}
\chapter{Visual Features Affecting Perceptual Estimates Also Affect Beliefs About Correlations}\label{chap:belief_change}
\subfile{chapters_tex/chapters_quarto/7_belief_change.tex}
\chapter{Conclusion}\label{chap:conclusion}
\subfile{chapters_tex/chapters_quarto/8_conclusion.tex}

%%%%%%%%%%%%%%%%%% REFERENCES %%%%%%%%%%%%%%%%%%
\printbibliography[title={References},heading=bibintoc] % a single list of references for the whole thesis

%%%%%%%%%%%%%%%%%% APPENDIX %%%%%%%%%%%%%%%%%%%%

\begin{uomappendix}
	\chapter{Full List of Statements in Chapter \ref{chap:belief_change}}
\begin{longtable}{@{}lllll@{}}
\toprule
Number & Statements                                                                                           &  &  &  \\* \midrule
\endhead
%
\bottomrule
\endfoot
%
\endlastfoot
%
1      & Higher consumption of fruits and vegetables is associated with lower   risks of heart disease.       &  &  &  \\
2      & Increased physical activity leads to improved mental health.                                         &  &  &  \\
3      & As educational attainment rises, unemployment rates typically decrease.                              &  &  &  \\
4      & Greater social media usage is linked with higher levels of anxiety.                                  &  &  &  \\
5      & Increased sugar intake is associated with a higher risk of type 2   diabetes.                        &  &  &  \\
6      & Higher levels of air pollution are correlated with increased respiratory   problems.                 &  &  &  \\
7      & More frequent reading in childhood is associated with better language   skills in adulthood.         &  &  &  \\
8      & As global temperatures rise, ice caps and glaciers melt at a faster rate.                            &  &  &  \\
9      & Greater water consumption is linked to improved kidney function.                                     &  &  &  \\
10     & Higher levels of job satisfaction are associated with lower employee   turnover rates.               &  &  &  \\
11     & As the amount of sleep decreases, the risk of obesity increases.                                     &  &  &  \\
12     & Increased exposure to sunlight is correlated with higher vitamin D   levels.                         &  &  &  \\
13     & Greater involvement in community activities is linked with higher levels   of happiness.             &  &  &  \\
14     & As caffeine consumption increases, so does the average heart rate.                                   &  &  &  \\
15     & Higher educational investments in a country are associated with better   economic growth.            &  &  &  \\
16     & Increased alcohol consumption is linked to higher risks of liver disease.                            &  &  &  \\
17     & As screen time increases, physical activity levels tend to decrease.                                 &  &  &  \\
18     & Greater intake of omega-3 fatty acids is associated with lower   inflammation levels.                &  &  &  \\
19     & Higher levels of parental involvement in education are correlated with   better student performance. &  &  &  \\
20     & Increased urbanization is linked to higher levels of air pollution.                                  &  &  &  \\
21     & As personal savings rates rise, financial stability tends to improve.                                &  &  &  \\
22     & Greater consumption of processed foods is associated with an increased   risk of chronic diseases.   &  &  &  \\
23     & As the use of renewable energy sources increases, greenhouse gas   emissions decrease.               &  &  &  \\
24     & Higher stress levels are linked to a greater risk of heart disease.                                  &  &  &  \\
25     & Increased time spent in nature is associated with improved mental health.                            &  &  &  \\
26     & As the population density increases, the spread of infectious diseases   becomes more likely.        &  &  &  \\
27     & Greater frequency of exercise is linked to a lower risk of depression.                               &  &  &  \\
28     & Higher dietary fiber intake is associated with lower rates of colorectal   cancer.                   &  &  &  \\
29     & As water scarcity increases, agricultural yields tend to decrease.                                   &  &  &  \\
30     & Increased use of antibiotics is linked to the development of   antibiotic-resistant bacteria.        &  &  &  \\
31     & Higher consumption of antioxidants is associated with a lower risk of   certain cancers.             &  &  &  \\
32     & As public transportation usage increases, urban air quality tends to   improve.                      &  &  &  \\
33     & Greater exposure to music education is linked to enhanced cognitive   development in children.       &  &  &  \\
34  & Higher levels of social support are associated with better recovery   outcomes for mental health conditions.          &  &  &  \\
35     & Increased consumption of fast food is linked to obesity.                                             &  &  &  \\
36     & As voter turnout increases, the representation of public interests in   government may improve.      &  &  &  \\
37     & Greater use of helmets is associated with a lower incidence of head   injuries in cyclists.          &  &  &  \\
38     & Higher minimum wages are linked to reduced poverty rates.                                            &  &  &  \\
39     & As the frequency of handwashing increases, the spread of common illnesses   decreases.               &  &  &  \\
40     & Increased green space in urban areas is associated with lower stress   levels among residents.       &  &  &  \\
41     & Higher intake of trans fats is linked to an increased risk of heart   disease.                       &  &  &  \\
42     & As the quality of healthcare improves, life expectancy tends to increase.                            &  &  &  \\
43     & Greater exposure to secondhand smoke is associated with higher risks of   respiratory diseases.      &  &  &  \\
44     & Higher levels of job autonomy are linked to increased job satisfaction.                              &  &  &  \\
45     & As access to clean water improves, the incidence of waterborne diseases   decreases.                 &  &  &  \\
46     & Increased participation in team sports is associated with better social   skills in children.        &  &  &  \\
47     & Higher alcohol taxes are linked to lower rates of alcohol-related harm.                              &  &  &  \\
48     & As the number of trees in an area increases, air quality tends to   improve.                         &  &  &  \\
49     & Greater use of digital devices before bedtime is associated with poorer   sleep quality.             &  &  &  \\
50     & Higher levels of empathy are linked to stronger interpersonal   relationships.                       &  &  &  \\
51     & As community safety improves, property values tend to increase.                                      &  &  &  \\
52     & Increased intake of saturated fats is associated with higher cholesterol   levels.                   &  &  &  \\
53     & Greater frequency of meditation is linked to reduced stress levels.                                  &  &  &  \\
54     & As public spending on education increases, literacy rates tend to   improve.                         &  &  &  \\
55     & Higher exposure to air conditioning is associated with increased   respiratory issues.               &  &  &  \\
56     & Increased levels of physical fitness are linked to lower mortality rates.                            &  &  &  \\
57     & As soil quality degrades, agricultural productivity tends to decrease.                               &  &  &  \\
58     & Greater consumption of red meat is associated with an increased risk of   heart disease.             &  &  &  \\
59     & Higher levels of civic engagement are linked to a stronger sense of   community.                     &  &  &  \\
60     & As the amount of green space increases, urban temperatures tend to   decrease.                       &  &  &  \\
61     & Increased exposure to violent media is associated with higher aggression   levels.                   &  &  &  \\
62     & Greater intake of calcium is linked to improved bone health.                                         &  &  &  \\
63     & As diversity in the workplace increases, innovation and creativity may   improve.                    &  &  &  \\
64     & Higher sugar consumption is associated with an increased risk of dental   cavities.                  &  &  &  \\
65     & Increased mindfulness practice is linked to lower anxiety levels.                                    &  &  &  \\
66     & As the rate of deforestation increases, biodiversity tends to decrease.                              &  &  &  \\
67     & Greater levels of trust in society are associated with lower crime rates.                            &  &  &  \\
68     & Higher frequency of family meals is linked to better eating habits in   children.                    &  &  &  \\
69  & Increased physical proximity to parks and recreational areas is   associated with higher levels of physical activity. &  &  &  \\
70     & As participation in community arts programs increases, local cultural   engagement tends to rise.    &  &  &  \\
71     & Greater daily water intake is linked to enhanced skin hydration and   appearance.                    &  &  &  \\
72     & Higher intake of vitamin C is associated with a reduced duration of the   common cold.               &  &  &  \\
73     & As the use of public transit increases, traffic congestion tends to   decrease.                      &  &  &  \\
74     & Increased levels of financial literacy are linked to better personal   finance management.           &  &  &  \\
75     & Higher consumption of spicy foods is associated with a lower risk of   certain types of cancer.      &  &  &  \\
76     & As indoor air quality improves, asthma symptoms tend to decrease.                                    &  &  &  \\
77     & Greater social connectivity is linked to lower risks of dementia in older   adults.                  &  &  &  \\
78     & Higher levels of bilingual education are associated with improved   cognitive flexibility.           &  &  &  \\
79     & Increased frequency of laughter is linked to improved immune system   function.                      &  &  &  \\
80     & As the amount of recyclable waste increases, the environmental impact of   waste decreases.          &  &  &  \\
81     & Greater exposure to diverse cultures is associated with more open-minded   attitudes.                &  &  &  \\
82     & Higher attendance at preventive health screenings is linked to earlier   detection of diseases.      &  &  &  \\
83     & As pet ownership increases, levels of stress and loneliness tend to   decrease.                      &  &  &  \\
84     & Increased consumption of whole grains is associated with lower risks of   heart disease.             &  &  &  \\
85  & Higher engagement with science and technology education is linked to   increased innovation in societies.             &  &  &  \\
86     & As awareness of mental health issues increases, stigma tends to decrease.                            &  &  &  \\
87     & Greater time spent on hobbies is associated with higher levels of life   satisfaction.               &  &  &  \\
88     & Higher levels of community greenery are linked to reduced urban heat   island effect.                &  &  &  \\
89     & As exposure to natural light during the day increases, sleep quality   tends to improve.             &  &  &  \\
90     & Increased use of energy-efficient appliances is associated with lower   electricity bills.           &  &  &  \\
91     & Greater adherence to a Mediterranean diet is linked to a lower risk of   neurodegenerative diseases. &  &  &  \\
92     & As pedestrian-friendly infrastructure improves, urban walkability tends   to increase.               &  &  &  \\
93     & Higher consumption of nuts and seeds is associated with reduced risk of   cardiovascular diseases.   &  &  &  \\
94     & Increased engagement in volunteer work is linked to a greater sense of   purpose and well-being.     &  &  &  \\
95     & As the availability of affordable housing increases, homelessness tends   to decrease.               &  &  &  \\
96     & Greater regularity in sleep patterns is associated with improved mental   health.                    &  &  &  \\
97     & Higher intake of probiotics is linked to better gut health.                                          &  &  &  \\
98     & As cultural preservation efforts increase, community identity and   cohesion tend to strengthen.     &  &  &  \\
99     & Increased practice of gratitude is associated with higher levels of   happiness and optimism.        &  &  &  \\
100 & Higher exposure to interactive educational activities is linked to   enhanced learning outcomes for children.         &  &  &  \\* \bottomrule
\end{longtable}
\end{uomappendix}
%%%%%%%%%%%%%%%%%% END MATTER %%%%%%%%%%%%%%%%%%
\end{document}