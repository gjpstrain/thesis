%%%%%%%%%%%%%%%%%% USAGE INSTRUCTIONS - from original template %%%%%%%%%%%%%%%%%%
% - Compile using LuaLaTeX and biber, unless there is a particular reason not to. Do not use the older LaTex/PDFLaTeX or BibTeX (the fonts won't work correctly.)
% - Font and the report 'year' must be specified when all \documentclass or the template won't work correctly. (There's no error checking/default cases!)
% - Options for fonts are: calibri, times, palatino, garamond, arial, tahoma, verdana, trebuchet. Note however that not all these are installed on Overleaf, you need to install to your project 
% - Options for 'year' are: first, second, thesis.
% - If not a thesis, should probably remove the COVID-19 impact statement page
% - As many further packages as wanted can be loaded. Below are just an example set. Note that template itself loads a number of packages, including hyperref.
% - References are handed using biblatex.
% - Don't need to include a \uomdeclarations unless this is a thesis
% - Note there is more documentation in the header of uom_these_casson.cls file if you need more help
% - Link to the presentation of theses policy: http://www.regulations.manchester.ac.uk/pgr-presentation-theses/



%%%%%%%%%%%%%%%%%% META DATA SETUP %%%%%%%%%%%%%%%%%%
% This is where the document title and author are set. Other details for the title page are set later
\begin{filecontents*}{\jobname.xmpdata}
  \Title{The Effects of Visual and Design Features on the Perception of Correlation in Scatterplots} 
  \Author{Gabriel Strain}
  \Language{en-GB}
  \Copyrighted{True}
  % More meta-data fielda can be added here if wanted, see https://ctan.org/pkg/pdfx?lang=en for fields
\end{filecontents*}



%%%%%%%%%%%%%%%%%% DOCUMENT SETUP %%%%%%%%%%%%%%%%%%
\documentclass[calibri,thesis]{uom_thesis_casson} % See above for font options Year can be: first, second or thesis.



%%%%%%%%%%%%%%%%%% PACKAGES AND COMMANDS %%%%%%%%%%%%%%%%%%

% Packages - some useful examples
\usepackage{amsmath}               % assumes amsmath package installed
  \allowdisplaybreaks[1]           % allow eqnarrays to break across pages
\usepackage{amssymb}               % assumes amsmath package installed 
\usepackage{url}                   % format hyperlinks correctly
\usepackage{rotating}              % allow portrait figures and tables
\usepackage{multirow}              % allows merging of rows in tables
\usepackage{lscape}                % allows pages to be typeset in landscape mode
\usepackage{tabularx}              % allows fixed width tables
\usepackage{verbatim}              % enhanced version of built-in verbatim environment
\usepackage{footnote}              % allows more control over footnote environments
%\usepackage{float}                 % allows H option on floats to force here placement
\usepackage{booktabs}              % improve table line spacing
\usepackage[base]{babel}           % required for lisum package
\usepackage{lipsum}                % for adding dummy text here
\usepackage{subcaption}
\captionsetup[subfigure]{justification=centering}
\captionsetup[subtable]{justification=centering}

% Force top alignment of subtables
\usepackage{floatrow}
\floatsetup[table]{style=Plaintop}

\usepackage{siunitx}               % add SI units
% Add your packages here
\usepackage{subfiles}
\usepackage{bibentry}
\usepackage{longtable}			   % more control of tables 

%\usepackage[printonlyused]{acronym}
\captionsetup{justification   = raggedright,
              singlelinecheck = false}
\usepackage{fancyref}

\usepackage{xcolor}

% Custom commands
\newcommand{\degree}{\ensuremath{^\circ}}
\newcommand{\sus}[1]{$^{\mbox{\scriptsize #1}}$} % superscript in text (e.g. 1st can be 1\sus{st})
\newcommand{\sub}[1]{$_{\mbox{\scriptsize #1}}$} % subscript in text
\newcommand{\chap}[1]{Chapter~\ref{#1}}
\newcommand{\sect}[1]{Section~\ref{#1}}
\newcommand{\fig}[1]{Figure~\ref{#1}}
\newcommand{\tab}[1]{Table~\ref{#1}}
\newcommand{\equ}[1]{Equation~(\ref{#1})}
\newcommand{\appx}[1]{Appendix~\ref{#1}}
% Add your commands here



%%%%%%%%%%%%%%%%%% REFERENCES SETUP %%%%%%%%%%%%%%%%%%

% Setup your references here. Change the reference style here if wanted
\usepackage[style=numeric-comp,backend=biber, maxbibnames = 99, maxcitenames=3, minnames=1, backref=true,hyperref=auto,natbib=true]{biblatex}
% Note backref=true adds a page number (and hyperlink) to each reference so you can easily go back from the references to the main document. You may prefer backref=false if you need to stick strictly to a given reference style


% Fixes which can't be applied in the .cls file
\DefineBibliographyStrings{english}{backrefpage = {cited on p\adddot},  backrefpages = {cited on pp\adddot}}
  \renewcommand*{\bibfont}{\large}


% Add more .bib files here if wanted
\addbibresource{thesis.bib}

% trying to address pandoc issue (12.05.2025)
\newcommand{\pandocbounded}[1]{#1}


%%%%%%%%%%%%%%%%%% START DOCUMENT %%%%%%%%%%%%%%%%%%
\begin{document}

%%%%%%%%%%%%%%%%%% TITLE PAGE %%%%%%%%%%%%%%%%%%

% Title and author are automatically taken from the document meta-data defined above
\makeatletter
\title{\xmp@Title}
\author{\xmp@Author}
\makeatother

% Set the below yourself
\faculty{Science and Engineering}                  % "Faculty of" is added automatically
\department{Department of Computer Science} % regulations allow School, Division, or Department to be put here
\submitdate{2024}                                  % regulations ask only for the year, not month
\wordcount{1000}		                           % use \wordcount{} to set the count, \thewordcount to print in the text
\maketitle



%%%%%%%%%%%%%%%%%% LISTS OF CONTENT %%%%%%%%%%%%%%%%%%

% Probably don't need all of these unless final thesis
\uomtoc % contents 
\uomlof % figures
\uomlot % tables

%%%%%%%%%%%%%%%%%% ABSTRACT %%%%%%%%%%%%%%%%%%
\begin{abstract} % put abstract here. Limit is 1 page.
Data visualisations are a crucial part of making data accessible and interpretable. By leveraging human perceptual and cognitive systems, data visualisations can be designed that communicate information more accurately and efficiently than text and numbers. Despite their ubiquity, biases exist which cause viewers to make incorrect judgements about the levels of relatedness displayed in positively correlated scatterplots. This thesis investigates this underestimation bias, revealing insights into how changing simple visual features in scatterplots is able to affect perceptions and cognitions about correlations.

A set of four experiments in this paper explore the effects of changing the opacities and sizes of scatterplot points on participants' estimates of correlation. To summarise, it was found that uniformly reducing the opacity of scatterplot points could increase the correlation underestimation bias, that reducing opacity as a function of residual magnitude could increase estimates of correlation and partially correct for the bias, that reducing size using the same function could correct for the underestimation to a greater degree, and that combining both manipulations produced an overcorrection and consequent overestimation of correlation. The final experiment investigated whether the effects seen in the previous experiments could be extended into a cognitive space. By contextualising scatterplots as part of news items, evidence was provided that the previously established perceptual effects could also increase the degree to which participants changed their beliefs about a variable pair.

The work presented was conducted with a focus on reproducibility and open science. This includes open and public sharing of all data and code, and the facilitation of containerised environments to enable computational reproducibility.

The work in this thesis reveals how changing the opacities and sizes of points on scatterplots is able to affect viewers' perceptions of, and beliefs about, the levels of correlation between a pair of variables. These results provide insights into the nature of correlation perception, and provide guidance for those designing with the perception of positive correlation in mind.
\vspace{-1cm}
\end{abstract}%
%%%%%%%%%%%%%%%%%% DECLARATIONS %%%%%%%%%%%%%%%%%%
\uomdeclarations % Don't need unless final thesis. No options are needed. Having this command will add the required declarations

% %%%%%%%%%%%%%%%%%% LIST OF THESIS REVISIONS %%%%%%%%%%%%%%%%%%
% \begin{uomlotr} % Only required for resubmitted theses
% Put list of revisions here. Only required for resubmitted theses. Delete if not needed
% \end{uomlotr} 



%%%%%%%%%%%%%%%%%% ACKNOWLEDGEMENTS %%%%%%%%%%%%%%%%%%
\begin{uomacknowledgements} % probably don't need unless final thesis
None of the work described in this thesis took place in isolation, and none of it would have been possible without my family, friends, and colleagues; in many instances, happily, these identities intersect. The academic journey I took to arrive at this point was equal parts meandering and exhilarating. It required a strong foundation of support, which has come, over the years, from places too numerous to fully recount here; these acknowledgements are by no means exhaustive.

First and foremost I thank my wife, Hannah. Your love, support, and companionship are all I ever really needed. I thank my supervisors, Andrew Stewart, Caroline Jay, and Paul Warren. Your encouragement to always continue learning has been inspirational, and I consider myself extremely lucky to have received such a high level of personal, professional, and academic support; I look forward to many more years of collaboration and fun. I would like to thank everyone I worked with in the lab over the years, in particular David, Chris (\#1), Chris (\#2), and Hamila. You made the lab a place I looked forward to being in, which in turn made my PhD a joy to complete. Special thanks goes to David; meeting one of my closest friends through postgraduate study has been the unexpected icing on an already-delicious cake. I would like to thank Jamie and David for their biweekly squash sessions; there are few things I look forward to more after spending the day staring at computer screens and thinking about scatterplots.

Finally, I would like to thank my family. From you I learnt curiosity, an appreciation of vigorous debate, and the need to not take anything in life too seriously.
\end{uomacknowledgements}

%%%%%%%% CHAPTERS %%%%%%%%%%%%%%%%%%
\chapter{Introduction} \label{chap:introduction}
\subfile{chapters_tex/chapters_quarto/1_introduction.tex}
\chapter{Related Work}\label{chap:related_work}
\subfile{chapters_tex/chapters_quarto/2_related_work.tex}
\chapter{General Methodology}\label{chap:gen_methods}
\subfile{chapters_tex/chapters_quarto/3_general_methodology.tex}
\chapter{Adjusting the Opacities of Scatterplot Points Can Affect Correlation Estimates}\label{chap:adjusting_opacity}
\subfile{chapters_tex/chapters_quarto/4_adjusting_opacity.tex}
\chapter{Adjusting the Sizes of Scatterplot Points Can Correct for a Historical Correlation Underestimation Bias}\label{chap:adjusting_size}
\subfile{chapters_tex/chapters_quarto/5_adjusting_size.tex}
\chapter{Interactions of Opacity and Size Adjustments}\label{chap:interactions_opacity_size}
\subfile{chapters_tex/chapters_quarto/6_interactions_opacity_size.tex}
\chapter{Visual Features Affecting Perceptual Estimates Also Affect Beliefs About Correlations}\label{chap:belief_change}
\subfile{chapters_tex/chapters_quarto/7_belief_change.tex}
\chapter{Conclusion}\label{chap:conclusion}
\subfile{chapters_tex/chapters_quarto/8_conclusion.tex}

%%%%%%%%%%%%%%%%%% REFERENCES %%%%%%%%%%%%%%%%%%
\printbibliography[title={References},heading=bibintoc] % a single list of references for the whole thesis



%%%%%%%%%%%%%%%%%% APPENDICES %%%%%%%%%%%%%%%%%%
\begin{uomappendix} 
  \chapter{Appendix A}
\end{uomappendix}


%%%%%%%%%%%%%%%%%% END MATTER %%%%%%%%%%%%%%%%%%
\end{document}