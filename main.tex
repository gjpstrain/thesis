%%%%%%%%%%%%%%%%%% USAGE INSTRUCTIONS %%%%%%%%%%%%%%%%%%
% - Compile using LuaLaTeX and biber, unless there is a particular reason not to. Do not use the older LaTex/PDFLaTeX or BibTeX (the fonts won't work correctly.)
% - Font and the report 'year' must be specified when all \documentclass or the template won't work correctly. (There's no error checking/default cases!)
% - Options for fonts are: calibri, times, palatino, garamond, arial, tahoma, verdana, trebuchet. Note however that not all these are installed on Overleaf, you need to install to your project 
% - Options for 'year' are: first, second, thesis.
% - If not a thesis, should probably remove the COVID-19 impact statement page
% - As many further packages as wanted can be loaded. Below are just an example set. Note that template itself loads a number of packages, including hyperref.
% - References are handed using biblatex.
% - Don't need to include a \uomdeclarations unless this is a thesis
% - Note there is more documentation in the header of uom_these_casson.cls file if you need more help
% - Link to the presentation of theses policy: http://www.regulations.manchester.ac.uk/pgr-presentation-theses/



%%%%%%%%%%%%%%%%%% META DATA SETUP %%%%%%%%%%%%%%%%%%
% This is where the document title and author are set. Other details for the title page are set later
\begin{filecontents*}{\jobname.xmpdata}
  \Title{Optimising Virtual Environments in Real-Time Using Perceptually-Based Rendering} 
  \Author{David G. Petrescu}
  \Language{en-GB}
  \Copyrighted{True}
  % More meta-data fielda can be added here if wanted, see https://ctan.org/pkg/pdfx?lang=en for fields
\end{filecontents*}



%%%%%%%%%%%%%%%%%% DOCUMENT SETUP %%%%%%%%%%%%%%%%%%
\documentclass[times,thesis]{uom_thesis_casson} % See above for font options Year can be: first, second or thesis.



%%%%%%%%%%%%%%%%%% PACKAGES AND COMMANDS %%%%%%%%%%%%%%%%%%

% Packages - some useful examples
\usepackage{graphicx,psfrag,color} % for postscript graphics files
  \graphicspath{ {./images/FovWalking}{./images/LoDVel}}
\usepackage{amsmath}               % assumes amsmath package installed
  \allowdisplaybreaks[1]           % allow eqnarrays to break across pages
\usepackage{amssymb}               % assumes amsmath package installed 
\usepackage{url}                   % format hyperlinks correctly
\usepackage{rotating}              % allow portrait figures and tables
\usepackage{multirow}              % allows merging of rows in tables
\usepackage{lscape}                % allows pages to be typeset in landscape mode
\usepackage{tabularx}              % allows fixed width tables
\usepackage{verbatim}              % enhanced version of built-in verbatim environment
\usepackage{footnote}              % allows more control over footnote environments
\usepackage{float}                 % allows H option on floats to force here placement
\usepackage{booktabs}              % improve table line spacing
\usepackage[base]{babel}           % required for lisum package
\usepackage{lipsum}                % for adding dummy text here
\usepackage{subcaption}            % for multiple sub-figures in a single float
\usepackage{siunitx}               % add SI units
% Add your packages here
\usepackage{subfiles}
\usepackage{bibentry}

%\usepackage[printonlyused]{acronym}
\input{acros}

\usepackage{fancyref}

\usepackage{xcolor}
\usepackage{tikz}

\usetikzlibrary{arrows.meta}
\tikzset{%
  >={Latex[width=5mm,length=5mm]},
  % Specifications for style of nodes:
            base/.style = {rectangle, rounded corners, draw=black,
                           minimum width=6cm, minimum height=1.5cm,
                           text centered},
  activityStarts/.style = {base, fill=blue!20},
       startstop/.style = {base, fill=red!20},
    activityRuns/.style = {base, fill=green!19},
         process/.style = {base, minimum width=2.5cm, fill=orange!15
                           },
}



% Custom commands
\newcommand{\degree}{\ensuremath{^\circ}}
\newcommand{\sus}[1]{$^{\mbox{\scriptsize #1}}$} % superscript in text (e.g. 1st can be 1\sus{st})
\newcommand{\sub}[1]{$_{\mbox{\scriptsize #1}}$} % subscript in text
\newcommand{\chap}[1]{Chapter~\ref{#1}}
\newcommand{\sect}[1]{Section~\ref{#1}}
\newcommand{\fig}[1]{Figure~\ref{#1}}
\newcommand{\tab}[1]{Table~\ref{#1}}
\newcommand{\equ}[1]{Equation~(\ref{#1})}
\newcommand{\appx}[1]{Appendix~\ref{#1}}
% Add your commands here



%%%%%%%%%%%%%%%%%% REFERENCES SETUP %%%%%%%%%%%%%%%%%%

% Setup your references here. Change the reference style here if wanted
\usepackage[style=numeric-comp,backend=biber, maxbibnames = 99, maxcitenames=3, minnames=1, backref=true,hyperref=auto,natbib=true]{biblatex}
% Note backref=true adds a page number (and hyperlink) to each reference so you can easily go back from the references to the main document. You may prefer backref=false if you need to stick strictly to a given reference style


% Fixes which can't be applied in the .cls file
\DefineBibliographyStrings{english}{backrefpage = {cited on p\adddot},  backrefpages = {cited on pp\adddot}}
  \renewcommand*{\bibfont}{\large}


% Add more .bib files here if wanted
\addbibresource{references.bib}
\addbibresource{phd.bib}




%%%%%%%%%%%%%%%%%% START DOCUMENT %%%%%%%%%%%%%%%%%%
\begin{document}



%%%%%%%%%%%%%%%%%% COVID-19 impact statement %%%%%%%%%%%%%%%%%%
\begin{uomcovid} % policy asks for statement to be before the title page
  % Policy asks for this to be removed from the final thesis post-examination, which messes up the page numbers somewhat. Done here so that the final post-examination thesis is correct. In the pre-examination thesis the numbers displayed on the page will be one lower than the number displayed by the PDF reader. Remove this page if not wanted/needed, and/or is a first or second year report
  Optional. Delete this section if not needed. This is COVID-19 impact text.
  
  \lipsum[1-3] % generate dummy text for here
\end{uomcovid}% \clearpage is added automatically



%%%%%%%%%%%%%%%%%% TITLE PAGE %%%%%%%%%%%%%%%%%%

% Title and author are automatically taken from the document meta-data defined above
\makeatletter
\title{\xmp@Title}
\author{\xmp@Author}
\makeatother

% Set the below yourself
\faculty{Science and Engineering}                  % "Faculty of" is added automatically
\department{Department of Computer Science} % regulations allow School, Division, or Department to be put here
\submitdate{2024}                                  % regulations ask only for the year, not month
\wordcount{1000}		                           % use \wordcount{} to set the count, \thewordcount to print in the text
\maketitle



%%%%%%%%%%%%%%%%%% LISTS OF CONTENT %%%%%%%%%%%%%%%%%%

% Probably don't need all of these unless final thesis
\uomtoc % contents 
\uomlof % figures
\uomlot % tables
\begin{uomlop} % list of publications.
  % Can use biblatex to \printbibliography[heading=none] to populate automatically, or can add a custom list via the tocloft package, but probably easier to just type in unless you have lots!
  Publications go here.
\end{uomlop}


    
\begin{uomterms}
\printacronyms[heading=empty]
\end{uomterms}



%%%%%%%%%%%%%%%%%% ABSTRACT %%%%%%%%%%%%%%%%%%
\begin{abstract} % put abstract here. Limit is 1 page.

Rendering realistic virtual environments require intense computation that will always scale to the zeitgeist of what users consider high-quality. With the renewed interest in Virtual Reality and display technologies becoming more proficient, computational requirements will increase even further. However, human perception is limited, which means that high-quality rendering is often unnecessary as it does not affect the visual experience of the user. 
The aim of this thesis is to explore aspects of human perception that have the potential of optimising rendering which have traditionally received less attention from the graphics community. An extensive framework of perceptual rendering will be introduced which reveals promising overlapping areas which have not been explored. To this extent, the relation between the perception of causality and other factors that limit visual acuity such as retinal eccentricity, velocity and area will be disentangled. The rest of the thesis is concerned with how self-induced movement in Virtual Reality (VR) affects sensitivity to changes in geometrical Level-of-Detail (LOD) and Foveated Rendering (FR). With respect to the latter, this thesis shows that current FR algorithms can be significantly improved if users engage in active movement, which is a dimension characteristic of VR usage.  
  
  \lipsum[1-2]
\end{abstract}%
\clearpage



%%%%%%%%%%%%%%%%%% LAY ABSTRACT %%%%%%%%%%%%%%%%%%
\begin{uomlay} % put lay abstract here. Limit is 1 page. Not compulsory
  This is lay abstract text. 
  
  \lipsum[1-2]
\end{uomlay}



%%%%%%%%%%%%%%%%%% DECLARATIONS %%%%%%%%%%%%%%%%%%
\uomdeclarations % Don't need unless final thesis. No options are needed. Having this command will add the required declarations



% %%%%%%%%%%%%%%%%%% LIST OF THESIS REVISIONS %%%%%%%%%%%%%%%%%%
% \begin{uomlotr} % Only required for resubmitted theses
% Put list of revisions here. Only required for resubmitted theses. Delete if not needed
% \end{uomlotr} 



%%%%%%%%%%%%%%%%%% ACKNOWLEDGEMENTS %%%%%%%%%%%%%%%%%%
\begin{uomacknowledgements} % probably don't need unless final thesis
Acknowledgements go here.
\end{uomacknowledgements}

%%%%%%%% CHAPTERS %%%%%%%%%%%%%%%%%%
\chapter{Introduction} 
\subfile{chapters/introduction}
\chapter{Background}\label{chap:background}
\subfile{chapters/LitReview}
\chapter{General Methods}\label{chap:methods}
\subfile{chapters/methods.tex}
\chapter{Hipster Michotte}\label{chap:Michotte}
\subfile{chapters/causalityChap}
\chapter{Velocity LOD}\label{chap:velLOD}
\subfile{chapters/velocityLOD}
\chapter{Foveated Walking}\label{chap:fovWalking}
\subfile{chapters/foveatedWalking}
\chapter{Thinking on Your Feet}\label{chap:TOYF}
\subfile{chapters/toyf}
\chapter{Conclusion and Future Work}\label{chap:conclusion}
\subfile{chapters/conclusion}



%%%%%%%%%%%%%%%%%% REFERENCES %%%%%%%%%%%%%%%%%%
\printbibliography[title={References},heading=bibintoc] % a single list of references for the whole thesis



%%%%%%%%%%%%%%%%%% APPENDICES %%%%%%%%%%%%%%%%%%
\begin{uomappendix} 
  \chapter{First appendix}
    \section{Section in Appendix}
    \lipsum[1-6]
\end{uomappendix}


%%%%%%%%%%%%%%%%%% END MATTER %%%%%%%%%%%%%%%%%%
\end{document}